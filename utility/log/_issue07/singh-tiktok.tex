\setvariables[article][shortauthor={Singh}, date={May 2023}, issue={7}, DOI={10.7916/archipelagos-0704}]

\setupinteraction[title={“I know the world by how I speak the world”: TikTok ABCs, Disaster Language, and Andrew Salkey's *Hurricane*},author={Kris Singh}, date={May 2023}, subtitle={“I know the world by how I speak the world”}, state=start]
\environment env_journal


\starttext


\startchapter[title={\quotation{I know the world by how I speak the world}: TikTok ABCs, Disaster Language, and Andrew Salkey's {\em Hurricane}}
, marking={\quotation{I know the world by how I speak the world}}
, bookmark={“I know the world by how I speak the world”: TikTok ABCs, Disaster Language, and Andrew Salkey's *Hurricane*}]


\startlines
{\bf
Kris Singh
}
\stoplines


{\startnarrower\it This paper considers how Caribbean ways of speaking map onto social media by analyzing two popular TikTok series as well as the testimony of a survivor of the 2021 eruption of La Soufrière that was widely circulated on Facebook. Understanding what Caribbean cultural production on social media entails means appreciating social media platforms like TikTok as commercial entities. As such, they define what can be produced and influence what audiences are likely to register as funny. Though Caribbean people's representations of themselves may seem more accessible than ever, the ways in which we hear each other on social media remains unevenly defined. Juxtaposing this consideration of the Caribbean digital landscape with Andrew Salkey's {\em Hurricane}, I argue that the connotations that social media usages of vernacular have accumulated risk compromising the opportunities for ideological inspection that fictional and non-fictional representations of disaster make possible.

 \stopnarrower}

\blank[2*line]
\blackrule[width=\textwidth,height=.01pt]
\blank[2*line]

The \quotation{Trini Alphabet Series} was a Caribbean social media sensation created in 2020 by Stephon Felmine (@stephon23). In this series of videos, Felmine assumes the role of a teacher rehearsing the ABCs for fictive schoolchildren. With ruler in hand and an authoritative thwack to grab your attention, Felmine introduces the letter of the day at the start of each video and offers a single word whose meaning he then clarifies with one or two inventive examples. Notable in his ability to cherry-pick loaded local lingo not typically invoked for pedagogy, Felmine offers selections like \quotation{L is for Lickrish} and \quotation{M is for Mamaguy} so that viewers are immediately primed for guffaws, but Felmine's expressive articulation and practiced schoolteacher mien are what sell his skits.\footnote{Stephon Felmine, "Trini Alphabet Series: L is for Lickrish," TikTok Video, Oct 5, 2020 https://web.archive.org/web/20210521212749/https://www.tiktok.com/@stephon23/video/6880263506437033217

  Stephon Felmine, "Trini Alphabet Series: M is for Mamaguy," TikTok Video, Oct 6, 2020 https://web.archive.org/web/20210521212940/https://www.tiktok.com/@stephon23/video/6880626990009847041} \quotation{B is for Braggadang,} Felmine says earnestly in the second video of this series before clarifying, \quotation{Braggadang means old or in a state of disrepair. For example, my neighbour making a whole set of ole noise with he braggadang car.}\footnote{Stephon Felmine, "Trini Alphabet Series: B is for Braggadang," TikTok Video, Sept 25, 2020 https://web.archive.org/web/20210517124550/https://www.tiktok.com/@stephon23/video/6876543232352750849} Felmine manages to maintain accuracy in his definitions and examples, conjuring plausible but comical scenes and only slightly exaggerating local speech rhythms. His intonation and pacing are skillful such that nuances of minor expressions like \quotation{eh-heh} (in the video \quotation{E is for Eh-heh}) are recognizable.\footnote{Stephon Felmine, "Trini Alphabet Series: E is for Eh-heh," TikTok Video, Sept 29, 2020 https://web.archive.org/web/20210517152619/https://www.tiktok.com/@stephon23/video/6877648583332482305}

Felmine's performances and their popularity engage the processes whereby languages are (de)legitimized and bring to mind the many Caribbean writers who have long explored these processes. In {\em The Cartographer Tries to Map a Way to Zion}, for instance, the poet Kei Miller juxtaposes the formal English and scientific method of a cartographer with the language and philosophy of a Rastaman.\footnote{Kei Miller, {\em The Cartographer Tries to Map a Way to Zion} (Manchester: Carcanet Press, 2014).} Miller explains, \quotation{The whole book is interested obviously in maps and cartography and the things that maps miss, but in a larger sense it's interested in language.}\footnote{Kei Miller, "An Interview with Kei Miller," interview by Eleanor Wachtel, {\em Brick: A Literary Journal} (2018): no pages. https://web.archive.org/web/20210506182419/https://brickmag.com/an-interview-with-kei-miller/} Miller's poetry suggests that understanding why certain ways of speaking have been marginalized and others naturalized involves investigating not just histories of inclusions and exclusions, but also occlusions. Referring to Sam Selvon's novel {\em The Lonely Londoners} and how its use of Caribbean ways of speaking \quotation{renders London in a new language,} Miller emphasizes, \quotation{I know the world by how I speak the world.}\footnote{Miller, "An Interview with Kei Miller."} The languages employed in any place can inhibit, just as they can transform, our knowing; the dissonance that comes from their juxtaposition can be revelatory.

The work of Selvon and his contemporaries as well as the more recent contributions of writers like Miller explore the epistemologies embedded in Caribbean vernacular.\footnote{Given the shifting terminology for local forms of speech across the Caribbean, I use the term vernacular to refer to Caribbean ways of speaking aside from what is formally recognized as Standard English or popularly referred to as proper English.} Felmine is a beneficiary of these writers, but he also satisfies an appetite on social media for cultural showcases. On TikTok, for instance, a variety of users attract views and followers by offering introductions to language that are specific to their cultures, communities, or nations. An established celebrity like Sean Paul (@duttypaul) seeks to maintain the brand he has established through his music by creating TikTok videos that explain Jamaican English. In one post titled \quotation{Jamaican Lingo Pt. 1,} he loosely defines \quotation{big up} and includes the hashtags \#jamaicanlingo \#patois \#patwa \#howtochatjamaican.\footnote{Sean Paul, "Jamaican Lingo Pt. 1," TikTok Video (Apr 7, 2020), https://web.archive.org/web/20211222014609/https://www.tiktok.com/@duttypaul/video/6948484636750220549} Paul simplifies Jamaican English to basic vocabulary and positions himself as a figure of linguistic authority who grants his followers access to fluency. Likewise, less established users turn to this trend in an effort to build their following. An aspiring rapper named Tenn Buick (@tennbuick), for instance, offers a multi-part series called the \quotation{Hindi/Guyanese Creole Challenge} that features him and his partner comparing words from their respective communities in quick succession.\footnote{Tenn Buick, "Hindi/Guyanese Creole Challenge Pt. 7," TikTok Video (Oct 20, 2020), https://web.archive.org/web/20211222014832/https://www.tiktok.com/@tennbuick/video/6885832612460416262} Covering multiple words per video, they offer little explanation or commentary, such that the exclusivity of the comparison appears to be the main appeal.

This pattern of performances prompts this article's consideration of how Caribbean speech maps onto the digital landscape. I begin by analyzing Felmine's TikTok series, which centers Trinidadian English, alongside its predecessor, a TikTok series by Kwame \quote{Wello} Simpson (@wellonyc), which centers Guyanese English. Acknowledging the use of irony in these videos, I interrogate the affordances of the TikTok form and question easy assumptions that such videos might effectively disrupt colonial and imperial linguistic hierarchies. Moreover, I suggest that usages of Caribbean vernacular on social media accumulate connotations of funniness. The term \quotation{funniness} aims to capture their likelihood of registering as innocuously entertaining and/or being cast as outside of serious inquiry. I unpack the implications of these connotations by turning to the testimony of a survivor of the 2021 eruption of La Soufrière in St.~Vincent and the reactions on Facebook to the survivor's use of St.~Vincentian English. Pairing that testimony with Andrew Salkey's 1964 novel {\em Hurricane}, I argue that funniness risks compromising the opportunities for ideological inspection that fictional and non-fictional representations of disaster make possible. The inclusion of Salkey reiterates my underlying claim in this paper that Caribbean literature offers analytical provocations that usefully serve the study of social media usage in the Caribbean.

\startsectionlevel[title={Caribbean TikTok},reference={caribbean-tiktok}]

\placefigure[here]{Trini Alphabet Series @stephon23}{\externalfigure[issue07/felmine.jpg]}


\placefigure[here]{@wellonyc}{\externalfigure[issue07/simpson.jpg]}


Felmine copied the structure of his skits from Simpson, who also goes by Ragga Rebel.\footnote{A dispute between Simpson and Felmine regarding intellectual property emerged. See Janine Mendes-Franco, "Caribbean Alphabet Series Provides Lots of Laughs on TikTok," {\em Global Voices} (15 October 2020): no pages, https://web.archive.org/web/20210526221358/https://globalvoices.org/2020/10/15/caribbean-alphabet-series-provides-lots-of-laughs-on-tiktok/} Hailing from Guyana and now based in New York, Simpson also teaches his ABCs on TikTok but with the words and enunciations circulating in Guyana, as in \quotation{A is for Arinj.}\footnote{Kwame Simpson, "Arinj," TikTok Video (22 May 2020), https://web.archive.org/web/20210517150915/https://www.tiktok.com/@wellonyc/video/6829872495324155142} He uses background images, costumes, and props more readily than Felmine, drawing on his theater training.\footnote{Shamar Meusa, "'Wello' Makes Waves with Guyanese Alphabet Series," {\em Stabroek News} (26 July 2020), https://www.stabroeknews.com/2020/07/26/sunday/wello-makes-waves-with-guyanese-alphabet-series/>} While the delight derived from both Felmine's and Simpson's performances includes recognition (we really does talk like that) and pride (only we could talk like that), the humor in both primarily hinges on appreciation of how the classroom setting and the performer's seriousness are inconsistent with what is usually perceived as informal or indecorous language. The defining similarity, then, is that while both sets of performances are overtly expositive in tone, neither prioritizes explaining these words to cultural outsiders. Instead, they work with irony, depending as they do on an audience's familiarity with, and alertness to, context.

This dynamic between performer and audience on a social media platform can be better understood through reference to Trevor Boffone's {\em Renegades}. Analyzing how Generation Z uses dance and hip hop to fashion their identities in the United States, Boffone argues that \quotation{{[}d{]}igital spaces such as Dubsmash and TikTok allow Zoomers---and especially teens of color---to produce cultural content that, in turn, facilitates young people's participation in an increasingly inclusive and democratized media culture.}\footnote{Trevor Boffone, {\em Renegades:} {\em Digital Dance Cultures from Dubsmash to TikTok} (Oxford: Oxford UP, 2021), 12.} Given the capacity of these apps to launch users into widespread visibility and notwithstanding the prevalence of anti-Black racism and cultural appropriation on these apps, Boffone claims that these spaces allow young people to \quotation{construct their digital personas and performances in such a way as to signal a powerful future in which community is the focus and previously marginalized identities are the mainstream.}\footnote{Boffone, {\em Renegades}, 17.} He firmly believes that sharing, learning, celebrating, and collaborating on dance performances \quotation{can be used as a strategy to bring the audience in and help them identify with a collective culture.}\footnote{Boffone, {\em Renegades}, 81.} Boffone considers which digital spaces maximize this potential for community building in the United States, suggesting what might be at work via TikTok performances in the Caribbean. Moreover, Boffone argues for the capacity for such spaces to disrupt and transform stubborn systems of bias: \quotation{By foregrounding Dubsmash as a fundamental part of my pedagogy, obligation to each other instead became a way for my students and me to trouble the traditional hierarchies and relationships seen in urban public education.}\footnote{Boffone, {\em Renegades}, 89-90.}

Given that Felmine's and Simpson's performances center jokes rather than dances, I also turn to \quotation{SRSLY?? A Typology of Online Ironic Markers} by Noam Gal, Zohar Kampf, and Limor Shifman, a study of ironic humor on social media.\footnote{Noam Gal, Zohar Kampf and Limor Shifman, "SRSLY?? A Typology of Online Ironic Markers{\em ," Information, Communication & Society} (September 2020), \useURL[url1][https://doi.org/10.1080/1369118X.2020.1814380]\from[url1]} It focuses on the Israeli socio-political context and Facebook---though posts on Twitter, Tumblr, Reddit, and Instagram are also considered---to argue that although irony in both digital and non-digital spaces depends on \quotation{an intended amusing gap between the meant and the said,} nevertheless decoding irony in \quotation{the networked, fragmented, and hyperlinked arena of social media} often functions differently.\footnote{Gal, Kampf, and Limor, 1-4.} Their analysis \quotation{yielded a classification of five domains of ironic marking: platform, participants, linguistic style, intra-textual elements, and contextual knowledge.}\footnote{Gal, Kampf, and Limor, 7.} On the one hand, absence of contextual knowledge or lack of familiarity with an individual's personal style of expression can compromise another user's ability to detect irony. On the other hand, the social media profile, the use of hashtags and emojis, comments from other viewers, or the specific group or page to which material is posted are all factors that could offer explicit clues for easier detection of irony.

The work of Boffone and Gal et al.~clarifies how Simpson and Felmine wield irony on TikTok. In their performances, their audiences are meant to detect and enjoy the difference between what is said and what is meant: \quotation{this pedagogical exercise is in earnest} vs.~\quotation{we both know that these words are typically cast as the antithesis of suitable classroom material.} Thus, one is more likely to appreciate the performances' ironic humor when one is already familiar with the language, power dynamics, and teacherly posture found in many a Caribbean classroom. Felmine furthers the irony by admitting that his actual profession is teaching by regularly adding \#boredteacher to his videos, acknowledging that he is exaggerating mannerisms he may adopt in a real classroom. These skits adopt the pretence of furthering a viewer's literacy, but they are actually dependent on that viewer's pre-existing vernacular and digital literacies to get the joke. Given the finding by Gal et al.~that the \quotation{enhanced potential for interpretive failure in the digital network sphere may be used for the creation of in-group bonds,} these videos function as opportunities to consolidate Caribbean identities via their ironic humor rather than through shared learning.\footnote{Gal, Kampf, and Limor, 2.} Notably, both series also embed more obvious ironic markers. For example, both series almost immediately draw on scatological humor: \quotation{Ah two two mehself,} in Felmine's \quotation{A is for Ah,} and Simpson's \quotation{C is for Coongcy.}\footnote{Stephon Felmine, "Trini Alphabet Series: A is for Ah," TikTok Video (September 24, 2020 ), https://web.archive.org/web/20210517152122/https://www.tiktok.com/@stephon23/video/6876192861981805825

  Kwame Simpson, "the letter of the day!" TikTok Video (June 7, 2020), https://web.archive.org/web/20210517151749/https://www.tiktok.com/@wellonyc/video/6835776554933439750} Inclusion of examples like these as well as more derisive comments in other videos more readily identify this material as incongruous with a classroom setting, increasing the likelihood that cultural outsiders would also get the joke.

It would be a misstep to retrospectively position these videos as aiming to maintain Caribbean artistic traditions of challenging colonial and imperial legacies. The creators' participation in the calculus of social media defines the situation. Without denying that the videos offer opportunities to gain knowledge and erode what is authoritative, the primary goal here is jokes that translate into likes and follows. Put simply, Felmine's and Simpson's videos, while valuable in their own ways, are not part of a larger enterprise of revolutionizing pedagogy. Their efforts are distinct from projects like the creation of Grenada's {\em Marryshow Reader}, under the auspices of the New Jewel Movement. Shalini Puri explains this project in {\em The Grenada Revolution in the Caribbean Present}: \quotation{Merle Hodge, the Trinidadian writer who came to Grenada to work on education, was at the forefront of the collaborative process through which the {\em Marryshow Reader} was created\ldots{}.{[}I{]}t contributed to an educational system in which Grenadians could recognize themselves and their experience as legitimate and normal rather than as aberrations from a British bourgeois cultural norm.}\footnote{Shalini Puri, {\em The Grenada Revolution in the Caribbean Present: Operation Urgent Memory} (London: Palgrave MacMillan, 2014), 52.} That undertaking represented an attempt at social transformation that was systematic and lasting. The TikTok videos are more accurately described as fleeting by design: they continue to exist after their consumption, but that existence is secondary to the promise of future consumption. Hence Felmine's subsequent launch of \quotation{Trini Number Series,} \quotation{Trinounciations,} \quotation{Tridioms,} and \quotation{Stephen's School of Thought," which all seek to continue the success of the}Trini Alphabet Series” and thereby entrench his profile as one worth following. TikTok depends on this implicit promise of incessant generation and consumption of new cultural products whose aim is to immediately supersede the relevance of what was previously produced. That promise is what allows one to accumulate followers.

This distinction emphasizes the fact that cultural production on social media is bound by the limits of these platforms. Indeed, as Kelly Baker Josephs explains in \quotation{Me, Myself, and Unno,} a study of self-fashioning via blogs, \quotation{{[}d{]}iscourse on digital publishing often promotes the practice as freeing and enabling for creative enterprise, but, as with the technologies of language and print, digital communication also structures what can be said.}\footnote{Kelly Baker Josephs, "Me, Myself, and Unno: Writing the Queer Caribbean Self into Digital Community," {\em archipelagos} 5 (2020): 6.} At first blush, TikTok seems to afford people a free creative outlet, as they can produce and circulate short video clips that can be conveniently recorded and seamlessly captioned, altered, and stitched in a variety of ways. Brevity and ease of creation and collaboration contribute to the high production and consumption rates that characterize the platform. However, scholars like Josephs, Boffone, and Gal et al.~suggest that it is necessary to analyze more rigorously TikTok's enabling capacities and limitations. Do the publishing platform's rules about video length incline users like Felmine and Simpson to depend on sharp expressiveness rather than extended etymological histories? How can we reconcile the fact that TikTok's format fundamentally dictates the type of content that proliferates on its platform with the ways in which creators innovate within those parameters?

In an introduction to a special section on the app in the {\em International Journal of Communication}, Jing Zeng, Crystal Abidin, and Mike S. Schäfer detail the app's history and the foci of scholarship on the platform.\footnote{Jing Zeng, Crystal Abidin, and Mike S. Schäfer, "Research Perspectives on TikTok and Its Legacy Apps: Introduction{\em ," International Journal of Communication} 15 (2021): 3161-3172.} The authors explain that 2018 saw "the Chinese technological company ByteDance's launch of TikTok for the global market, following the swift success of its sister app, Douyin, in China" and with "the COVID-19 outbreak in 2020, the global lockdowns further catalyzed the massive expansion and diversification of TikTok's user groups."\footnote{Zeng, Abidin, and Schäfer, \quotation{Research,} 3161-3163.} They note the burgeoning scholarship that \quotation{direct{[}s{]} scholarly attention to the systemic inequalities entrenched in the platform logics and governance structures, the platform's initial dependence on young people and their vernacular innovations that constitute the early norms on TikTok, and other emerging short-video platforms.}\footnote{Zeng, Abidin, and Schäfer, \quotation{Research,} 3166.}

In another article, Abidin argues for careful study of internet celebrity on the app, noting a number of useful distinctions.\footnote{Crystal Abidin, "Mapping Internet Celebrity on TikTok: Exploring Attention Economies and Visibility Labours," {\em Cultural Science Journal} 12, no. 1 (2020): 77--103.} She explains that, as opposed to \quotation{the era of persona-based or profile-anchored fame} that was maintained during the dominance of YouTube and Instagram, on an app like TikTok, \quotation{the nature of fame and virality has shifted, and tends to be based on the performance of users' individual posts.}\footnote{Abidin, \quotation{Mapping,} 79.} Users are less likely to maintain a uniform persona and are more likely to adapt to each latest trend or social media challenge. Viral fame gained from the success of a post or a series of posts can easily incline a user to leverage it into a profitable career. As Abidin clarifies, this \quotation{era of post-based fame} inclines TikTok users to opt into \quotation{what is \quote{going viral} at the moment in order to remain visible to others on the app} and to \quotation{attempt to introduce new trends that may fly or flop, wax and wane, depending on the flavour of the moment}\footnote{Abidin, \quotation{Mapping,} 79-80.} In pursuit of such fame, users often aim to offer posts that simultaneously satisfy target audiences and the app's algorithms, which favor and popularize select content.

Therefore, studying content on TikTok involves investigating how the app facilitates collaboration and imitation alongside the ways in which the app's algorithms determine whose videos are promoted to whom. Zeng et al.~refer to TikTok's For You Page, explaining that \quotation{videos are promoted to users via TikTok's algorithmic recommenders, which are savvy at personalizing interests and engagement\ldots{}This recommendation system is the core intellectual asset of ByteDance.}\footnote{Zeng, Abidin, and Schäfer, \quotation{Research,} 3163.} Andreas Schellewald clarifies that TikTok's algorithms \quotation{observe how users engage with the videos they see, if they tag them with \quote{like} or \quote{not interested,} if they rewatch them, leave a comment, or read those of others, follow the creator of the video, or look at their profile pages.}\footnote{Andreas Schellewald, "Communicative Forms on TikTok: Perspectives from Digital Ethnography," {\em International Journal of Communication} 15 2021: 1438.} Such surveillance then determines the app's recommendations. Schellewald notices the videos' ephemeral nature and offers detailed ethnographic studies of them, which allows him to declare six overarching communicative forms: comedic, documentary, communal, interactive, explanatory, and meta.

The affordances of TikTok's communicative forms meet what is already at work in the digital culture of the Caribbean. These dynamics are traced by Jonathan J. Felix, who considers a meme constellation created by Warren Le Platte in 2016. Felix argues that \quotation{{[}c{]}onceptually, the alternative media framework acknowledges the political act of self-publishing as a critical practice that resists commodification, democratizes public involvement, and allows for the creation of independent sites or \quote{free spaces} for counter-discourse and experimentation.}\footnote{Jonathan J. Felix, "Culture Jamming in the Caribbean: A Case of Alternative Media through Double Alternativity in Trinidad and Tobago," {\em archipelagos} 5 (2020): 21.} The existence of such spaces within Trinidad and Tobago's digital landscape is plausible, and Felmine even suggests such a reading of his videos, since he retrospectively frames his creations in terms of the validity of the Creole continuum. In a newspaper interview, he states, \quotation{You can learn to speak \quote{properly} but you need to know where you come from. It makes you who you are and you shouldn't be ashamed of it. It is not \quote{broken English} but part of the colloquial tongue.}\footnote{Julien Neaves, "From ABC to 123, Trini Alphabet Guy Goes Viral," {\em Trinidad and Tobago Newsday} (10 Nov 2020), \useURL[url2][https://web.archive.org/web/20201117185825/https://newsday.co.tt/2020/11/10/from-abc-to-123-trini-alphabet-guy-goes-viral/]\from[url2].} Felmine is obviously right that the language he revels in is constitutive of individual and collective identities and worldviews. His words resonate with Miller's statement \quotation{I know the world by how I speak the world.} Miller and Felmine share, in some sense, an appreciation of how languages relate to reality.

However, Felix's assessment appears rather idealized, as social media is also a complex commercial industry. As Boffone puts it, \quotation{going viral is no longer about becoming popular for a moment. Going viral is about clout and making money.}\footnote{Boffone, {\em Renegades}, 57.} This applies to both content creators and those who can piggyback on that visibility. Companies in the Caribbean and elsewhere often roll the latest social media sensations into their marketing strategies to flog an array of commodities. Boffone explains, \quotation{{[}y{]}oung people with significant follower counts are lucrative marketing spaces -- effectively, hyper-specific social media billboards ­that large companies have become increasingly interested in.}\footnote{Boffone, {\em Renegades}, 57.} Unsurprisingly, Felmine was recruited by a local company, Tootoolbay.com, to reproduce his skit format, including the premise of instruction and the use of vernacular, for an advertisement for their services. The \quotation{T is for Tootoolbay} video reiterates how the cultural production found on social media can, and often does, serve broader capitalist needs.\footnote{Tootoolbay.com, "Letter of the Day: T is for Tootoolbay.com," (Oct 2020), \useURL[url3][https://www.facebook.com/tootoolbay/videos/649301035954820/]\from[url3]} Admittedly, drawing on local comedic talent in this way is not entirely new. Previous decades saw Dennis \quote{Sprangalang} Hall touting Breeze Soap Powder or Shirley \quote{Beulah} King and Errol \quote{Stork} St.~Hill plugging Kaleidoscope paint. Yet the old and new ways in which social media and commerce are entangled beg further scrutiny.

In Felmine's case, the interplay between his social media endeavors and other commercial endeavors suggests that the connotations of his use of Caribbean vernacular are compatible with the desires of companies. Any resistance, experimentation, or counter-discourse originally on offer seems easily tamed, raising the question of whether or not Caribbean ways of speaking register as a quirk that is easily and profitably slotted into pre-established digital and commercial forms. These claims aim to challenge Felix's argument that \quotation{Trinbagonian digital culture\ldots{}is also a means of maintaining the contours of local identity, considering increasing globalization and especially the pervasive process of Americanization} (22). This understanding of Trinbagonian digital culture, specifically, and of Caribbean digital culture, more broadly, does not match Felmine's and Simpson's digital undertakings. While social media users like them may draw on and continue to transform local histories of humor in some ways, their cultural products are definitively structured by forces of globalization, specifically the communicative forms permitted by corporations like Facebook and ByteDance, which intentionally gel with capitalist needs.

Moreover, as various scholars in TikTok studies make clear, understanding the significance and risks of what is created within the app means evaluating how the app extends what is already in existence. Boffone argues that, \quotation{while TikTok is a space that can become a source of empowerment and agency for some, it is also an app that is plagued with many of the material issues of racialized identity that are hallmarks of social life in the United States} (33). Likewise, Melanie Kennedy, in her study of \quotation{the viral spectacle of girls' bedroom culture,} positions certain TikTok phenomena as the \quotation{continuation and intensification of girl culture and the ideals of young female celebrity of the past two decades} (1070). Yet she also acknowledges how the app allows for the \quotation{obscuring of the dangers and impacts faced by girls around the world who are situated outside of the ideals embodied in TikTok stars.}\footnote{Melanie Kennedy, "'If the rise of the TikTok dance and e-girl aesthetic has taught us anything, it's that teenage girls rule the internet right now': TikTok Celebrity, Girls and the Coronavirus Crisis," {\em European Journal of Cultural Studies} 23, no. 6 (2020): 1070-1071.} For this reason, analysis of what users generate and circulate on an app like TikTok requires attention to all that is new, but must refer to what preceded it.

In terms of analyzing the reception of Caribbean ways of speaking on social media, I suggest connections to the circumscription of Caribbean ways of speaking within popular music. The Jamaican dancehall artist Mr. Vegas or Clifford Smith discussed this issue in his {\em MV Corner Vlog} on Facebook in 2016. He made a series of posts to address how Jamaican music, artistry, and ways of speaking were co-opted by the commercially successful Canadian rapper Aubrey Drake Graham in his fourth studio album {\em Views}.\footnote{I have previously interrogated Drake's positioning in the hip hop genre. See Kris Singh and Dale Tracy, "Assuming Niceness: Private and Public Relationships in Drake's {\em Nothing Was the Same}," {\em Popular Music} 34, no. 1 (2015): 94-112.} Mr.~Vegas' main contention was that Drake samples but does not explicitly credit or include Jamaican artists in his album, thereby continuing Western exploitation of Caribbean musicians.\footnote{See Shereita Grizzle, "Drake is Fake! -- Mr.~Vegas," {\em The Jamaica Star} (May 21, 2016), \useURL[url4][https://web.archive.org/web/20210521130352/http://jamaica-star.com/article/entertainment/20160521/drake-fake-mr-vegas]\from[url4]} In making this point, he clarifies that while U.S. radio stations rebuff Jamaican artists, they embrace U.S. and Canadian artists who mimic Jamaican music: \quotation{When we take it to your radio station, you're saying you cannot play our reggae, but you're playing Justin Bieber reggae. {[}\ldots{}{]} People are saying \quote{oh you guys are not making quality music. We can't understand what you are saying.} That is straight English right there, unless you don't know how to speak English.}\footnote{Clifford Smith, "MV Corner: Mr.~Vegas Responds\ldots{}" Facebook Video (23 May 2016),

  \useURL[url5][https://www.facebook.com/MrVegasMusic/videos/10154195593144486/]\from[url5].} His comment about \quotation{straight English} expresses in a nutshell how Caribbean ways of speaking may register: they are valid when innocuously garnishing dominant forms but unintelligible when disruptive to the status quo or not profitable. His words echo a much quoted statement by Galahad in Selvon's {\em The Lonely Londoners}. Galahad retorts to his white English date who complains about not being able to understand him, \quotation{\quote{Is English we speaking.}}\footnote{Samuel Selvon, {\em The Lonely Londoners} (London: Penguin, 2006), 82.} Selvon tackles how Caribbean ways of speaking are heard in interpersonal relationships. Mr.~Vegas addresses how they are heard in popular music and within the music industry. I find much of their critiques to be equally valid when it comes to social media.

There is obviously much more to Caribbean TikTok than what I chose to focus on in analyzing Felmine's and Simpson's videos. For example, Abidin cites one of the most noted features of the app, the easy integration of pre-existing audio clips into original videos, explaining that it \quotation{can be understood as an \quote{audio meme,} and is the driving template for content production on TikTok} (80). While both Felmine and Simpson use original audio, many other Caribbean TikTokers make use of these audio memes in publishing diverse performances, dances, parodies, and explicit social commentary. Caribbean TikTok and Caribbean social media at large offer much novelty and innovation for scholars to engage with. However, as I show in the following section by attending to how natural disaster appears in Caribbean social media and in Caribbean literature, there are also connections to pre-existing social dynamics and artistic traditions that can be traced.

\stopsectionlevel

\startsectionlevel[title={Disaster Language: La Soufrière and Andrew Salkey's {\em Hurricane}},reference={disaster-language-la-soufrière-and-andrew-salkeys-hurricane}]

Having considered TikTok examples that aim to, and succeed in, conjuring laughter, I turn to a more serious example that nonetheless finds laughter as a response. Upon the devastating eruption of La Soufrière in April 2021, news reports detailing the event were in high demand, and iWitness News offered multiple interviews with the people of St. Vincent, who were simultaneously contending with volcanic ash and the COVID-19 pandemic. Kenton Chance's interview with a farmer identified only as \quotation{Roni aka \quote{Smalls}/\quote{Small Man}} from Chateaubelair became especially popular once posted on iWitness News' Facebook and YouTube pages.\footnote{"Two Nights on an Erupting Volcano," {\em iWitness News}, You Tube (14 Apr 2021), \useURL[url6][https://www.youtube.com/watch?v=oFfu1RDmZNg]\from[url6]} On Facebook, the video has surpassed 531K views, with over 2700 comments, far exceeding the popularity of other posts from this news organization.\footnote{"Two Night on an Erupting Volcano," {\em iWitness News}, Facebook (14 April 2021), \useURL[url7][https://www.facebook.com/iwnsvg/videos/381706549527009]\from[url7]} The appended comments reveal a few reasons for the surge of interest in this interview. Some social media users appreciate the interviewee's knack for storytelling and his intimate perspective on the ramifications of this natural disaster on St. Vincent's agricultural sector. Others celebrate or bemoan his foolhardiness for staying on the mountains for two nights during the eruption before eventually fleeing. And a few individuals berate other commenters for appearing to make light of the man's precarious position, as many individuals simply offer emojis or gifs depicting laughter while tagging friends, digitally replicating the conspicuous contagion that laughter often is.

What is readily apparent in the interview is the contrast in language between interviewer and interviewee. While Chance employs Standard English with the recognizable tone and enunciation of many a Caribbean broadcaster, Roni works with the vernacular of rural St.~Vincent. His stark descriptions such as \quotation{Ashes mash up the whole place,} together with his citation of the loss of all the coconut trees in the area and the dead birds littering the ground begin to depict the absolute devastation that accompanied this eruption. His comment that the sound of the volcano's eruption was so loud that \quotation{town people might have hear that and all} directs viewers' attention to how the pre-existing divide between rural and urban St.~Vincent is momentarily transgressed by disaster. His descriptions evidence close familiarity with the local environment, referring to place names and landscape details not immediately known to the interviewer. His testimony reiterates what Miller notes in his writing: registering the effects of natural disaster means grappling with sites, people, and activities that official records often omit. As the interview wraps up, Chance secures access to Roni's digital documentation of the volcano's devastation on his tablet computer. The final minutes of the video show a crowd circling Roni as everyone peruses the photographs and videos of the volcano's devastation that Roni has prepared them for and continues to narrate.

To situate such testimony of disaster survival in longer histories, I turn to Lizabeth Paravisini-Gebert's \quotation{Mourning the Dead of St.~Pierre in the Yellow Press."\footnote{Lizabeth Paravisini-Gebert, "Mourning the Dead of St.~Pierre in the Yellow Press," {\em Small Axe Salon} 12 (May 2013), https://web.archive.org/web/20210127160448/http://smallaxe.net/sxsalon/discussions/mourning-dead-st-pierre-yellow-press} She considers the 1902 eruption of Mont Pelée in Martinique, which she identifies as both}the twentieth century's deadliest volcanic eruption” and \quotation{the first natural disaster of the protomodern mass media.}\footnote{Paravisini-Gebert, "Mourning the Dead of St.~Pierre in the Yellow Press," paragraphs 1 and 16.} Given the technological advancements of that era, representations of Mont Pelée's eruption in newspaper stories and photographs circulated widely. Paravisini-Gebert analyzes the difference in treatment of dead white people and dead Black Martinicans in these representations, especially in the yellow press. In reading what was and was not photographed in the volcano's aftermath, Paravisini-Gebert compares \quotation{the objectified dead} and \quotation{the mourned dead,} since the corpses of Black Martinicans were photographed in a way that \quotation{ascribes agency to the viewers and relegates the dead to a curiosity,} producing \quotation{an ambivalent space marked by the absence of concern or a sense of loss.}\footnote{Paravisini-Gebert, "Mourning," paragraph 16.}

Unlike natural disasters of previous eras, local inhabitants' narration of their own experience of natural disasters in the present-day Caribbean is more likely and more accessible, suggesting disruption of that history of absences identified by Paravisini-Gebert. Testimonies like Roni's begin to answer the difficult question of what language relays the ineffable destruction that may follow a natural disaster. Hearing his tale allows one to reconsider, for instance, human fragility in relation to the natural world, but there is no guarantee against mishearing testimonies like Roni's, especially in, as Gal et. al.~put it, \quotation{the networked, fragmented, and hyperlinked arena of social media.} Given the accumulated connotations, what inclines or disinclines Caribbean and non-Caribbean audiences from registering vernacular expressions of grief, fear, or trauma that circulate on social media as anything but a curiosity? In terms of fictional representations of grief, fear, and trauma that is offered in Caribbean vernacular, there is a history of readers receiving the humor without necessarily appreciating the hardship. For example, I have previously considered this issue in relation to Selvon's lonely Londoners, arguing that readers may read his description of the plight of Caribbean immigrants in London as funny episodes told compellingly without fully acknowledging how the dignity of his characters is repeatedly compromised.\footnote{Kris Singh, "Is Joke You Joking?," {\em Pree: Caribbean. Writing} 3 (Spring 2019), \useURL[url8][https://web.archive.org/web/20201201081351/https://preelit.com/2019/04/20/is-joke-you-joking/]\from[url8].} To hear one without the other is to further their loneliness.

Without discounting the fact that one can use humor to cope with hardship or the fact that one may draw on humor when narrating one's own experience, I am suggesting that many social media users (especially, but not exclusively, those outside the Caribbean and therefore less likely to encounter vernacular regularly) are primed to hear vernacular testimonies as funny, because of the connotations that the vernacular has accumulated in the digital environment. Since academic work, news reports, government documents and the like favor Standard English in the Caribbean digital landscape, Standard English monopolizes the serious, while the vernacular is pigeonholed as the funny. That label of funniness not only means that the concerns of speakers of the vernacular are made obtusely humorous but speakers are also cast as undeserving of serious consideration, without political consequence, or on the margins of the public sphere. As with irony, contextual cues surrounding a post could disrupt such impositions of funniness, but they could just as easily reinforce them.

What is at stake in hearing vernacular descriptions of disaster in this manner is made clear by Andrew Salkey in his novel {\em Hurricane}. I refer to this novel as it is one that lays bare the intersections of disaster, language, and media. My reading of the novel supports my argument that, if users of social media replicate extant social dynamics, then analysis of social media use in the Caribbean benefits from the analytical work done by writers like Salkey. His novel provides heuristics for understanding how individuals scramble for language to describe experiences that defy easy description, how we may turn to the language popularized by dominant cultural forms, and how such moments of disaster threaten individual and collective sensibilities.

While Salkey's works remain understudied, they endure as favourites. For instance, in a 2016 interview with {\em The} {\em New York Times}, Zadie Smith is prompted, \quotation{Tell us about your favorite overlooked or underheralded writer,} to which she replies, \quotation{A Jamaican writer called Andrew Salkey, who wrote a Y.A. novel called {\em Hurricane} before Y.A. was a term. I remember it as the book that made me want to write.}\footnote{Zadie Smith, "Zadie Smith: By the Book," {\em The New York Times} (November 17, 2016), \useURL[url9][https://web.archive.org/web/20201112011909/https://www.nytimes.com/2016/11/20/books/review/zadie-smith-by-the-book.html]\from[url9]} Published in 1964, {\em Hurricane} is the first in Salkey's disaster quartet: it was followed by {\em Earthquake} (1965), {\em Drought} (1966), and {\em Riot} (1967). Given its intended audience, the novel takes the form of a relatively straightforward first-person narrative told by thirteen year-old Joe Brown as he anticipates and survives the onslaught of Hurricane Chod, bunkered in his Kingston home with his mother, father, and sister. Readers are not fellow survivors of the depicted hurricane. Instead, as witnesses of Joe's survival, and that of his family and friends, readers see the limitations of characters' perspectives. As a representation of disaster, {\em Hurricane} essentially creates space to inspect the intersections of competing discourses and ideologies as they relate to comprehending disaster.

This focus on the hurricane itself departs from other readings of the novel that view the hurricane and characters' survival as commentary on the national consciousness of newly independent Jamaica. In \quotation{The Political Ecology of Storms in the Caribbean,} Sharae Deckard briefly engages with {\em Hurricane} within her excellent consideration of storms and storm aesthetics in texts from the Anglophone, Hispanophone, and Francophone Caribbean. She argues that the hurricane \quotation{engenders a collective geographical consciousness symbolized by Joe's constant imagination of the plight of the rest of the island.}\footnote{Sharae Deckard, "The Political Ecology of Storms in the Caribbean," in {\em The Caribbean: Aesthetics, World-Ecology, Politics}, eds.~Chris Campbell and Michael Niblett (Liverpool: Liverpool UP, 2016), 34.} She reads {\em Hurricane} as a novel that \quotation{narrativizes Jamaica's resilience,} thereby fitting the novel into a category of Anglophone Caribbean children's literature in which hurricanes are \quotation{straightforwardly linked to a celebratory politics of decolonization.}\footnote{Deckard, "The Political,” 33-34.}

Deckard is right in identifying the novel's emphasis on collective bonds, yet {\em Hurricane} also stresses the fact that different characters draw on different lexicons to contend with this moment. {\em Hurricane} foregrounds Joe's maturation, but it also aims to understand what is involved in articulating disaster in the moment of its occurrence. For example, Joe's father refers to his previous experience with natural disasters in Jamaica, recalling the drought of 1932-1933. He understands the silence during the hurricane's eye by identifying it as the \quotation{\quote{same sort of silence when there's a drought on.}}\footnote{Andrew Salkey, {\em Hurricane} (Leeds: Peepal Tree Press, 2011), 72.} That capacity to partially make sense of the moment provides him with some semblance of security when little is assured for him or his family. Joe's mother, in turn, refers to the Bible. She makes numerous, humorous references to Noah's Ark before the hurricane's arrival and then rereads and recites Psalm 91 for comfort while enduring Chod's worst. Her relationship with Christianity is distinct from the doomsday anticipation of the street preacher, Mother Samuel, who warns at the start of the novel that the \quotation{\quote{entire population of this wicked Island of ours is going to perish by hurricane at midnight tonight.}}\footnote{Salkey, {\em Hurricane}, 20.} While Mother Samuel stands out as an alarmist with her apocalyptic vision, Joe's mother produces reassurance via her religious sensibilities. Her attitude, Joe explains, allows \quotation{{[}y{]}ou {[}to{]} feel that she has taken everything off your shoulders} because \quotation{{[}i{]}t's as though the hurricane is attacking {\em her} personally.}\footnote{Salkey, {\em Hurricane}, 60.} She positions herself as a protective, self-sacrificing force in the lives of her children by taking instruction from the benevolent promise of protection found in Psalm 91.

Likewise, the novel considers what children like Joe and his friends reach for to understand what seems incomprehensible. While Milton Perkins or \quote{B.B} (as in Brainy Boy) draws on statistics and scientific facts, movies constitute Joe's frame of reference. The movie theater itself, the Carib, is an outsized landmark in his worldview; hence his likening it to Jamaica's Blue Mountain.\footnote{Salkey, {\em Hurricane}, 34.} The fact that the Carib emerges unscathed at the novel's end is an overt suggestion that all is not lost despite the hurricane's devastation. Joe's dependence on movies, however, goes beyond his love for the Carib. The movies ground his vocabulary. As he listens to Hurricane Chod gain strength, Joe compares the thunder to \quotation{a bomb exploding in a war-film.}\footnote{Salkey, {\em Hurricane}, 56.} Surveying the aftermath of the hurricane and discovering a lamppost coated in mud and debris, Joe is \quotation{reminded\ldots{}of a creature from outer space in a science-fiction film.}\footnote{Salkey, {\em Hurricane}, 90.} Joe depends on U.S. films seen in the Carib theater to make sense of the immediate. They loom so large in his imagination that their clichés and conventions are what he draws on to comprehend what is, to him, unprecedented. They not only allow him a filmic vocabulary but also enable his conception of his relationship to this context. As the hurricane becomes stronger, Joe explains he was not afraid but thrilled: \quotation{I felt the way someone feels when he's sitting down in Carib and watching a film that's packed with suspense and big drama.}\footnote{Salkey, {\em Hurricane}, 56.} For better or worse, Joe understands himself not as potential casualty but as a safe and entertained audience.

Joe's turn to movies to make sense of Chod accords with his descriptions of his friends, family, and himself through such cultural products. As narrator, he defines his friends' personalities through reference to their typical behaviour in the cinema, whether it's \quote{Four Eyes} Dodd's habit of \quotation{supplying background music when there wasn't any,} \quote{Popeye} Mason's desire to \quotation{shadow-box every time there was a fight} on screen, or \quote{Shaved Head} Chin being so sensitive so as to be \quotation{known to cry during a cartoon film.}\footnote{Salkey, {\em Hurricane}, 18.} Likewise, when his sister, Mary, questions him about his whereabouts while everyone else was sleeping, he compares her to one of those \quotation{police officers in the gangster films where the poor suspect is put under a large, bright desk-lamp and cross-questioned for hours on end.}\footnote{Salkey, {\em Hurricane}, 43.} In some ways, readers may chalk up Joe's consistent reference to movies as indicative of children's indefatigable imagination. Karen Sands-O'Connor reads {\em Hurricane} within the category of British Children's Literature and argues that Salkey shows that \quotation{{[}i{]}magination is one of humanity's tools; it can enhance life, as in the case of the movie theater, and it can overcome disaster, but only when used in community.}\footnote{Karen Sands-O'Connor, {\em Soon Come Home to This Island: West Indians in British Caribbean Literature} (New York: Routledge, 2008), 144-5.} Similarly, Deckard emphasizes the novel's display of \quotation{the power of culture and imagination.}\footnote{Deckard, "The Political," 34.} She argues that for Joe the hurricane \quotation{is not traumatizing,} but \quotation{gives rise to Joe's own desire to write the story of its impact,} which she sees as emblematic of \quotation{the mythopoetic impulse embodied in the new generation.}\footnote{Deckard, "The Political," 34.}

That {\em Hurricane} foregrounds Joe's imaginative flare is undeniable, but the novel is more than an appreciation of the power of one child's imagination or the imaginative power of Jamaicans. The novel takes pains to consistently show that Joe is drawing on popular culture to understand what is in many ways incomprehensible. He has never seen nature so aggressive, his house so fragile, or adults so fraught with worry. Despite the seasonal nature of hurricanes in the Caribbean, thirteen-year-old Joe is experiencing something novel; and given the narrative structure of {\em Hurricane}, readers have intimate access to his evolving imagination. The novel makes space for readers to understand how disaster may strain and even rupture individual sensibilities, and readers are encouraged to consider what ideological shifts Joe manages via the models of his parents. For example, readers learn of Joe's initial investment in fantasies about handling the hurricane as one would a threat in a Western movie. Before the hurricane's arrival, Joe absconds to the Race Course to act out a cliché scene that incorporates the arrival of \quotation{Big Blow,} which curtails the showdown between the Sheriff and Rango.\footnote{Salkey, {\em Hurricane}, 38.} He abandons these fantasies and dread sets in as the hurricane becomes stronger and the devastation more real. As the house floods, he says, \quotation{I, for one, began feeling a creeping sense of panic rising to my throat} (87). Perceiving the \quotation{tired and strained} faces of his mother and sister, Joe elects to follow his father's lead: \quotation{he and I got down to work} (87). In this moment of heightened risk, Joe shifts from modelling himself after movies to emulating the words, views, and actions of his father.

The hurricane's aftermath bolsters this sensibility. As father and son engage in post-disaster repair, Joe becomes noticeably dismayed at the wreck of his father's workroom, and his father scolds, \quotation{\quote{Big men don't cry, Joe. They plan.}}\footnote{Salkey, {\em Hurricane}, 89.} Later, observing the damage done to the new house in Harbour View that the Browns had been anticipating moving into, Joe repeats this maxim to his father.\footnote{Salkey, {\em Hurricane}, 97.} Salkey may have been sincere in this misguided advice, given other instances of rigid gender norms in the novel. For instance, when questioned by his sister, Joe says, \quotation{I consoled myself when I remembered that all females, young and old, are pretty much on the same level where tormenting a man is concerned} (43). Regardless of the blinkered machismo put forth, the novel simultaneously demonstrates that disaster can be described as a moment of extreme ideological need wherein the ideas or discourses that one previously relied on are put to the test and either proven insufficient and discarded or prove useful and are concretized. This is not to say those dispositions are inherently natural, good, or healthy, but in not being disproven in a moment where survival is in question, they are in effect taken as validated and worth bequeathing. Representations of such disasters, whether factual testimonies or fictional narratives, therefore offer opportunities to resolve urgent ideological contradictions.

\stopsectionlevel

\startsectionlevel[title={Conclusion},reference={conclusion}]

Salkey is not alone in advocating for the capacity to carefully parse how Caribbean people know and speak disaster. In his invaluable exploration of nation language and \quotation{the tyranny of the pentameter} in {\em History of the Voice}, Kamau Brathwaite declares, \quotation{The hurricane does not roar in pentameters.}\footnote{Edward Kamau Brathwaite, {\em History of the Voice} (London: New Beacon Books, 1984), 32 and 10.} Brathwaite expounds on the limits of imposed rhythms and poetic forms when it comes to representing the Caribbean. He instead directs our attention to nation language or, as Brathwaite puts it, \quotation{an English which is not the standard, imported, educated English, but that of the submerged, surrealist experience and sensibility, which has always been there and which is now increasingly coming to the surface and influencing the perception of contemporary Caribbean people.}\footnote{Brathwaite, {\em History}, 13.} Likewise, Ishion Hutchinson, in his poem \quotation{After the Hurricane,} asks readers to note whose words are heard when it comes to assessing a hurricane's aftermath. The speaker of the poem contrasts what government surveyors officially document with Aunt May's individual experience of loss. They \quotation{scribble facts} and \quotation{draw tables} but \quotation{they can't say / how it tore down her senses.}\footnote{Ishion Hutchinson, "After the Hurricane," in {\em House of Lords and Commons} (New York: Farrar, Straus and Giroux, 2016), 15-16.} Hutchinson's poem, like Salkey's novel, argues that grappling with disaster means finding, in Brathwaite's words, "the syllabic intelligence, to describe the hurricane, which is our own experience."\footnote{Brathwaite, {\em History}, 8.}

These writers demonstrate the evolution of artistic traditions in the Caribbean that critically evaluate dominant ways of representing and knowing reality, offer alternatives, and promote the aptitude to engage with those alternatives. These traditions demand rather than resist rigorous attention to technological and cultural innovation and offer valuable models of cultural critique. Take, for instance, Brathwaite's discussion of the calypso form in {\em History of the Voice}. Brathwaite identifies Trinidadian calypsonian Mighty Sparrow or Slinger Francisco as a folk poet whose work exemplifies the power of nation language, and he offers this reading of Sparrow's "Ten to One is Murder": "There is in fact a 'tent' form known as calypso drama, which calls upon Trinidadian nation-forms like {\em grand charge, picong, robber talk}, and so on, which Sparrow is in fact consciously using in this calypso."\footnote{Brathwaite, {\em History}, 26.} One might assume that adapting such nation-forms into TikTok videos or other social media content inevitably broadens the reach of nation language. However, as this article asserts, usage of vernacular on social media risks decontextualization, distortion, and willful misinterpretation. Mitigation of these risks is not about social media users learning their vernacular ABCs but about overturning all that circumscribes vernacular to funniness. Until then, uses of vernacular on social media are more likely to register as the dialect "{[}c{]}aricature speaks in" rather than as what Brathwaite names nation language.\footnote{Brathwaite, {\em History}, 13.}

\thinrule

\stopsectionlevel

\page
\subsection{Kris Singh}

Kris Singh is a faculty member in the English Department at Kwantlen Polytechnic University. His scholarly and creative attention spans the legacy of indentureship, the relationships among writers of the Caribbean diaspora, and the ways in which social media and popular culture shape each other.

\stopchapter
\stoptext