\setvariables[article][shortauthor={Gelbard}, date={May 2023}, issue={7}, DOI={Upcoming}]

\setupinteraction[title={The Cabildo de Regla Project: A Digital Archive},author={Alexandra P. Gelbard}, date={May 2023}, subtitle={The Cabildo de Regla Project}, state=start]
\environment env_journal


\starttext


\startchapter[title={The Cabildo de Regla Project: A Digital Archive}
, marking={The Cabildo de Regla Project}
, bookmark={The Cabildo de Regla Project: A Digital Archive}]


\startlines
{\bf
Alexandra P. Gelbard
}
\stoplines


\startsectionlevel[title={{\em archipelagos} presents The Cabildo de Regla Project: A Digital Archive},reference={archipelagos-presents-the-cabildo-de-regla-project-a-digital-archive}]

Scholarly practice in the 21st century is marked by an existential demand to break down the walls of academia and work more closely with communities on the ground. Of the many shapes that this outreach can take, one of the closest to our traditional practices is the re-shaping of the cultural record to include historically marginalized voices in a close dialogue among equal partners. The Cabildo de Regla Project: A Digital Archive provides us with a model of how a scholar, completely new to digital humanities, has used their growing knowledge and access to resources to document and amplify the practices of a Caribbean afro-religious community with crucial historical and present-day significance for the region. The below exchange highlights the project's growth, always in dialogue, always in relation.

\stopsectionlevel

\startsectionlevel[title={Introduction to the project by Alexandra Gelbard},reference={introduction-to-the-project-by-alexandra-gelbard}]

The Cabildo de Regla Digital Archive documents, preserves, and disseminates the re-initiation process of the annual Lucumí (Santería) Orisha processional in the port town of Regla, La Habana, Cuba as part of a broader community collaboration. After the prohibition of this processional for 54-years, community members successfully revived its public tradition in 2016 as the Cabildo de Regla. It succeeded through a local collaboration of three State-recognized organizational entities: the Museum of Regla, the town's cultural and archival center; the Ile Olorun, a formally registered Lucumí practitioner house; and the Guïros de San Cristobal, a \quotation{cultural carrier} folkloric group organized around a set of sacred Lucumí ritual {\em bata} drums over one hundred years old.~

As one of the island-archipelago's African Diasporic epicenters, Regla became a central locale in the formation of at least two of Cuba's African-inspired religious practices during the 19th century: Lucumí and Abakuá.\footnote{The erasure of West Central African spiritual practices, generally referred to as Reglas Congo or Palo Monte, within the history of African-inspired religious practices, mutual-aid organizations called {\em cabildos de nación}, and the processional practice is a significant issue that I have not yet engaged within Regla. However, it should be noted that historically, West Central Africans, generalized under the Congo ethnonym, were likely in Regla from the sixteenth century onward, and there are active Palo practitioner communities whose members also include Cabildo de Regla participants.} The first half of the 20th century sedimented the annual Orisha processional, becoming an important data source for Cuban ethnographers bifurcating Lucumí life ways into the nationalistic concepts of Afro-Cuban folklore and Afro-Cuban religion. With the success of the Cuban revolution in 1959 and subsequent embrace of Russian Marxism's stance on religion, the government prohibited public expressions of religiosity after 1961. This policy unintentionally stymied the generational transmission of historical knowledge about Regla's African Diasporic religious life, as elders stopped sharing to align themselves within Cuba's revolutionary society. The push to re-initiate this processional never waned, but a shift towards emphasizing its social role as a localized practice fostering collective consciousness and social cohesion through a participatory and interactive multi-generational embodied practice, proved successful. Also central to this initiative is its public assertion of religiosity, providing a counterpoint to the folklorization of Lucumí religion, while encouraging elders to feel comfortable sharing their knowledge.~

In 2016, I was invited to meet with the community organizers about their project at the recommendation of Henry Heredia, an Olorisha and head of international relations at Juan Marinello, the research branch of the Cuban ministry of culture. My research on Cuban African Diasporic communities, religious practices, cultural production, and processionals aligned with the intent of the Cabildo de Regla project. I am also a Lucumí practitioner, as are most of the community organizers involved. At our initial meeting in the spring of 2016, we had no clear vision of a digital humanities project. Organizers expressed a desire for the creation of a visual archive made accessible to the local community as a preservation strategy for future generations. A small catalogue of photographs taken during the mid-20th century served as a key reference for the organizers; they wanted to build upon that collection with a mix of both ethnographic and artistic images. Since Cuban Lucumí is also practiced around the world, they wanted a way to virtually connect and disseminate this project to other practitioners and community members across the globe. Initially, we created a Facebook page while reserving the more artistic images for exhibitions. After the first two years, my growing awareness of Facebook's problematic dynamics stressed the need for an alternative virtual archival space. The onset of the 2020 COVID-19 pandemic stopped the processional's continuation while also inspiring the creation of the initial site, which launched at the Screening Scholarship Media Festival (SSMF) in 2021. With sparse experience in website creation and difficulty using Wordpress, I initially used Wix because of its easy design features. This was a huge mistake as I came to realize that Wix is blocked in Cuba, but one I am rectifying as I transfer the site to Wordpress, which is accessible on the island.~

The site intends to be a centralized hub for the overall project: the visual archive, a rotating sample of artistic exhibition photos, an overview of the various organizational and individual collaborators, and a space for public scholarship about the project. After consulting with several digital humanities scholars, archivists, documentary photographers, in addition to the spiritual messages received, I am preparing to re-design the photography galleries in response to their feedback. This will include reducing the number of artistic photos to reserve them for future exhibitions and cataloging the archival images in a systematic way. A mapping component will also be included in the future to better demonstrate how these ritual activities contribute to place-making and community identity.~

\stopsectionlevel

\startsectionlevel[title={{\em archipelagos} Anonymous Review},reference={archipelagos-anonymous-review}]

The Cabildo de Regla Project: A Digital Archive/Proyecto Cabildo de Regla: Un archivo digital captures the revival of the Orisha procession in Regla, La Habana, Cuba, after the government lifted the procession's banning in 2016. The processional is led by the Museum of Regla, the Ilé Olorun practitioner house, and the Guïros de San Cristobal performance group. Curator and researcher, Alexandra P. Gelbard, works in collaboration with the community to document these years of re-initation from 2016 onward.

The themes discussed on the site are relevant to understanding social dynamics and historical changes in Afro-Caribbean religions. However, the site could benefit from structural changes to efficiently convey the history of Regla, the Orisha procession, and its practitioners. The project's importance relies on capturing a unique moment that brings together local members to revive the community's processional tradition. The introduction could expand upon this relevance, the project goals, and the visual archive's uniqueness. The site's opening could also explain how the curator envisions users' engagement with the text and images that make up the whole project.

The site's current structure makes it challenging for users to navigate and explore its different features. The site's main sections, "Exhibition" and "Project Narratives," could work more efficiently by integrating the textual analysis into the image galleries. The Exhibition page contains photos of the Regla processions divided into three themes: Preparations, Processional, and People. Each gallery includes a brief description and stunning images of the activities, the material culture, and the people involved in the processions. The curator notes how these photographs serve three purposes: "as an archive for the community, as artistic interpretations, and as a visual data set for qualitative research." However, the galleries could greatly benefit from specific metadata and descriptions to fulfill these objectives. Except for the watermarks in the images (which sometimes indicate the photographers and the year), users can only learn a little about the photos if they have prior knowledge about the procession. If the project intends to be for public outreach and archiving, metadata and descriptions are crucial to contextualize the images. Another vital element the project could make more evident is historical change. The second gallery, Processional, explains how the procession changed over four years after re-initiation. Still, the organization of the exhibit and the need for more information in the images make these changes less apparent. The Project Essays, separate from the galleries, gather several essays with background information about the project, the procession, and the town of Regla. The site creators could restructure these essays to make them work in conjunction with the photo galleries to enrich the images' descriptions.

The project could also bring the discussion on photography, methods, and collaborators to a central place on the site. The essay, Photography and Knowledge Production, discusses in detail the different intentions behind using photographs to recreate and capture the procession in the present day. But this explanation needs to be more evident in the site's navigation. Likewise, the project's methodology and collaborative nature are only apparent once users visit the Collaborators section. It takes the user through several pages in this section to understand the different participants and their roles, which include photographers, researchers, informants, and local organizations. This collaborative nature is essential for the project and the Regla's religions and merits a prominent place on the site.

{\em Cabildo de Regla} has the potential to become an important platform for examining religious transformations among Afro-Cuban communities in the present. The procession and the project can teach users about the role of communal and mutual-aid organizations in Cuba, which, in the past and the present, have taken a central role in bringing religious traditions to life. The migration of the project to a new management system could offer the possibility to work with formats that enhance its storytelling and the interaction between images and texts. As the site intends to serve as a community archive, future steps could also include designing pedagogical tools for educators and the community at large. The Cabildo team could take inspiration from projects like \goto{Digital Aponte}[url(https://aponte.hosting.nyu.edu/)], \goto{Vistas: Visual Culture in Spanish America}[url(https://vistas.ace.fordham.edu/)], and the\goto{Early Caribbean Digital Archive}[url(https://ecda.northeastern.edu/)] to make future changes.

\stopsectionlevel

\startsectionlevel[title={Response to {\em archipelagos} review by Gelbard.},reference={response-to-archipelagos-review-by-gelbard.}]

On behalf of the Cabildo de Regla community organizers and myself, I wish to thank the editors of {\em archipelagos} and the reviewers for their thoughtful comments and recommendations for the \goto{Cabildo de Regla Digital Archive}[url(https://cabildoderegla.org/)].\footnote{The original title of the site at the time of the review was The Cabildo de Regla Project: Re-Initiating Tradition, reflecting the initial site as a hybrid archive and artistic exhibition, combining the two facets of the original intention of the documentary component of the Cabildo de Regla collaborative project. With the uncertainty of the COVID-19 pandemic, we were unsure if in-person artistic exhibitions would be possible, prompting the hybrid nature of the initial site. As time has gone on, and in-person events are occurring, I am shifting the site into a digital archive as it originally intended.} As a community-based initiative, the Cabildo de Regla processional re-initiation is constantly evolving, adjusting to the social conditions faced by the people of Regla, new historical information that may come to light, and/or the spiritual mandates communicated through ritual. This media documentation component, shared through this digital humanities project, also maintains a flexible approach in response to the community's needs first and foremost. Yet, Regla's influence as a central locale for Lucumí religious practices also connects it to the global network of practitioners, as well as those interested in Cuba's African Diasporic history. As the coordinator of the media documentation team, curator, and developer of the digital platform, while also participating as a researcher, I aim to provide a virtual space, independent of social media, to balance these local and global audiences. I envision this site where people can learn more about the project, the Regla community, Lucumí religion, and access the visual archive.

With no previous background in digital humanities, I created the reviewed iteration of this site under the address www.cabildoreglaexhibit.com after the first year of the COVID-19 pandemic, prompted by \goto{SSMF 2021: Rupture and Repair (Screening Scholarship Media Festival)}[url(https://ssmf2021.camrapenn.org/)]. I aimed to present a hybrid artistic exhibition and archive, reflecting the dual intent of the images requested by the community. That iteration was never intended to remain as the finalized site and many of the recommendations articulated by the reviewers were already in our future plans but had not yet been implemented. The helpful reviewer comments regarding the structure of the website, suggestions for this site's potential, and incorporation of pedagogical tools are aspects that I'd not yet considered, but now guides my approach towards organizing revisions and future updates. In the following paragraphs, I will outline my current plans to implement the reviewer's recommendations in addition to the changes I'd already planned to implement throughout 2023 and 2024. As I continue to learn more about digital archiving and site development, no doubt further ideas for future revisions will be realized, inspired by the suggestions made within the above review.

Since the review, I've changed the address of the site to \goto{www.cabildoderegla.org}[url(http://www.cabildoderegla.org)] and migrated the site to Wordpress to foster accessibility for those in Cuba. In this re-design I've also focused on the site's mobile view, as the majority of Cubans on the island would likely access it from their smart phones, rather than a computer. The homepage now reflects buttons to connect with the photo archive, project essays, collaborator pages, acknowledgements, and the artistic images page to improve site navigation. The archival image galleries have been edited, removing photos slated for artistic in-person exhibitions.\footnote{The first in-person exhibition is slated for Spring/Summer 2023 in Regla. We've had interest in exhibitions in the U.S., but it is important that the first one be in the Regla community.} Sample artistic images and information about potential exhibitions are now on a \goto{separate page}[url(https://cabildoderegla.org/artistic-images/)] within the site to demonstrate what we offer.

By the end of 2023, the visual archive will incorporate metadata descriptions of existing images in addition to the inclusion of the 2022 processional photos, not yet published on any platform. The reviewer's comments about better integrating the archival images and the explanatory essays are valuable critiques, which I will continue to work towards implementing. One initial strategy could be clarified once metadata and reference codes are added to each image, allowing us to specifically cite images within the essays. With specific regards to the \goto{processional image gallery}[url(https://cabildoderegla.org/exhibition/processional-gallery/)], I've organized the images to visually narrate the processional as it moves across the Regla landscape, integrating the years. However, this reviewer comment prompts consideration as to whether viewers should have the option to view the images by year as well. To do so would take additional skill building on my part, which I would not be able to achieve before the end of 2023 but is an option for the future.

To amplify the range of perspectives within the \goto{essay sections}[url(https://cabildoderegla.org/project-essays/)], additional public scholarship essay contributions from other research collaborators will be solicited and published between the end of 2023 and first half of 2024. This is one area that intends to further connect the research component of the project with public scholarship to foster discussion across the researcher, community, and religious practitioner communities about the histories and issues surrounding this localized example of Cuba's broader processional tradition.

By the end of 2024, two aspects are slated for incorporation into the site: use of the video footage and digital mapping. We have a cache of video footage taken by our media documentation team but have yet to develop it beyond the two videos found within the blog posts. Since this footage as it can be used for archival purposes, the creation of documentary films, and exhibition purposes, I aim to release a series of short films chronicling the process of the processional's re-initiation thus far. Given the centrality of how the processional fosters people's spiritual interaction with the landscape, mapping would help contextualize the images with the ritual activities they portray while increasing the interactive experience for site visitors. The acquisition of additional historical maps will aid in how we can visually connect to the historical narratives presented within the site to analyze processional routes and the formation of Regla's Black community. Existing projects centering mapping, especially those reviewed in past issues of this journal, serve as helpful guides towards implementing this component. This could also serve as the first step in integrating pedagogical tools recommended by the reviewers. One aspect that currently remains unresolved is how to use mapping to articulate the ways practitioners understand spiritual relationships to place, prompting innovative and creative ways of mapping that challenges 2-D assumptions of mapping-as-cartography. This has the potential to expand visual expressions of rhizomatic relationships between spiritual entities, landscape, and people. We will engage this idea, in addition to initiating conversations about developing further pedagogical tools during the next meeting of the project's collaborative network.

I wish to thank the editors and reviewers again for their thoughtful and insightful comments and suggestions. This review has inspired new directions for this digital project me to think about the site's potential as it evolves. Feedback is always welcome, and we have included a survey form on the site under the curator's note in both Spanish and English for visitors to provide insight should they chose to do so.

\thinrule

\stopsectionlevel

\page
\subsection{Alexandra P. Gelbard}

Alexandra P. Gelbard is an interdisciplinary social scientist, completing her Ph.D.~in Global and Sociocultural Studies with a concentration in Sociology at Florida International University in 2023. Her academic work engages African Diaspora community formation in Cuba, religion, cultural production, and street processionals. Her dissertation and current book project examine these themes within the Los Hoyos community of Santiago de Cuba and the traditional popular music genre of conga. She is originally from Washington, D.C. and her broader research scope encompasses African Diasporic cultural, spiritual, and liberation connectivity across the Atlantic world. She is also a photographer and is committed to integrating academic knowledge production and public scholarship through creative, accessible means. This includes experience within radio broadcasting, museum exhibitions, facilitating cultural exchanges between Cuban and U.S. artists, and additional consultancies with performing artists.

\stopchapter
\stoptext