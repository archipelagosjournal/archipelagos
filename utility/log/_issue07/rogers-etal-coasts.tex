\setvariables[article][shortauthor={Foote, Pérez Martínez, Rogers}, date={May 2023}, issue={7}, DOI={Upcoming}]

\setupinteraction[title={Coasts in Crisis: Caribbean Arts and Cultures After Hurricanes},author={Rebecca Elise Foote, Winnie E. Pérez Martínez, Charlotte Rogers}, date={May 2023}, subtitle={Coasts in Crisis}, state=start]
\environment env_journal


\starttext


\startchapter[title={Coasts in Crisis: Caribbean Arts and Cultures After Hurricanes}
, marking={Coasts in Crisis}
, bookmark={Coasts in Crisis: Caribbean Arts and Cultures After Hurricanes}]


\startlines
{\bf
Rebecca Elise Foote
Winnie E. Pérez Martínez
Charlotte Rogers
}
\stoplines


\startsectionlevel[title={{\em archipelagos} presents {\em Coasts in Crisis}},reference={archipelagos-presents-coasts-in-crisis}]

\quotation{What is the role of art amid climate disaster?} This is the fundamental question posed the team of the Coasts in Crisis project. They provide many examples, including their very own in the form of a spiraling curatorial space. Straddling the boundaries between art and scholarship, disaster response and media production, enviromentalism and library science. Coasts in Crisis offers us a model of curation and digital archiving that goes beyond the homogeneity of mainstream digital archiving systems, preserving the delicate balance between context over metadata. The project statement and the dialogue that follows below reflect the richness of thought that such digital exercises enable.

\stopsectionlevel

\startsectionlevel[title={Project statement by Elise Foote, Winnie E. Pérez Martínez and Charlotte Rogers},reference={project-statement-by-elise-foote-winnie-e.-pérez-martínez-and-charlotte-rogers}]

As Hurricane Maria pummeled Puerto Rico in the early morning hours of September 19th, 2019, Sarabel Santos-Negrón huddled with her mother in the bathroom of her apartment, hoping that the cement walls would protect them from flying debris. Amid the chaos, Santos-Negrón began to record the eerie sounds of the storm as a way of occupying her anxious mind. The wailing of the wind and the violent collision of household objects became the soundtrack to her immersive installation piece, \quotation{\goto{Groundscapes Displaced}[url(http://coastsincrisis.net/groundscapesdisplaced)].} When pianist and composer Alfonso Fuentes Colón emerged from his home on September 20th, he noticed that the native trees in his neighborhood, some of them hundreds of years old, had been uprooted and fractured by the storm. In the months without electricity that followed, Fuentes bought a battery-operated keyboard and composed \quotation{\goto{Ode to the Fallen Trees}[url(http://coastsincrisis.net/odetothefallentrees)]} in a state of inspired mourning. In the Florida Keys, Sally Binard salvaged wood from her home, destroyed in Hurricane Irma, and made a self-portrait from the wreckage: in \quotation{Afraid to Launch,} nails embedded in the wood form a halo around her face. What drove Santos-Negrón, Fuentes, and Binard to create in the midst of destruction? Why did they respond to loss with art? With the needs for shelter, food, and water often going unmet, why do the arts matter after a hurricane?

\placefigure[here]{\goto{{\em Afraid to Launch} (http://coastsincrisis.net/afterthehurricane)}[url(http://coastsincrisis.net/afterthehurricane)]}{\externalfigure[issue07/binard-afraid.jpg]}


These questions animate \goto{{\em Coasts in Crisis: Caribbean Arts and Cultures After Hurricanes}}[url(http://coastsincrisis.net/)], a digital humanities exhibit featuring art, music, poetry, and conversation from artists who experienced Hurricanes Irma and Maria in 2017. The site transforms a local event held at the University of Virginia on September 19th, 2019 into a freely accessible, curated digital art exhibition. {\em Coasts in Crisis} documents human ingenuity and creativity in the face of unthinkable loss. At once an artistic documentation of individual experiences and a collective expression of Caribbean survival, {\em Coasts in Crisis} offers several answers to the question of why art matters after disaster.

\startsectionlevel[title={Project Origins and Contribution},reference={project-origins-and-contribution}]

\quotation{Coasts in Crisis} began as an interdisciplinary artistic gathering on September 19th, 2019, the two-year anniversary of Hurricane Maria's landfall in the Caribbean. As part of the Environmental Humanities Convergences Festival at UVa, the live event brought together artists from Florida, the U.S. Virgin Islands, and Puerto Rico, all of whom were affected by the storms. They played music, displayed photography and installation art, read poetry, and discussed the role of art after a hurricane. The gathering recognized the exchange of ideas and collaborations that arise when artists occupy the same physical space.

The second phase of the endeavor is \goto{Coasts in Crisis: Caribbean Arts and Cultures After Hurricanes}[url(http://coastsincrisis.net/)], a website that transforms the physical event into a digital exhibit. Its content includes visual and aural responses to the hurricanes including poetry, still images of photography and mixed media, sound recordings, and a video of the artists' panel discussion from the live gathering on September 19th, 2019. The digital project also includes interviews with individual artists and a brief curatorial explanation for each artistic piece. Our main goal in creating the site was to showcase the archipelagic impact of and responses to the 2017 hurricane season in art, music, and literature. More broadly, this project seeks to use the digital realm to connect hurricane survivors through art, both within the archipelago and beyond it, thus contributing to and expanding the \quotation{digital turn} in Caribbean arts and letters.

\quotation{Coasts in Crisis} aims to mobilize art to foster solidarity across islands and to build relationships beyond the archipelago, outside of colonial dynamics, in both its physical and digital iterations. The colonial legacies of slavery, environmental degradation, and financial and political exploitation apparent today contribute to the discounting of Caribbean lives and landscapes in the aftermath of disaster. In the face of what artist David Berg calls the \quotation{Deafening Silence} of the U.S. towards its territories after the hurricanes, creative practices emerge as a means through which colonized peoples can make their voices heard.

\placefigure[here]{\goto{{\em Deafening Silence} (http://coastsincrisis.net/deafeningsilence)}[url(http://coastsincrisis.net/deafeningsilence)]}{\externalfigure[issue07/berg-deafening.jpg]}


Beyond the popular term \quotation{artivism,} poetry, music, and art matter after disaster because they can reframe how we see and hear the world to offer innovative solutions to ecological and social crises. The digital exhibit of \quotation{Coasts in Crisis} seeks to make the artists' insights freely accessible to anyone with an Internet connection by illuminating points of commonality with other peoples and environments affected by tropical storms.

Hurricanes Harvey, Irma, and Maria in 2017 caused severe damage across Texas, Florida, and the Caribbean, disproportionately affecting those who live amid precarity on a daily basis and, as with Hurricane Katrina in 2004, laying bare the social inequities that predated the storms. Moreover, the colonial status of Puerto Rico, St.~Croix, and other non-sovereign areas of the Caribbean has been driven home by the government's response to the damage Hurricanes Irma and Maria caused in September 2017.\footnote{This section of the essay draws on Charlotte Rogers' forthcoming article \quotation{Art and Debt in the Oldest Colony: Creative Resistance in Contemporary Puerto Rican Culture,} {\em The Routledge Companion to Twentieth and Twenty-first Century Latin American Literary and Cultural Forms}. Edited by Guillermina De Ferrari and Mariano Siskind. Routledge, publication scheduled for August 2022.} In the summer of 2018, local and federal officials calculated that Hurricane Maria caused only sixty-four deaths in Puerto Rico, while a study published in the New England Journal of Medicine conservatively estimated that 4,645 people died as a direct or indirect result of the storm.\footnote{Kishore, Nishant Domingo Marqués, Ayesha Mahmud, Mathew V. Kiang, Irmary Rodriguez, Arlan Fuller, Peggy Ebner, et al., \quotation{Mortality in Puerto Rico After Hurricane Maria,} New England Journal of Medicine 379 (May 2018): 162-70. DOI: doi.org/10.1056/NEJMsa1803972, 167.} The disparity between the local administration's death count and the professional study reveals an official repudiation of Caribbean lives. President Trump's callous gesture of tossing paper in towels at hurricane survivors in 2017 further embodies the governmental diminishment of human suffering, especially since most residents were struggling without water or power.\footnote{Kenny, Caroline. \quotation{Trump Tosses Paper Towels in Puerto Rico Crowd.} October 3, 2017. https://www.cnn.com/2017/10/03/politics/donald-trump-paper-towels-puerto-rico/index.html} Relief policies reflect the institutional disrespect for citizens of U.S. possessions: St.~Croix and Puerto Rico received less aid more slowly than other states affected by 2017 hurricanes Irma and Harvey in Texas and Florida, leaving impoverished populations on the islands with fewer governmental resources than those on the mainland in the aftermath of storms.\footnote{Willison, Charley E., Phillip M. Singer, Melissa S. Creary, and Scott L. Greer. \quotation{Quantifying Inequities in US Federal Response to Hurricane Disaster in Texas and Florida Compared with Puerto Rico.} BMJ Global Health, 4, no. 1, (January 2019) p.~e001191. doi:10.1136/bmjgh-2018-001191, 1.}

Across the hurricane zone, art converges with social justice activism as a means for residents of the tropics to make their voices heard in the public sphere. They have achieved some significant gains. In July 2019, for example, Puerto Rican Governor Ricardo Rosselló resigned in the face of massive demonstrations against his treatment of his fellow islanders. Art and music were instrumental in channeling public outrage through protest songs like \quotation{Sharpening the Knives} {[}\quotation{Afilando los cuchillos}{]} by Residente with iLe and Bad Bunny and \quotation{Ricky I Dumped You} {[}\quotation{Te Boté Remix}{]} by Casper, Nio García, Darell, Nicky Jam, Bad Bunny, and Ozuna.

Across the archipelago and the mainland, art and music have emerged as key modes of refuting the discounting of vulnerable lives by the government. As Marianne Ramírez Aponte, the director of the Museo de Arte Contemporáneo de Puerto Rico, has written, art spaces are \quotation{sites of ideological power and potential social agency and change} amid the storm recovery.\footnote{Ramírez-Aponte, Marianne. \quotation{The Importance of Politically Engaged Artistic and Curatorial Practices in the Aftermath of Hurricane María,} {\em The Aftershocks of Disaster: Puerto Rico Before and After the Storm}, edited by Yarimar Bonilla and Marisol LeBrón, Haymarket Books, 2019, 163.} The most powerful contemporary art does not just call the people to attention and action, however; it is action in itself, action that reclaims, restores, and revalorizes Greater Caribbean lives and landscapes. \quotation{Coasts in Crisis} is therefore not only an exhibit, but also an action that brings art practices out of traditional museum settings. In doing so, it seeks to establish alternatives to the colonial paradigm of dependency on the United States and to feature people of the Caribbean advocating for their own visions of the future through art.

\stopsectionlevel

\startsectionlevel[title={Project Design and Description},reference={project-design-and-description}]

\placefigure[here]{\goto{Three 2017 hurricane paths (http://coastsincrisis.net/)}[url(http://coastsincrisis.net/)]}{\externalfigure[issue07/site-homepage.png]}


The homepage of \quotation{Coasts in Crisis} digital features a map of the Caribbean overlaid with the pathways of Hurricanes Maria, Irma, and Jose, according to the National Oceanic and Atmospheric Association. This design feature presents the website's navigational structure and plays on the ubiquity of meteorological maps in storm tracking. The ways these maps partition geography into a predetermined grid reveals a persistence of colonial logic that hurricanes, with their unpredictable paths, call into question. Cutting through the longitude and latitude lines, the storms' routes destabilize the legacy of colonialism implicit in cartography, a tension that we reconfigure by showcasing the nonlinear aesthetics of Caribbean artistic expression in a flat and linear digital form. Just as the unruliness of the hurricanes' cones of uncertainty escapes scientific specificity, the Caribbean escapes linear histories. In this way, \quotation{Coasts in Crisis} digital contributes to ongoing discussions within Caribbean Digital Humanities that ask linear digital forms to function in nonlinear ways, as shown by projects such as Laura Vargas Zuleta's \quotation{\goto{Hadriana in Context: Timelines}[url(http://archipelagosjournal.org/issue06/vargas-real-life.html)],} which uses Knight Lab timelines to articulate Caribbean temporalities. To communicate this nonlinearity in our own project, we took the spiral as an important feature of Caribbean life, from the shape of the hurricane vortex to the amalgamation of histories and cultures in the region. This geometry enables a theorization of the spiral as a sense of repetition with difference, which we use as a governing aesthetic for the site.

We interpret these storms as working in patterns, and the grid allows us to place three separate trajectories from the 2017 hurricane season---Maria, Irma, and Jose---in the same space, a counterpoint to media that narrates hurricanes as exceptional events. Because they share the same digital space on the site, we begin to see these storms as interconnected both to each other and to human and non-human ecologies across time in the Caribbean. With this project we can highlight, then, a deeper, more-than-human history that reveals the increasing frequency and intensity of these storms due to climate change. Each of these storms speaks to its own specific contexts and localities and yet, when brought together in this digital project, they become inflection points stretching forward and backward in time. They extend beyond their individual contexts to make the longer pattern of storms in the Caribbean differently legible.

Navigation for the site relies on a design that invites user engagement with the homepage. From an aesthetic standpoint, we elected to use a gray and teal color palette; teal evokes the Caribbean Sea, while the gray background allows us to showcase the featured artistic works without distractions. The teal font and buttons show well against the gray to make navigation readily visible. The spinning hurricane icons cue users to hover the mouse over the map, which then brings buttons into view for the three pathways on the site. The pathways---Entangled Ecologies, Creative Resistance, and Predictable Precarity---correspond to the three hurricane trajectories. These icons mimic the graphics used in the media and also evoke the symbol of the Taino deity of storms, {\em juracán}, whose spiral form proliferates in Indigenous arts of the Americas.\footnote{Ortiz, Fernando. El huracán: su mitología y sus símbolos. Fondo de Cultura Económica, 1986, 107-128.} The icons thus indicate the continuity of hurricanes' importance in human existence and incorporate something of the storms' swirling motion into an otherwise static site. We would like to incorporate more of this sense of movement into the site to continue challenging linearity. When the buttons appear on the homepage, so too do a series of dots along each hurricane path that visually represent stops showcasing the artists' work along each pathway. When users hover the mouse over the map, they activate the pathways as lines of critical inquiry.

Like the Caribbean spiral, the site's three pathways destabilize the conventional user experience. For one, their odd number resists the balance of an even number of pathways and therefore extends the project's interrogation of linearity in space and time. Through the pathways, the site hands itself over to the user: in order to move through the exhibit, the user must choose one of them. While designing the homepage, we intentionally placed the pathway buttons so that none would have primacy over the others. Each pathway therefore offers a different opportunity to consider these storms as local iterations in a larger pattern within the broader context of global climate change, which results in more powerful and frequent storms. Placing the buttons and hurricane icons posed a design challenge; we had to override the theme with our own CSS to give them highly-customized locations on the homepage. Even design choices as small as button placement raise questions of how to represent the storms in a digital space.

Users' movement along the pathways also follows the spiral's repetition with difference. We took our cues from the spiral by including buttons on the bottom of each pathway stop that give the option to either \quotation{Spiral Forward} or \quotation{Spiral Back.} Here, we once again felt ourselves pressing against digital linearity with the Caribbean spiral. Although we give the option to move either backwards or forward along the pathways, our project's questioning of linear time has its limits. Moving backwards and forwards along a timeline ultimately confines navigation to a bidirectional movement that is visually similar to a line's. To this end, we wonder how we might incorporate unpredictability more thoroughly into the pathways. What, for example, would a pathway look like in a more circular trajectory? Something of the interaction between the line and the circle in the spiral form shows through in how each pathway end returns the user back to the homepage to select a new pathway. Users return to the homepage with a difference; they have now seen the artistic expressions on one pathway and bring this knowledge forward to another pathway or to sites beyond our project. In this way, the homepage becomes uncanny, both familiar and strange. This tension of similarity and difference extends across and between the pathways. Their lines of critical inquiry work relationally, following Glissant's {\em Poetics of Relation}, as concepts, motifs, and the works of particular artists appear in more than one pathway.

The artwork evokes some of the sensory panorama of the hurricane, even in the flat digital realm, throughout the pathway stops. Each stop has its own title corresponding to the artists' work on the page. The work ranges from photographs of visual art and installations to poetry, music, and Santos-Negrón's recording of Hurricane Maria. On each page, our priority was to maintain the integrity of the featured art to the best of our ability in this remediation through careful design decisions. To this end, we borrowed features used in physical museums, such as the curatorial shadowbox that sets off text belonging to each piece, to limit our intrusion on the work and to gesture towards the experiential qualities of our project. That said, we continue to consider the layout for the pages in order to maximize aesthetic appeal and user facility in balance with the display requirements for each piece of art. Some of the work on the site posed challenges here; for instance, it is difficult to translate the more tactile qualities, whether the jagged edges of recycled objects or the physical manipulation of tarot cards, into a digital space. Nevertheless, some of our design choices attempt to communicate this sense of manual craft, as shown by the typeset we used to display Nicole Delgado's poetry.

\placefigure[here]{\goto{{\em Las casas vacías} (http://coastsincrisis.net/ageofabandonment)}[url(http://coastsincrisis.net/ageofabandonment)]}{\externalfigure[issue07/delgado-poem.png]}


We hope, too, that the individual care we gave each piece of art in its placement on the page is evident. The pages also include audio from interviews with the featured artists featured. Because Drupal is such a heavy content management system, we decided to outsource the hosting of these audio files to YouTube and then embed the videos into each page.

\stopsectionlevel

\startsectionlevel[title={Credit and Preservation Concerns},reference={credit-and-preservation-concerns}]

For the creation of {\em Coasts} digital, Charlotte Rogers procured funding from the Center for Global Inquiry and Innovation at UVa to support hosting costs and the paid work of graduate students Rebecca Elise Foote and Winnie E. Pérez Martínez, which included site design, implementation, and curation. The design team would like to thank the staff of UVa's Scholars' Lab, especially Jeremy Boggs and Ammon Shepard, for their patience, time, and support. Professor Rogers thanks graduate students Elise and Winnie for their dedication and insightful contributions to the project on top of their coursework and dissertation writing. All three of the site's creators thank the seven Greater Caribbean artists for their commitment to the project over the past two years and for allowing their work to be displayed on the site. Lastly, special thanks to Caroline Whitcomb, the event coordinator, for her continued support and planning during the artist gathering and the digital project launch. As the funding from UVa winds down, we seek a reasonably durable future for the project.

In addition to meeting the scholarly standards we have set for the project, we also aim for it to remain viable and reachable beyond our respective involvements with its development. In the article \quotation{The Directory Paradox,} Quinn Dombrowski underscores the challenges and difficulties that plague the reality of keeping a digital humanities project alive over a long period of time.\footnote{Dombrowski, Quinn. \quotation{The Directory Paradox,} People, Practice, Power: Digital Humanities Outside the Center, edited by Anne McGrail, Angel David Nieves, and Siobhan Senier, University of Minnesota Press, 2021, 91-92.} Access to sufficient time, labor (paid or not), and recurrent funds constitute salient high-risk factors that can either consolidate the existence of a project or, as more often happens, place an expiration date on the lives of hundreds of digital endeavors despite creators' and collaborators' intentions. Dombrowski's eloquent cautionary tale pushed us to reconsider our project goals and priorities to guarantee longevity for {\em Coasts}.

To carve a future for \quotation{Coasts in Crisis}, we evaluated the site infrastructure and developed a plan to migrate and transition its components into a framework and hosting environment better suited to its needs and to our long-term goals for the project. Our parameters for the post-launch examination of the site centered around maintenance, how Caribbean users engage with the Internet, and the preservation of {\em Coasts} as an art exhibition website that challenges the linear and hierarchical understanding of experiencing a hurricane. Experiencing the site through this assessment allowed us to identify an urgent need to migrate the site and to change our hosting as well as domain services.

At present, the site runs on a privately-owned domain name with low but recurrent renewal and shared-hosting costs through Reclaim Hosting. While the expenditures and manual maintenance are low, they do require consistent periodic labor, funds to ensure the stable online presence of the site, and someone to pay these funds. In our next iteration, we plan to move from Reclaim Hosting to GitHub Pages because it offers a free hosting service that includes a second-level domain. This transition will eliminate any need of finding funds and remembering to renew both the hosting and domain services. {\em Coasts} will be able to operate and remain stable regardless of changes in its development and administration team. By hosting the exhibition's files on GitHub, we allow the possibility for future interested individuals to contribute, expand, or re-imagine the project. Through this process, {\em Coasts} gains a durable status that provides opportunities for its growth without the need for hands-on updates from its creators and caretakers.

Similar preservation and access concerns surfaced during the development stage. We used the popular CMS platform, Drupal, which spurred us to consider whether alternative frameworks would be more conducive to our goals. Primarily, we want to make sure that users with low-bandwidth internet connections can access the website and that the development learning curve will not be too steep for those who might want to work on the project, but who do not have a background in development. Initially, we considered using Omeka or Wax, but we thought it would be easier to produce a highly-customized design and site structure in Drupal. With its robust user and developer community, Drupal already had plenty of online forums to help with troubleshooting inquiries and optimizing our interface.

In practice, Drupal caused many problems, including pushing back the launch date for \quotation{Coasts in Crisis} digital. Its decentralized handling of HTML through its TWIG-based template system turns adding and editing CSS rules into an involved process for beginners. Likewise, its variety of components complicate the process of styling and properly structuring the content elements of our site, which is especially trying for troubleshooting. Drupal's automatic update option broke our site twice, forcing us to resort to manual updates. At first, we did not realize this was the issue because by default, Drupal's PHP and scripts do not provide verbose error messages unless they are manually indicated to do so. This feature proved a major setback in our development process since we had to troubleshoot blindly before we managed to turn on the setting to display detailed error messages. Lastly, one of our ongoing concerns is the server-heavy components that make the CMS versatile but that also hinder the speed with which the site can be visited and developed. It requires users to have a stable internet connection to guarantee access. These difficulties proved that in order for the project to work as intended and to increase its chance of success, our best option was to move away from Drupal. Nevertheless, this experience with Drupal helped us grow as digital humanists and improved our ability to scope the needs of similar digital projects.

\stopsectionlevel

\startsectionlevel[title={Future Steps and Expectations for Sustainability in Caribbean DH},reference={future-steps-and-expectations-for-sustainability-in-caribbean-dh}]

Our interest in grappling with the experience of hurricanes in the Caribbean necessarily leads us to confront the issue of sustainability in the region's infrastructures. We aim to make the academic and artistic collaboration that is \quotation{Coasts in Crisis} widely accessible. Often, strong internet connections and services are not guaranteed in the Caribbean. Intermittent access to the Internet and frequently interrupted electric service, if available at all, leads Caribbean users to rely heavily on mobile devices to stay connected. These devices require less electric charge and have access to mobile Internet networks, however unstable their connections may be. Further, purchasing a smartphone that can connect to the web and carry out many of a computer's functionalities costs less than acquiring both a phone and a computer, especially since people can buy phones secondhand or exchange them.\footnote{Here we draw from Winnie's experience growing up and living in Puerto Rico for over twenty years while observing this cultural practice. David Nemer and Padma Chirumamilla observed a similar phenomenon among the population of Brazilian favelas in the city of Vitória in their essay \quotation{Living in the Broken City: Infrastructural Inequity, Uncertainty, and the Materiality of the Digital in Brazil.} {\em digitalSTS: A Field Study Guide for Science & Technology Studies}, edited by Janet Vertesi and David Ribes, Princeton University Press, 2019.} With this in mind, we restrained our search for a Drupal alternative to light frameworks and static site generators that require minimal upkeep and, most importantly, that can run on spotty internet connections.

Our decision to turn to a lighter framework draws on the work done by the Global Outlook in the Digital Humanities collaborative group, especially their concept of minimal computing. This concept contours a set of flexible practical principles to scope and carry out conscientious digital projects tailored to a specific technical, cultural, and academic context. Both Alex Gil and Élika Ortega advocate for its implementation using the following characteristics: \quotation{ease of use, ease of creation, increased access, and reductions in computing - and by extension, electricity.} \footnote{Gil, Alex, and Élika Ortega. \quotation{Global Outlooks in Digital Humanities: Multilingual Practices and Minimal Computing,} Doing Digital Humanities: Practice, Training, and Research, edited by Constance Crompton, Richard J. Lane, and Ray Siemens, Routledge, 2016, 29.}Thinking of our project's goals and our intended audience with attention to minimal computing, we reconsidered the current state of our project and discussed an overhaul of our technical conditions to improve the site and to extend its accessibility.

In consultation with the technologists at the University of Virginia's Scholars' Lab, we decided that migrating our site from Drupal to Jekyll would be the best course of action to support access and sustainability of the digital exhibition. Jekyll is a static site generator with a large community of users that has gained popularity within the DH community in the last few years. While Jekyll lacks a sophisticated admin to create user accounts or a collaborator panel to generate, edit, and publish new content, it uses Markdown to simplify adding content directly to pure HTML. Moreover, since the project has a small dataset and few development needs, the different levels of user access that Drupal offers goes beyond the technical needs of a small project like ours. Rather than downscaling, our move away from Drupal and into Jekyll allows us to use a platform better suited to the needs of a small scale DH project with a bespoke and ambitious design.

We plan to abide by the principles of mobile-first development to provide a simpler version of the site that accurately conveys the questions and conversations from both the in-person event and the large-screen website on phones and similar devices during the migration and optimization process. As part of this major revision, we will work on creating and uploading project documentation on our GitHub repository to provide a centralized account of our rationale that includes step-by-step guides for the more complicated portions of the project. Documentation is a vital yet frequently overlooked and tedious portion of responsible development; arguably, it is one of, if not the most, important output generated as part of a project because it is crucial to the project data's longevity.\footnote{Warwick, Claire, Isabel Galina, Jon Rimmer, Melissa Terras, Ann Blandford, Jeremy Gow, and George Buchanan. \quotation{Documentation and the Users of Digital Resources in the Humanities,} Journal of Documentation 65, no.1 (Jan 2009): 33-57. DOI: doi.org/10.1108/00220410910926112, 34.} As each of the original creators and collaborators move on to other endeavors, we want to leave an independent project with detailed and thorough documentation that can enable anyone who is interested in the project to expand or modify our work. We believe that \quotation{Coasts in Crisis} will be ready to exist sustainably upon completion of these discussed changes. It will gain independence from the labor of its creators while possessing the necessary components to grow and support discussions about digital humanities work in Caribbean Studies.

As it stands, \quotation{Coasts in Crisis} still has a long way to go to reach our initial vision, but its current form nevertheless offers a free and public glimpse into the work and perspectives of artists who lived through the 2017 hurricanes. As a critical curatorial initiative, {\em Coasts} transforms an ephemeral gathering of artists into an online exhibition that both guides and misguides users in an approximation of the confusion experienced when living through a hurricane and its aftermath. In our ideal version of the project, we planned to provide an immersive, interactive, visual and aural experience that could capture a snapshot of hurricane survivors' feelings and reactions. Creating a customized, spiral-inspired interface for the artwork seemed like the most straightforward approach to reach our goal, but engaging with this multidirectional shape proved to be a daunting albeit generative challenge. Mimicking the spiral movement of hurricanes, we have come to learn, functions as a line of critical inquiry in its own right rather than simply a display format.

Along with using the site as a way to disorient and expose visitors to hurricanes through art, we planned to include brief insights from the artists gathered during the event. Although we have added links to their interviews on YouTube, we believe there are still other ways to state the connection between artists and the rationale behind their art more directly. We are currently grappling with what these other ways might mean and the forms they could take. It was similarly complex for us to arrange and introduce the artwork to follow the spiral shape. At present, we are unsure if we can continue pursuing the spiral form in the future iteration of our exhibit, but there are a few other goals that remain feasible and important additions to the site. In {\em Coasts}' current state, there exists no mobile-friendly version of the site. Further, the site was not developed following best practices for accessibility. This is a crucial dimension of the project that we will incorporate in future versions. In this case, the question of accessibility poses a crucial challenge: how to remain true to the customized and deliberately nonlinear design of the site while making it open to the widest audience possible?

In the five years that have passed since the 2017 hurricane season, Santos-Negrón's mother's home has been razed and rebuilt. New growth is visible in the Yunque Forest, but many native, centuries-old trees are gone forever. Blue tarps remain and now even replace roofs across much of St.~Croix. The coquí song has returned to Puerto Rico, but thousands of people displaced after the storm have not, including Jo Cosme, one of the artists featured on this site. As Hurricanes Irma and Maria recede with each new hurricane season, art remains to bear witness to the living and the dead, and to call attention to the injustice of local and global responses to Caribbean disasters. Cosme's deck of tarot cards, \quotation{Crónicas de un futuro catastrófico} narrates a litany of corruption, disaster capitalism, and official disregard for Caribbean life after the storms that reveals the injustice in disaster responses. Cosme's photography series \quotation{Welcome to Paradise} takes on new resonance as wealthy investors from outside the archipelago purchase second homes while living costs become untenable for Caribbean residents.

\placefigure[here]{Welcome to Paradise I}{\externalfigure[issue07/cosme-welcome.jpg]}


The digital iteration of \quotation{Coasts in Crisis} aims to endure long after the glare of media attention has turned its gaze elsewhere. As time passes, the meanings of these images, recordings, and interviews will change as the visitors to the site reflect on the latest hurricane season. We might remember that before Hurricane Maria, David Berg often photographed a silk-cotton tree that stood sentinel for centuries, a tree whose witnessing reached back to the horrors of enslavement. After the storm, Berg photographed this same tree felled by the wind, and the swath of open sky once filled by its branches in his \quotation{Regrowth} series. Our project points, time and again, to these empty spaces that still linger, and it offers a testament to regeneration.

\placefigure[here]{\goto{{\em Regrowth I} (http://coastsincrisis.net/slavetrees)}[url(http://coastsincrisis.net/slavetrees)]}{\externalfigure[issue07/berg-regrowth.jpg]}


\thinrule

\stopsectionlevel

\stopsectionlevel

\startsectionlevel[title={{\em archipelagos} Review of Coasts in Crisis},reference={archipelagos-review-of-coasts-in-crisis}]

\startsectionlevel[title={Contribution},reference={contribution}]

How can art help people navigate the revelations and transformations brought about by climate disaster? Coasts in Crisis explores this question through a digital exhibition of music, photography, poetry, installations, and visual art created in the aftermath of the 2017 hurricane season. The project is organized around three thematic sections; each section brings together the work of artists based in different locations and working with different media, as well as interviews with the artists and interpretive text.

The thematic organization of the materials encourages audiences to think about connections and shared experiences across the Caribbean, rather than isolated case studies centered on individual islands or individual hurricanes. At the same time, however, the project could pay closer attention to the geography it maps out. The homepage states that the project is about the Greater Caribbean. The artwork, indeed, speaks to experiences in St.~Croix, Puerto Rico, and the Florida Keys. Yet, when considered in relation to each other, these places speak to a particular geography within the Greater Caribbean---all three are territories under U.S. rule. Within the context of this project, what is gained and what is lost when these connections produced by U.S. empire remain detectable but unacknowledged?

Further observations emerge from the project's geographical positioning. The curatorial text opening the Entangled Ecologies section states that \quotation{these works reveal to us that the lives and landscapes of Puerto Rico are deeply and beautifully entangled.} Yet the section also features photographs taken in St.~Croix, which is not part of the Puerto Rico archipelago.

The regional dimensions of the project also invite questions about existing and potential audiences. Among the six artists whose work is showcased, four are from Puerto Rico. The curatorial text is in English. Given how Puerto Rican art features so prominently, are there plans to make the project more accessible to Spanish-speaking audiences? Four artist interviews are in Spanish, while two are in English. Are there plans to provide bilingual transcriptions or subtitles?

The author's project statement declares that by bringing art out of museums into a digital space, \quotation{it seeks to establish alternatives to the colonial pattern of dependency on the United States.} While the art and interviews do center Caribbean voices, it is unclear how the digital format of the project disrupts colonial dependency on the United States, especially considering the location and source of the project's funding, labor, and institutional support. It would be helpful for the authors to identify the specific forms of dependency that the project runs the risk of reproducing, or that the project wishes to challenge, and discuss the measures they have taken to do digital humanities otherwise. For example, how does the project, through its processes or digital format, address power asymmetries between artists and audiences based in the Caribbean and the host institution based in the United States? How might it contribute to local forms of autonomy in Puerto Rico, St.~Croix, and the Florida Keys? Or how does the project otherwise conceptualize and challenge dependency?

\stopsectionlevel

\startsectionlevel[title={Credit},reference={credit}]

The project has a dedicated page crediting the artists and project team with photos, short biographies, and descriptions of project roles for each contributor, including students. A resources page cites relevant digital humanities and scholarly works. With respect to the interviews, audiences might want to know who conducted them and under what circumstances.

\stopsectionlevel

\startsectionlevel[title={Design},reference={design}]

The project's design adds another layer of complexity to the exhibition materials by exploring the tension between colonial projects and Caribbean forces, both human and natural. This tension manifests visually through the grid of latitude and longitude lines that constitute the background for the map on the home page, on one hand, and the hurricane pathways that wind across the map, on the other hand. While the grid represents colonial efforts to make the Caribbean legible, the hurricane paths---each of which represent one of the three thematic sections---reference the uncontrollable nature of Caribbean expressions generated by artists as well as storms. The design guides audience engagement with this tension through the spiral motif that ties the exhibition together. Visually, the motif manifests in the hurricane icons for each pathway. The spiral also guides the audience journey through the exhibition, as each thematic section in the end loops back to the homepage, where audiences may select another pathway. This design strategy is effective in that it encourages audiences to consider similarities and differences across locations and specific hurricanes, thus disrupting linear configurations of space and time represented by the cartographic grid.

Although the spiral subverts the colonial map, some of the other visual elements generate questions about the fit between the map and the exhibition's content and aims. To what extent does the map show the Greater Caribbean evoked in the artwork? The map shows the arc of Caribbean islands from Cuba and the Bahamas to Trinidad and Tobago, yet does not show the Florida Keys, where some of the featured art was produced. Neither is it clear why some islands or countries are shaded in dark grey, while others are light grey. The grid in the background, while it evokes latitude and longitude lines, does not accurately represent these measures. Audiences might also be curious about the dots that mark each hurricane pathway. While the dots convey how the journey through each thematic pathway consists of several stops, the number of dots does not match up with the number of stops that mark each journey.

While the artist interviews are embedded on the website, links to other sound files and external resources should open in a separate window.

\stopsectionlevel

\startsectionlevel[title={Preservation},reference={preservation}]

With the long-term life of the project in mind, the authors plan to move the site from the private domain name on Reclaim Hosting to GitHub Pages. As a small-scale digital project with few development needs, they also plan to overhaul the project so that it better aligns with the concept of minimal computing by migrating the site from Drupal to Jekyll. Part of this overhaul involves the development of a mobile version of the website with plans to upload documentation to the GitHub repository.

\stopsectionlevel

\stopsectionlevel

\startsectionlevel[title={Response to {\em archipelagos} review of \quotation{Coasts in Crisis} by Rebecca Elise Foote, Winnie Pérez Martínez, and Charlotte Rogers},reference={response-to-archipelagos-review-of-coasts-in-crisis-by-rebecca-elise-foote-winnie-pérez-martínez-and-charlotte-rogers}]

We thank the reviewer/s for this rigorous and precise review of \quotation{Coasts in Crisis.} The queries regarding the implicit role of U.S. empire in this project and our own positionality helped us reflect on our motivations as well as those of our home institution. We now address these issues explicitly in the second half of the \quotation{About} page of the site. \quotation{Coasts in Crisis} was funded by the University of Virginia, an educational institution located on unceded Monacan lands and originally constructed with enslaved labor. The University's founder, Thomas Jefferson, sought to isolate and impoverish Haiti after it liberated itself from French imperial control in 1804. As contemporary members of UVA's academic community, we recognize our university's role in U.S. attempts to marginalize and indebt the nations and peoples of the Caribbean. These power asymmetries linger in our economic and infrastructural capacity to host events and secure payment for the artists and the student workers. The project aims to take a step towards redressing that unequal situation by elevating the work and voices of artists and students who live under the ongoing shadow of U.S. colonialism and those whose families emigrated from the Caribbean and Latin America to the United States.

Acknowledging these local roots of U.S. imperialism within the Caribbean offers an enriched sense of how the United States remains complicit in the perpetuation of power imbalances that exacerbate the vulnerability of archipelagos to hurricane damage. More broadly, the increasing frequency and severity of the storms show how anthropogenic climate change, spurred in no small part by the fossil fuel consumption inherent in global capitalism, compounds the ongoing disaster of coloniality. The continuing extraction of wealth from these territories in the forms of debt, taxes, and unsustainable coastal construction renders the islands' frail social and physical infrastructures even more precarious. Lastly, this connection highlights the disparity in governmental aid for hurricane zones in the continental United States in contrast to the U.S. territories in the Caribbean.

We aim to counteract patterns of dependency by educating our increasingly diverse academic community and creating opportunities for artists to connect across islands and continents. The dual purpose of the original 2019 \quotation{Coasts in Crisis} event was to educate the UVA community about the role of art in responding to hurricanes in the Greater Caribbean and to create a sense of belonging and visibility for the Hispanic and Caribbean communities that are growing within this predominately white institution. Now, more than three years after the event, the artists remain in contact with each other. Just as the artists at the event only connected through the gathering at the University of Virginia, the site offers an opportunity for its users to \quotation{meet} the artists and their work by moving through the site. The original objectives informed the selection of collaborators on the site: our graduate student web designers, Winnie Pérez-Martínez and Rebecca Elise Foote, identify as Puerto Rican and Chicana, respectively, and bring their own experiences with the 2017 hurricane season to this project. The graduate students who conducted the interviews with the artists are of Latinx, Puerto Rican, and white heritage. The digital iteration of the project aspires to be accessible to a broader archipelagic and continental public, especially as the geoformal distinctions between them are eroded by climate change's rising seas and more powerful and frequent hurricanes.

Regarding the benefit of the project to the artists themselves, we articulate the project's aims as follows:

\startitemize[n,packed][stopper=.]
\item
  To amplify the work of Caribbean artists living in precarious communities after the 2017 hurricane season
\item
  To establish opportunities for intercontinental and archipelagic connections facilitated by artistic engagement
\item
  To begin an ongoing conversation on the relationship between art, disaster, and coloniality
\stopitemize

We have made several revisions to the site in light of the reviewer's generous and substantive feedback on the site's design and navigation. We have revised the text for the opening page of \quotation{Entangled Ecologies} to reflect that the islands of St.~Croix and Puerto Rico are explored in this pathway. In line with the reviewers' suggestions, we added the Florida Keys to the map on the home page, removed the black dots on the hurricane pathways, changed the background color of all islands to black, and modified all external links so that they open in a new window when clicked.

The site's main navigation image, a stylized re-imagining of the NOAA hurricane map, serves an important interactive purpose for site visitors. One of the drawbacks to this approach is that it makes superimposing the main navigation map on the correct latitude and longitude lines very difficult because it requires designing from scratch an accurately scaled map of the distance and sizes of all islands in the archipelagos' image and grid background image, none of which we could find readily available online. Instead of focusing on an accurately scaled representation of the region, we uploaded an image evoking the grid format as a nod to the map that inspired the design.

Regarding translations, one of the original goals of the project was to create a multilingual site reflecting the linguistic diversity of the Caribbean. The relationship between Spanish-speaking Caribbean audiences and the English-dominated spaces of academic inquiry is complex and riddled with colonial tensions. Our critical understanding of such a relationship led us to mark Spanish translations a key component in our initial plans to build the Coasts in Crisis site. We decided to publish Coasts in English first, understanding the ethical limitations of this approach, because all of the collaborators shared it as a {\em lingua franca}. At the time, we did not have a clear grasp of how much time it would take for us, as first-time Drupal users, to thoroughly complete our full vision for Coasts.

There is, perhaps, a relatively simple method for adding automated translations to our site, but the cost of implementing it, in the long run, outweighs its immediate benefits for our particular context. Drupal has four core dedicated modules for managing a multilingual website, but we avoided this approach based on our preservation plan. In the summer of 2023 we will migrate the digital exhibit from Drupal 9.5.4 to \goto{Jekyll}[url(https://jekyllrb.com/)], a popular static site generator (or SSG). Jekyll will allow Coasts to live on in a stable, low-maintenance environment hosted by GitHub Pages. Within this context, installing and troubleshooting Drupal's four core multilingual modules poses two conflicts. First, Jekyll's pragmatic approach to web publishing lacks the complexity to integrate Drupal's multi-module process. The second complication stems from this feature divergence: all the labor required to install and troubleshoot the modules, before producing any translations, would be entirely lost in the migration process to re-home Coasts in a Jekyll site. At present, this context prevents us from including Spanish translations on the site.

We have revised our preservation plan with the help of Jeremy Boggs, the Head of Research and Development at the Scholar's Lab at UVA, and plan to complete the migration by July 10th, 2023. Winnie Pérez Martínez has volunteered to be in charge of the migration and the documentation writing. Unfortunately, at this time we lack the resources to produce a stable mobile version of the site, but we do not discard the possibility of making one in the future.

\thinrule

\stopsectionlevel

\page
\subsection{Rebecca Elise Foote}

Rebecca Elise Foote is a Ph. D. candidate in the department of English at the University of Virginia. She received a BA in English and Hispanic Studies from Davidson College in 2017. She is currently writing a dissertation on contemporary Latinx poetry and performance theory and holds the Irby Cauthen Fellowship through the Jefferson Scholars Foundation. She is a former Praxis Fellow (2020-2021) in UVa's Scholars' Lab, the university's center for digital humanities.

\subsection{Winnie E. Pérez Martínez}

Winnie E. Pérez Martínez is a PhD student in Spanish and Digital Humanities in the Spanish, Italian, and Portuguese Department at the University of Virginia, where she is a recipient of the Interdisciplinary Doctoral Fellow in Caribbean Literatures, Arts, and Cultures. Her main research interests concern the collective imaginaries of technology in the Anglophone, Francophone, and Hispanophone Caribbean of the 20th and 21st centuries. She traces these imaginaries through representations of technology within Caribbean science fiction and urban planning.

\subsection{Charlotte Rogers}

Charlotte Rogers is the Lisa Smith Discovery Chair Associate Professor of Spanish at the University of Virginia, where she specializes in twentieth- and twenty-first-century Latin American and Caribbean environmental humanities. Her work takes a comparative approach to representations of tropical ecologies in contemporary literatures, arts, and cultures.

\stopchapter
\stoptext