\setvariables[article][shortauthor={Ortega}, date={February 2020}, issue={4}, DOI={Upcoming}]

\setupinteraction[title={From Street Protest to Twitter Protest: A Review of Leonardo Flores's Twitter Bots},author={Élika Ortega}, date={February 2020}, subtitle={Review of Leonardo Flores's Twitter Bots}]
\environment env_journal


\starttext


\startchapter[title={From Street Protest to Twitter Protest: A Review of Leonardo Flores's Twitter Bots}
, marking={Review of Leonardo Flores's Twitter Bots}
, bookmark={From Street Protest to Twitter Protest: A Review of Leonardo Flores's Twitter Bots}]


\startlines
{\bf
Élika Ortega
}
\stoplines


A Twitter bot (short for robot) is a small script designed to publish automatically to the social media site, either on a schedule or in response to other tweets. There are many kinds of Twitter bots with particular purposes and objectives---not all of them benign. Perhaps the kinds of bots that have received the most attention are those designed to spread propaganda or misinformation. The \quotation{\useURL[url1][https://points.datasociety.net/rise-of-the-pe\%C3\%B1abots-d35f9fe12d67][][Peñabots]\from[url1]} in Mexico provide a well-documented example.\footnote{See Luis Daniel, \quotation{Rise of the Peñabots,} {\em Points} (blog), Data and Society Research Institute, 24 February 2016, \useURL[url2][https://points.datasociety.net/rise-of-the-pe\%C3\%B1abots-d35f9fe12d67][][https://points.datasociety.net/rise-of-the-pe\letterpercent{}C3\letterpercent{}B1abots-d35f9fe12d67]\from[url2].} However, other bots use social media platforms to carry out surprising and sometimes amusing literary and artistic experiments. Within this subset, protest bots have become an expression of tactical media to raise awareness about injustices, to contribute to social movements' discourses online, and to critique negative political actions. Mark Sample, a pioneer botmaker, has devoted significant work exploring how Twitter bots can be harnessed for political engagement. He lists \useURL[url3][https://www.samplereality.com/2015/10/03/a-protest-bot-is-a-bot-so-specific-you-cant-mistake-it-for-bullshit/][][five characteristics]\from[url3] of a protest bot: it must be topical and respond to actual events or issues; it must be data based, that is, it has to be based on reality and actual information; protest bots must act cumulatively, such that its repetition emphasizes and insists on the problem it points out; it must be oppositional and take a stance in regard to the issues it addresses; and finally, it must be uncanny, revealing something hidden or unnoticed.\footnote{See Mark Sample, \quotation{A Protest Bot Is a Bot So Specific You Can't Mistake It for Bullshit,} {\em @samplereality} (blog), 3 October 2015, \useURL[url4][https://www.samplereality.com/2015/10/03/a-protest-bot-is-a-bot-so-specific-you-cant-mistake-it-for-bullshit/]\from[url4].}

Leonardo Flores is a US-based Puerto Rican researcher and a maker of Twitter bots. His bots {[}@Protestitas{]}(https://twitter.com/TinyProtests) and {[}@TinyProtests{]}(https://twitter.com/TinyProtests) are twin Twitter protest bots that doubtlessly have drawn inspiration from Sample's work.\footnote{See Protestitas (@Protestitas), \useURL[url5][https://twitter.com/protestitas]\from[url5]; and Tiny Protests (@TinyProtests), \useURL[url6][https://twitter.com/TinyProtests]\from[url6].} One functions in Spanish and is dedicated to supporting resistance movements in Puerto Rico; the other, in English, is focused on amplifying demonstrations and marches taking place in the United States. In both instances, Flores's expertly made bots construct scenes in which protesters, represented by sad, angry, and other humanoid emojis, gather around in town squares, congregate in front of gated government buildings, and march through streets where traffic is blocked and the police are present. In some cases, the tweets represent the forceful response of police using tear gas against weepy-eyed emojis; in others, it is the police who are surrounded by protesters. These sorts of tweets produced by the bots lend themselves to being paired with numerous protest slogans and chants, such as \quotation{¡EN DEFENSA DE LA EDUCACIÓN PÚBLICA!,} \quotation{¡FUERA LA JUNTA!,} \quotation{¡\#UNNUEVOPUERTORICO!,} \quotation{GUN CONTROL NOW!,} and \quotation{THE OCEANS ARE RISING AND SO ARE WE!} The slogans also reveal a very different political and social landscape in each place. On occasion, Flores adds new topical landscapes to address new developments in social movements. In December 2019 @TinyProtests added topical tweets about the impeachment of the US president. Similarly, a very poignant recurrent example in @Protestitas specifically addresses the revelation by a Harvard University study that Hurricane María caused over four thousand deaths---a number that is exponentially higher than the official count. Moreover, \useURL[url7][https://twitter.com/Protestitas/status/1228299199281860608][][this tweet]\from[url7] presents the protest scene in more metaphorical terms by using shoe emojis, instead of faces, to evoke those missing.\footnote{See Protestitas (@Protestitas), \quotation{¡4645 MUERTES! {[}with various emojis, including the US flag, the Puerto Rican flag, and multiple shoes{]},} Twitter, 14 February 2020, 7:45 am, \useURL[url8][https://twitter.com/Protestitas/status/1228299199281860608]\from[url8].}

By making bots that tweet on a schedule---@Protestitas every three hours and @TinyProtest every hour---Flores is capable of participating in the online resistance discourse as it develops through any given day. In addition to amplifying and keeping alive various protests chants, the bots offer a reminder that political resistance is an ongoing, cumulative effort. Further, by using actual protest hashtags, Flores is able to join the online discourse about any given social movement. Ultimately, the bots' appearances through the use of emojis easily makes their tweets stand out as surprising additions to the often-pernicious social media site. From another point of view, Flores's bots participate in a larger ecology of artistic and protest Twitter bots.

More specifically, @Protestitas and @TinyProtests are part of the universe of \quotation{tiny} Twitter bots. Other bots in the tiny universe similarly depict scenes through the use of emojis. A fairly comprehensive list of \quotation{tiny} bots can be found at \useURL[url9][https://botwiki.org/?s=tiny&search-filters-options\%5B\%5D=bots][][Botwiki]\from[url9].\footnote{See \quotation{Search Result(s) for {\em tiny},} {\em Botwiki}, \useURL[url10][https://botwiki.org/?s=tiny&search-filters-options\%5B\%5D=bots][][https://botwiki.org/?s=tiny&search-filters-options\letterpercent{}5B\letterpercent{}5D=bots]\from[url10].} It is important to note that, in general, other bots of the tiny universe do not engage in any kind of political discourse but largely create scenes evocative of places: a gallery, a zoo, a garden, dungeons, homes, isles, bus stops, and so on. This playful context offers Flores a format to build from and an immediate community for his bots. But it raises the question of whether some could consider Flores's bots too whimsical, given the urgent and pressing issues they touch.

On Twitter, the description of the bots is constrained to the profile biography and, therefore, is rather minimal. While one can easily infer Flores's stance from the output of the bots themselves, he has also provided a statement about this work hidden in the source code of the bots' version published in the \useURL[url11][https://taper.badquar.to/2/tiny_protests_protestitas.html][][online literary magazine {\em Taper}]\from[url11]. There, Flores offers a joint version of the two bots, one scrolling up after the other, as demonstrating contingents would appear on a street. Movingly, embedded in the code Flores states, \quotation{In all of Puerto Rico's government offices and buildings, the US and PR flags fly together, a reminder that our realities and politics are intertwined.}\footnote{See Leonardo Flores, \quotation{Tiny Protests/Protestitas,} {\em Taper}, no. 2 (fall 2018), \useURL[url12][https://taper.badquar.to/2/tiny_protests_protestitas.html]\from[url12]. To read the code and Flores's statement, use the \quotation{View Page Source}option on Firefox or Chrome.} By bringing the two pieces together, Flores underscores the two nations'uneven connectedness that burdens Puerto Rico with no representation in Congress, the inability to vote in the US presidential elections, and the denial of self-determination under US rule. In a way, the joint version of the bots suggest that Flores wishes for an allyship between the two locales---to make the interconnection about the shared struggle for social justice and against the politics of exploitation of the island.

Allison Parrish, a prolific Twitter poet, has likened Twitter bots to graffiti, in as much as they are \quotation{\useURL[url13][https://points.datasociety.net/bots-a-definition-and-some-historical-threads-47738c8ab1ce][][interventions in a public space]\from[url13] (to the extent that, e.g., Twitter can be defined as a public space).}\footnote{Allison Parrish, \quotation{Bots: A Definition and Some Historical Threads,} {\em Points} (blog), Data and Society Research Institute, 24 February 2016, \useURL[url14][https://points.datasociety.net/bots-a-definition-and-some-historical-threads-47738c8ab1ce]\from[url14].} It is easy to see how @Protestitas and @TinyProtests perfectly match this description because, despite being constrained to the online space, Flores's bots take a clear stand and candidly engage with the protesters and demonstrations happening in the streets. The playful and whimsical output of these bots also parallels the use of puns and humorous slang in protest signs. They show how creativity, even on Twitter and even through the use of emojis, can further and strengthen the reach and impact of social movements.

\thinrule

\page
\subsection{Élika Ortega}

Élika Ortega is an assistant professor in the Department of Spanish and Portuguese at the University of Colorado Boulder. Her research focuses on digital literature and media, cultural hybridity, reading practices and interfaces, books, and digital humanities. She is currently writing her first monograph, tentatively titled \quotation{Binding Media: Print-Digital Literature, 1980s--2010s,} in which she investigates print-digital works of literature from Argentina to Canada.

\stopchapter
\stoptext