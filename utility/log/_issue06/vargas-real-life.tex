\setvariables[article][shortauthor={Vargas Zuleta}, date={May 2022}, issue={6}, DOI={https://doi.org/10.7916/archipelagos-7w2x-a973}]

\setupinteraction[title={Real-life and Fictional Chronologies: on building a set of timelines around René Depestre's *Hadriana dans tous mes rêves*},author={Laura Vargas Zuleta}, date={May 2022}, subtitle={Real-life and Fictional Chronologies}, state=start, color=black, style=\tf]
\environment env_journal


\starttext


\startchapter[title={Real-life and Fictional Chronologies: on building a set of timelines around René Depestre's {\em Hadriana dans tous mes rêves}}
, marking={Real-life and Fictional Chronologies}
, bookmark={Real-life and Fictional Chronologies: on building a set of timelines around René Depestre's *Hadriana dans tous mes rêves*}]


\startlines
{\bf
Laura Vargas Zuleta
}
\stoplines


{\startnarrower\it In this essay I explore the connections between literature and historic documents through a series of digital timelines that I created around the novel {\em Hadriana dans tous mes rêves} (1988), by Haitian writer René Depestre. As part of the~In All My Dreams~book club and art exposition organized by Professors Kaiama L. Glover and Laurent Dubois around {\em Hadriana}, my digital project proposes an interactive way for readers to get a sense of the novel by contrasting the events in the book with historical ones. Focusing on the years the novel covers and those when Depestre wrote it, I used the Digital Library of the Caribbean (dLOC) archive of the Haitian newspaper~{\em Le Nouvelliste}~and selected a group of news items about Jacmel (the city where the story is set) and Depestre himself to include in my project. I selected the news items looking for content that could help readers imagine the setting of {\em Hadriana}. The novel, whose second movement includes an actual newspaper article about Jacmel, is narrated in a fragmented, nonlinear way, but it mentions the dates of major events. However, that nonlinearity made me wonder how to build a timeline that enriches reading of the novel instead of spoiling it. The final product consists of three timelines: the first focuses on the novel; the second includes photos of the news items I selected from~{\em Le Nouvelliste} and links to the original archive but has no information from the book; and the third combines the first two so readers can compare and contrast both realities. This project shows the possibilities that an archive such as dLOC offers for the reading of literary works such as Depestre's.

 \stopnarrower}

\blank[2*line]
\blackrule[width=\textwidth,height=.01pt]
\blank[2*line]

What sort of digital tools can help nonacademic readers navigate a novel whose temporal structure is fragmented and nonlinear? How can we build such tools so they function as an invitation or provocation instead of as a rigid, paternalistic guide in which information is provided to be absorbed instead of explored? My digital project, \quotation{Hadriana in Context: Timelines,} was guided and shaped by these two questions that, to me, were key to creating something useful and enriching for readers. As part of the second version of the In All My Dreams book club around Haitian writer René Depestre's novel {\em Hadriana dans tous me rêves} ({\em Hadriana in All My Dreams}, 1988) organized by Dr.~Kaiama L. Glover and Dr.~Laurent Dubois, I created a series of interactive digital timelines that combined the events of the novel with real-life news articles covering the place (Jacmel, a coastal city in southern Haiti) and time (mainly the 1930s and 1940s) in which the novel is set, using contemporary issues of the Haitian newspaper {\em Le Nouvelliste,} taken from the archive of the Digital Library of the Caribbean (dLOC). This digital project, which was included on the book club's website as part of its \quotation{Reader's Guide,} proposes an interactive way for readers to get a sense of Depestre's novel by contrasting its events with historical ones, as well as by guiding them to real-life newspaper archives from Haiti.\footnote{NOLA 2021, \quotation{Reader's Guide: A \quote{Map} of Hadriana,} accessed 18 January 2022, \useURL[url1][https://web.archive.org/web/20210520185531/https:/nola2021.iamdbookclub.com/a-map-of-hadriana-and-cast-of-characters/][][<https://web.archive.org>]\from[url1].} By introducing elements from a source outside the book, the timelines aim to enrich readers' experience of {\em Hadriana}, not only by situating the novel in its context but also by presenting it as part of Haitian history. My project also seeks to give visibility to dLOC as a useful database for exploring the context of Caribbean literary works.

Depestre wrote {\em Hadriana dans tous mes rêves} in France, where he had been living for many years. The novel tells the story of Hadriana, a young white French woman living in Jacmel who is turned into a zombie on the day of her wedding. The story is divided into three movements, as Depestre calls them, that focus on different moments: the first is narrated by Patrick, Hadriana's godbrother, and tells the events of her wedding, apparent death, and disappearance from her grave; the second is also mainly told by Patrick but takes place thirty-nine years later, when he is working as a university professor in Kingston, Jamaica, and includes different textual genres such as notes for a book, an imagined interview, and a newspaper article about Jacmel; the third movement takes us back to Jacmel to the day of the wedding. This movement is narrated by the protagonist herself, Hadriana Siloé, and corrects Patrick's version of the events. The presence of the newspaper article in the second movement inspired Professor Dubois and me to explore the archive of {\em Le Nouvelliste} and conceptualize a project that combined both sources, Depestre's novel and the newspaper issues.

{\em Hadriana}'s nonlinear structure, whose fragmentation is especially evident in the second movement, is a challenge for readers to follow and forces them to continually go back a few pages to rectify information or try to put the pieces together. For this reason, the website of the In All My Dreams book club includes a \quotation{Reader's Guide} with different tools (a list of characters, a map of the novel's structure) to give readers a better idea of how the novel is structured, who is who, and what is happening in each movement. The novel's nonlinearity made me ask myself what sort of project to build around it to contribute to the guide without spoiling the reading process by overexplaining. This central concern led to the creation of \quotation{Hadriana in Context: Timelines.}

Interactive 1. Hadriana in Context: Timelines

When I began the research, I did not know what the final product would be. However, Professor Dubois, with whom I was working as a research assistant, told me the organizers of the book club were trying to attract more nonacademic participants for its second session, so he suggested I create something with nonacademic readers in mind. The idea was to offer participants materials that were not exclusively from the novel so they could explore beyond the text and be able to connect literature with elements of daily life, such as newspaper articles. The materials did not have to be directly related to the novel; I was not expected to find articles only on Depestre and {\em Hadriana} but rather to unearth stories that could offer readers context on Jacmel's day-to-day life as a way of seeing what Depestre might have had in mind when he chose the space and time for Hadriana's story.

Given Depestre's inclusion of an actual 1972 newspaper article on Jacmel as part of the novel's second movement, and our choice to explore {\em Le Nouvelliste}'s archive, I decided to start with the years 1945 and 1946, when the author began his literary and political career and began to gain renown in Haiti. For this first approach, I relied heavily on dLOC's search tool, which allowed me to look for specific words in the four or six pages that constituted each newspaper issue. I looked up four main words: {\em Depestre}, {\em Ruche} ({\em La Ruche} was Depestre's Marxist newspaper), {\em Carnaval} (Hadriana's wedding, death, and funeral take place in the middle of Carnival), and {\em Jacmel}. I found many news items about Jacmel's infrastructure being damaged by heavy rain or about politicians' visits to the city, but not a single word about {\em La Ruche} or Jacmel's Carnival, and only two items on René Depestre: one from August 1946 titled \quotation{René Depestre malade {[}René Depestre Ill{]},} which wished the writer a speedy recovery, and another from October of the same year, \quotation{Départ de M. René Depestre {[}Departure of M. René Depestre{]},} announcing that he was leaving to study in France.

After reviewing all the {\em Nouvelliste} issues available for 1946, we decided it was time to go back a few years and explore from 1937, the year of Hadriana's wedding and all the main events in the novel's first movement. With the change of decade came a change in search keywords. Since Depestre would not have been in the news back then, I stopped searching for his name or {\em La Ruche}'s but kept looking for news items, columns, and articles about Jacmel. Because of the importance of Jacmel's natural environment at the end of {\em Hadriana}'s third movement, I also added more keywords to my search to see what sort of information I could find on environmental issues, words like {\em arbre} (tree), {\em reboisement} (reforestation), {\em déboisement} (deforestation), {\em plantes} (plants), {\em planter} (to plant), and so on. However, not all issues of {\em Le Nouvelliste} from the 1930s had the word search tool I had been using before, so in many cases I had to go through all the pages looking for clues of what I was searching for. This meant I had to devote more time per issue, but it also led me to a discovery we had not anticipated: a reforestation program supported by the state.

The Haitian president of the time, Sténio Vincent, started the program in 1937. It included brigades to plant trees in deforested areas, as well as fines for those who cut trees without planting new ones to replace them. Most 1937 issues of {\em Le Nouvelliste} include an ad consisting of a short text detailing the importance of plants, specifically of reforestation, for agriculture and the Haitian economy. These announcements notified readers of who would be participating in the program and reminded them of the importance of this issue for their security and that of their children. Although Depestre's novel does not delve into the issue of deforestation, natural environments, especially around Jacmel, are a key element in the text. This is particularly evident at the beginning of the first movement when Patrick is describing the area of Jacmel, and at the end of the third movement when Hadriana recounts how she fled her captors and left Haiti.\footnote{René Depestre,~{\em Hadriana dans tous mes rêves}~(Paris: Gallimard, 1988), 189; hereafter cited parenthetically in the text.} In the latter passage, Hadriana presents nature as a sort of ally that protects her better than the residents of Jacmel, so I continued searching for articles on Vincent's program.

Besides the advertisements, the newspaper also published articles about the program, like \quotation{S.E. Sténio et le problème du reboisement {[}H.E. (His Excellency) Sténio and the Problem of Reforestation{]},}\footnote{\quotation{S. E. Sténio et le problème du reboisement,} {\em Le Nouvelliste,} 17 July 1937, \useURL[url2][https://www.dloc.com/UF00000081/03078?l=en][][<https://www.dloc.com/>]\from[url2].} found on the first page of the 17 July 1937 issue, in which the newspaper congratulates the president on the amazing job being done thanks to the program; \quotation{Le reboisement {[}Reforestation{]},}\footnote{\quotation{Le Reboisement,} {\em Le Nouvelliste}, 26 July 1937, \useURL[url3][https://www.dloc.com/UF00000081/03085?l=en][][<https://www.dloc.com/>]\from[url3].} found on the first page of the 26 July 1937 issue, in which the newspaper gives some more information about how the program works and mentions Jacmel at the end as one of the regions where the program is at work; \quotation{Contraventions au décret sur le reboisement {[}Violations of the Reforestation Decree{]},}\footnote{\quotation{Propos et faits divers: Contraventions au décret sur le reboisement,} {\em Le Nouvelliste}, 30 April 1938, \useURL[url4][https://www.dloc.com/UF00000081/03324?l=en][][<https://www.dloc.com/>]\from[url4].} a news item from the 30 April 1938 issue reporting that ninety-one individuals were tried for deforestation and sixty of them convicted; or \quotation{Reboisez vos terres {[}Reforest your land{]},}\footnote{\quotation{Reboisez vos terres,} {\em Le Nouvelliste}, 26 June 1938, \useURL[url5][https://www.dloc.com/UF00000081/03060?l=en][][<https://www.dloc.com/>]\from[url5].} found on the first page of the 26 June 1938 issue, about the ceremony inaugurating the reforestation of mountains in Kenscoff. News of the reforestation program stops after 1938, so articles on this initiative only cover two years, but during 1937 and 1938 it is present in most issues of the newspaper, from news items to longer ones explaining to readers the new policies regarding the planting and cutting of trees in Haiti.

Apart from Vincent's reforestation campaign, I also found interesting news items on Jacmel from the years of 1937 and 1938 and even through 1941. The nature of the news was extremely diverse, from reporting on a general blackout\footnote{\quotation{Propos et faits divers: Jacmel dans l'obscurité,} {\em Le Nouvelliste}, 28 May 1938, \useURL[url6][https://www.dloc.com/UF00000081/03348?l=en][][<https://www.dloc.com/>]\from[url6].} in the commune to the mysterious disappearance of the Jacmel orchestra's instruments.\footnote{\quotation{Propos et fait divers: Jacmel réclame son orchestre,} {\em Le Nouvelliste}, 12 February 1938, \useURL[url7][https://www.dloc.com/UF00000081/03258?l=en][][<https://www.dloc.com/>]\from[url7].} One of the longest pieces I found in the archive was a column published in several September 1938 issues titled \quotation{Voyage autour de Jacmel {[}Trip around Jacmel{]}} and signed by \quotation{J.B.} (figure 1).\footnote{For the first part of this column, see \quotation{Voyage autour de Jacmel,} {\em Le Nouvelliste}, 16 September 1938, \useURL[url8][https://www.dloc.com/UF00000081/03443?l=en][][<https://www.dloc.com/>]\from[url8].} This column is reminiscent of the \quotation{Lettre de Jacmel} in Depestre's novel (figure 2), a series Patrick tells us was published by the Paris newspaper {\em Le Monde} in April 1972 (113--17).

\placefigure[here]{Voyage autour de Jacmel}{\externalfigure[issue06/valence_1_lv_jacmel-nouvelliste.jpg]}


\placefigure[here]{Lettre de Jacmel}{\externalfigure[issue06/valence_1_lv_jacmel-monde.jpg]}


I recorded all my findings in a series of Excel documents. Here I included the link from dLOC where I had found them, the issue number and the date when it was published, the page, the section (when necessary), the title, and a brief summary of what it said. Within the documents, I created different categories: Jacmel, Depestre, deforestation, and so on. I also made separate documents for each six-month period I covered in my research so the tables would be easier to manage: especially for the \quotation{Jacmel} category, I was finding more materials per year than could fit easily in a single document. At that time, I was focused not on finding articles or items directly related to events or themes of the novel but on creating a database of all materials that matched my keywords or topics, in order to compile a vast list from which to choose articles at a later point.

Having covered most of {\em Le Nouvelliste} issues from 1937 to 1946, I had enough material about Jacmel, Depestre, and life in Haiti to decide what kind of project was best for the \quotation{Reader's Guide.} We discussed the idea of creating an inviting digital object that would direct people to the dLOC database that I had been using, so they could find more intriguing content and materials. After giving it some thought, I came up with the idea of a timeline that would allow readers to contrast events in the book with news items from {\em Le Nouvelliste}. It occurred to me that the guide on the website, which at that point consisted mainly of text, needed a visual and interactive tool that people could use at their own pace, and in the order they chose, without adding much more reading to the equation.

Another advantage of the combined timeline was that readers could return to it before, during, and even after reading {\em Hadriana}; because it contained information beyond Depestre's novel, the timeline would not become obsolete after reading concluded{\em .} I also thought it would be an engaging and clear way of visualizing the story in its historical context, and thus encourage users to interact with a literary text as a complex cultural object that could be put in dialogue with other sorts of sources. However, because one of the novel's structural features is its nonlinearity, I had to make visible that structure, that way of conceiving of time, without erasing the fragmentary element. If the novel was refusing linearity in its presentation of events, would a timeline not achieve the exact opposite? It was clear to me that I needed to design my tool so that linearity would not act as a straitjacket, an obligatory path that all users had to walk. Moreover, paradoxically, I needed my project to invite readers to keep transgressing linear time even though I was offering them a timeline.

The objective of the project was not to educate readers on the main events in Haitian history (for which they could find multiple tools on the internet) but to provide a window on in Jacmel life that would enable non-Haitian readers to grasp the story's context. With this goal in mind, I had to choose from the materials I had collected in my research those that the book club participants would find interesting and even fun. Again, I was aiming to inspire users to explore more of what dLOC had to offer to complement their experience of Depestre's novel, not to lecture them on its context and facts. I hoped for my project to be one that curated unique items, things that could not be found in a quick Wikipedia or Google search, thus providing users with a new window through which to look at the text.

Taking this into account, I went back to the documents where I had recorded my findings and made a list of candidates for the timelines: news on missing objects, machete fights, cultural happenings, or the agriculture of the region, as well as some news items on the reforestation program and René Depestre himself. The first list of potential objects consisted of forty items, most of them news items and articles of which I had taken screenshots because I found them particularly interesting for the book club even before I knew I would use them for a timeline. Forty was a big number, and I thought it could make the timeline overwhelming for its users and spoil its purpose of inviting readers to explore the original archive, so I knew I had to make my list shorter. However, since I was not sure how many events from Depestre's novel I was going to be able to locate chronologically, I could not decide how many candidates to eliminate.

The next step was to reread the novel and take note of all the events that included a specific date or for which Depestre had mentioned at least the month in which an event took place. I was surprised to discover that the author had been fairly clear on dates for most of the novel's main events, and a detailed timeline could be made without difficulty. At the end I had fifteen dates from the novel, with only one, January 1918, going further in the past than my own findings from {\em Le Nouvelliste.} All of the events after 1946, the last year I included in my research, were set not in Haiti but in Europe and Jamaica, so it seemed that both chronologies fit together for the project I had in mind, to the point that Patrick's and Depestre's arrivals in Paris from Haiti happened in 1946 with only a few months' difference (figure 3).

\placefigure[here]{Patrick and Depestre arrive in Paris in 1946}{\externalfigure[issue06/valence_1_lv_patrick-paris.jpg]}


It was telling that even though the dates from the novel covered most of the twentieth century (1918--77), the biggest concentration of events was in the late 1930s, reminding us that Patrick's chronology was selective, that he and I both were creating a narrative through the events he chose. In his narrative he was building not the history of Jacmel but the history of Jacmel through Hadriana. I say \quotation{Patrick's chronology} because Hadriana's narrative does not include precise dates the way Patrick's does. She seems more focused on getting the facts right, on stating her position and correcting what has been said before {\em about her} and about {\em her experience} of all that happened {\em to her}. The only date she mentions is the one of her arrival in Jamaica, in February 1938, possibly because this is the date she started her new life, away from the expectations of the Jacmel community.

In the first two movements, Patrick builds a narrative that one could easily be tricked into reading as objective and rigorous, especially since he constantly contests other versions (32) and seems to study (takes notes, reads about zombification, includes other sources) the case of Jacmel and Hadriana. His narrative creates the illusion that he is somehow destined to know more than others, to tell Hadriana's story and, thereby, to understand the events that led to her disappearance. However, Hadriana's later retelling of the story forces readers to look back at Patrick's narrative with a fresh perspective and see how subjective it really was. I did not want to miss this in the final product of my project, even though the dates I was including were mainly sourced from Patrick's narrative. I started looking for ways to play with my own narrative, my own chronology, so users were also pushed to go back and look anew at the same items they saw at the beginning of each timeline.

After rereading {\em Hadriana} and visualizing a clear chronology of the novel, I came back to my list of candidate items from {\em Le Nouvelliste} and discarded eight that seemed less helpful for the project either because they did not include much information or because they were repetitive (meaning I had other similar items from other moments). This was easy to do because my goal was not to present an exhaustive list of each time Jacmel or Depestre was in the newspaper but rather to enrich other people's reading of {\em Hadriana} and to offer a new space to explore its context.

We reached out to Dr.~Eliza Bourque Dandridge, who guided us in selecting the best tool for the timeline. Dr.~Dandridge suggested that I use Northwestern University's Knight Lab since it was easy to manage: the lab offers a template in a Excel document that one can fill with the information one wants in the digital timeline and include images, change colors and fonts, and so on.\footnote{Northwestern University, Knight Lab, home page, accessed 18 January 2022, https://web.archive.org/web/20210519012906/https:/knightlab.northwestern.edu/.} Embedding the final product on a website was very simple. One of the reasons why I chose Knight Lab was that it allowed users to interact with the timelines so they could decide how to navigate the information. Dr.~Dandridge and I also discussed my ideas on how to proceed. I was worried the timeline would include too much key information for those who were about to read or still reading the novel; in other words, I did not want any \quotation{spoilers,} which would have defeated the purpose of the project. Given that Depestre's novel is not linear in its chronology, it was impossible for readers to control the tool in a way that only gave them events from the pages they had read, so my whole set of dates constituted a field plagued with spoilers. We agreed that we needed to offer a set of three timelines: one exclusively for the materials from {\em Le Nouvelliste} for those interested in exploring the historical context without spoilers, one exclusively for {\em Hadriana}, and one that combined the two. Both the {\em Hadriana} timeline and the combined timeline would have a spoiler alert at the beginning.

Because it was the shorter and simpler one, I decided to start with the {\em Hadriana} timeline while I familiarized myself with the tool I was using. It took a few tries, but when I managed to use it right it was exactly what I had in mind: the timeline allowed users to navigate the chronologies at their pace and in the order they preferred: in the center of the screen it showed one of the events with the information and images users had chosen for it; two arrows, one on each side, would allow users to go back and forth; and, at the bottom, users could skip to random events by clicking on them on a small timeline. The fact that the timelines could be navigated in many ways and not exclusively in chronological order fit perfectly with Depestre's nonlinear structure and seemed more inviting for users, a better way to make them feel engaged with the materials. This also helped my other objective: if users could explore events in different orders, this would encourage looking at them from different perspectives to see how they built different sorts of narratives depending on how the timelines were used. There was only one thing that bothered me: my {\em Hadriana} timeline needed a visual element: I only had text for it and none of the Knight Lab's editing options for font and color seemed to fill that lack.

While I thought of a solution for that issue, I begun to work on the {\em Nouvelliste} timeline with its thirty-two items. For this one I did have visual materials for each item since I had taken screenshots directly from dLOC's archive. Even though all materials from the newspaper were in French, it was easy to add a brief English summary of the content of each item in the description. The entries for each date, however, remained in French because I decided to use the original titles from {\em Le Nouvelliste} as a way to break with English a little. I must acknowledge here the absence of materials in Kreyòl in the project, which is problematic given that it is the only language spoken by 100 percent of Haitians, while French is only spoken by a small minority, usually members of the elite. Even if Depestre himself wrote in French and the book club was held in English, I am including materials in Kreyòl and inviting readers to explore the language of the Haitian people would make the project stronger. I hope to do this with the timelines and enrich them for future versions of the book club.

After a few weeks, I had both separated timelines but still needed to find some visual materials for the {\em Hadriana} version and then merge it with the {\em Nouvelliste} one for the third combined timeline. The problem with the visual factor of the {\em Hadriana} timeline was that I needed fifteen images that I could legally use and that looked good together. Depestre's novel does not include any illustrations, but the text is written in a way that seems very visual, so I was convinced I could not present the timeline as text only. After experimenting with different ideas, like looking for images of different editions of the novel or selecting random stock images that were related to each event, I went back to my notes for inspiration (figure 4). All the notes I had taken for the project, with the exception of the Excel documents, were handwritten using an iPad and a digital notebook app called GoodNotes, in which I could write as if on paper. Looking at my notes I realized that I had the visual element for the timeline right in front of me, so I began exploring ways to incorporate screenshots from it into the timeline. With this last touch, the {\em Hadriana} timeline was finally complete.

\placefigure[here]{{\em Hadriana}'s Timeline on the Reader's Guide}{\externalfigure[issue06/valence_1_lv_timeline-hadriana.jpg]}


The last thing I had to do was merge the two timelines to create the combined one. I first did this without altering the original format, but when I finished it became clear that some differentiation was needed between what belonged to {\em Le} {\em Nouvelliste}, or the real-life events, and what came from the novel, or the fictional events. It took me a week to resolve this problem since I wanted something that worked with the screenshots of both sources and that was not distracting for users. I experimented with different background colors and ended up with the simplest solution: all items from the novel would appear with a dark gray background that highlighted the colors of the images from my notes (cream yellow, blue, black, and red), and all items from the newspaper would have a white background so the black and white screenshots did not look too dark. All these formatting elements, the images, the background colors, helped a very long timeline (forty-seven entries or events in total) to look not overwhelming or monotonous but rich, stimulating, and easy to navigate. In this third timeline, I also included, as I did with the {\em Nouvelliste} one, all the relevant reference information for each item, including a link to dLOC's archive, so it would, again, encourage the readers to go to the source and explore further. Another way of doing this was that, for those items too long to fit in a single screenshot, I only included a portion, so if someone wanted to read it all they would have to go to the archive.

When the time came for the second version of the book club in February 2021, the timelines were already embedded in the \quotation{Reader's Guide} of the website. I presented them to the group, explaining briefly how to navigate them and where not to go in order to avoid spoilers. Many participants noted the complexity that {\em Hadriana}'s nonlinear structure represented for them. The guide we provided, including the timelines, thus helped them to read the novel without feeling too frustrated or confused. The reception of the timelines in the book club was good, and we referenced them in some sessions while discussing the novel, but one of the best results of the project was that they also are available for future readers who browse the internet in search of information about {\em Hadriana}. Even though I was thinking of a nonacademic audience when I designed the timelines, I believe they also constitute a useful tool for scholars revisiting the novel or searching for contextual information to study {\em Hadriana}, since the timelines allow people to review the story's main events and to access a rich database to approach literary texts from the Caribbean: dLOC.

Looking back at the project, I would add or do a few elements differently now. I have already mentioned the absence of materials in Kreyòl in my project, but I must also acknowledge that the project would benefit from a more diverse use of sources. As a non-Haitian scholar, I failed to see that even though {\em Le Nouvelliste} provided enough material in terms of the amount of items I was able to include in the timelines and the broad time frame they covered, it is still a newspaper whose main audience is the Haitian elite, that minority who can read French. Just as Patrick was telling the story of Jacmel {\em through} what he knew (or thought he knew) about Hadriana, I was presenting a narrative of the city through the lens of {\em Le Nouvelliste,} in both cases the narratives missed other perspectives. This could easily be solved by engaging in conversations with Haitian scholars such as Jean-Ellie Giles, whose work on the city of Jacmel and its influence in Haitian history could dialogue especially well with the purposes of my project.

Users of the timelines could also be helped if I included sources that present Jacmel through visual materials. Once again, dLOC proves to be useful for this sort of project, since the digital library includes a series of photos of the city from the 1930s,\footnote{Digital Library of the Caribbean, search results for \quotation{jacmel" anywhere and}Direction générale des Travaux Publics” as publisher, accessed 18 January 2022, \useURL[url9][https://dloc.com/results/brief/?t=jacmel,\%22Direction+generale+des+travaux+publics\%22&f=ZZ,PU][][<https://www.dloc.com/>]\from[url9].} the decade of Hadriana's wedding, which could add an extra layer to the timelines that would not depend on language. This is something I look forward to doing in the future, perhaps for a third version of the book club or just for the website where the project can be found.

That said, the timelines proved to be an interesting feature for the book club and allowed readers to engage with two crucial aspects of the novel: chronology and context. Even if I played with the format to separate items from both sources, the novel and the newspaper, the third timeline transgressed the limits between the real and the fictional, history and literature, inviting users to engage with both on the same plane. By inserting the fictional events into the chronology of real-life news and facts, users could construct their imaginary around Jacmel and Haiti in a form of bricolage echoing Depestre's style. For example, Depestre's own gesture from {\em Hadriana}'s second movement involves playing with genres and including a newspaper column on Jacmel, an imagined interview, and an anthropological study on zombification and the zombie as a myth as part of Patrick's narration of his struggle to survive his godsister's disappearance. This also resonated with Depestre's retelling of the story through Hadriana's voice in the third movement. Each of my timelines was a narrative with multiple navigation options, even though the items did not change from one to the other. By mixing both realities and choosing how to present each event, I was offering a new way of telling the stories they encapsulated. For a three-movement novel, I made three timelines, three attempts to locate the novel's chronology and context so readers could approach it from different angles. I hoped the timelines would convey the novel's richness and multiple possibilities and help readers build their own narratives of what they were reading. From this, I learned that these sorts of projects are also new ways of telling stories and that, maybe, this is what one should aim for in creating reading guides, no matter the audience, in order to inspire readers to explore literary materials.

\thinrule

\page
\subsection{Laura Vargas Zuleta}

Laura Vargas Zuleta is a PhD student in the Romance Studies Department at Duke University, where she is pursuing a Romance Track program with a focus on contemporary hispanophone and francophone Caribbean literature. She holds a BA in literature from the Universidad de los Andes in Bogotá and an MA in Hispanic studies from Boston College.

\stopchapter
\stoptext