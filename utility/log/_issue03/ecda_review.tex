\setvariables[article][shortauthor={Dillon, Aljoe}, date={July 9 2019}, issue={3}, DOI={10.7916/archipelagos-012j-wq44}]

\setupinteraction[title={Early Caribbean Digital Archive},author={Elizabeth Maddock Dillon, Nicole N. Aljoe}, date={July 9 2019}, subtitle={EACD}, state=start, color=black, style=\tf]
\environment env_journal


\starttext


\startchapter[title={Early Caribbean Digital Archive}
, marking={EACD}
, bookmark={Early Caribbean Digital Archive}]


\startlines
{\bf
Elizabeth Maddock Dillon
Nicole N. Aljoe
}
\stoplines


\subsection[title={{\em sx archipelagos} review},reference={sx-archipelagos-review}]

The {\em \useURL[url1][https://ecda.northeastern.edu/][][Early Caribbean Digital Archive]\from[url1]} is a compelling platform that engages with the pitfalls and possibilities of uncovering alternative narratives within the constrained space of the colonial historical record. Clearly structured~and aesthetically appealing, the site explicitly and implicitly raises and grapples with crucial~questions regarding methodology and scholarly responsibility.

Perhaps the most important question the {\em Early Caribbean Digital Archive} brings to the fore is whether---or to what extent---it is in fact possible to \quotation{decolonize the colonial archive} via digital practices of \quotation{remixing.} The site authors acknowledge at the outset that the materials they feature \quotation{are primarily authored and published by Europeans,} but they quickly promise to push against that coloniality by foregrounding the narratives and experiences of nonwhite peoples in the early Americas.\footnote{\quotation{The Early Caribbean Digital Archive,} https://ecda.northeastern.edu/.}~The project's stated goals are to go beyond digitization and cataloging to provide both understanding of the mechanisms of coloniality and opportunities for users to revise the archive so to tell different, lesser-known stories.

Before getting to the matter of whether the site accomplishes these objectives, it would be of interest for the site authors to reflect on a number of grounding methodological questions, perhaps within the context of the site's \quotation{\useURL[url2][https://ecda.northeastern.edu/home/about/][][About]\from[url2]} section. Notably, how do we think about those sources we know are curated (selected, abridged, otherwise manipulated) by those in power---sources whose very presence in the archive reflects choices made by colonial subjects? Even more provocatively, what precisely is \quotation{decolonizing} and (why) do we want to \quotation{decolonize the archive}---might not our goal be instead to replace the archive with anticolonial and antislavery methods? Why, also, as per Marisa Fuentes's citation on the site's landing page, should the goal of such projects be to present an \quotation{unbiased account?} Why should \quotation{coherency} be the goal? Might not the goal of history be to acknowledge our biases---or even to have \quotation{the right} biases (it might be argued, even, that the attempt to be neutral is to operate from the position of the colonizer---or what Mary Louise Pratt has called the \quotation{monarch-of-all-I-survey} position)?\footnote{Mary Louise Pratt, {\em Imperial Eyes: Travel Writing and Transculturation} (New York: Routledge, 1992), 201.}~Such broad questions are worth posing.

Also important,~if \quotation{troubling} to consider, is the matter of how to~determine whether the \quotation{embedded narratives} might in fact have been mere literary devices. In other words, given that the invented interlocutor is a major trope of the Enlightenment, can we accept at face value the idea that these enslaved narrators were real and not characters invented to illustrate something the colonial author wanted to prove? Foregrounding such opacities would do the work of responsibly framing the project---acknowledging not just its potentialities and affordances but also its constraints and open queries.

Given their compelling claim that digitization makes a difference, it would be of great value for the site authors to do the work of extraction for a greater number of the embedded slave narratives discoverable among the documents included in their archive. Clearly/prominently identifying/naming enslaved narrators as the defining offering of the platform would go a long way to doing the work of \quotation{decolonizing.} Indeed, this is an extremely intriguing and very promising dimension of the site's contributions.

The \quotation{About} section under the \quotation{Archive} tab might more accurately be labeled \quotation{Using the ECDA} or \quotation{Suggestion for Use,} or something along those lines.

The \quotation{Classroom} tab is fantastic. How, though, do the site authors determine whether materials uploaded via the \quotation{Submit Materials} button at the bottom of the page ultimately end up on the site? Are these materials vetted by a committee and, if so, what kinds of criteria are used? Clarity around these questions would be useful to include on the site.

On the whole, {\em The Early Caribbean Digital Archive} offers a useful and wide-ranging portal into significant materials from the plantation Americas and, as important, proposes compelling paths for querying and pushing against the constraints of the colonial record.

\subsection[title={Response from the creators of {\em The Early Caribbean Digital Archive}},reference={response-from-the-creators-of-the-early-caribbean-digital-archive}]

Nicole Aljoe {\em and} Elizabeth Maddock Dillon

The \useURL[url3][https://ecda.northeastern.edu/][][{\em Early Caribbean Digital Archive}]\from[url3] is a collaborative platform that aims to engage its users in thinking about the coloniality of the archive of the early Caribbean and possibilities for decolonizing this archive. It is an archive that foregrounds the constructed nature of any archive and that---we hope---encourages user engagement in the process of knowledge curating and knowledge making, including revision and expansion of the materials on our site. In this respect, it remains a work-in-progress. For that reason, we are particularly grateful for the thoughtful review of the {\em ECDA} provided by {\em sx archipelagos}.

Above all, the {\em ECDA} aims to use the digital nature of our archive to challenge existing modes of using and creating archives of colonialism. One of the key tenets that we have arrived at, as a team, is a commitment to a notion of the archive as a changing and changeable object---a notion of the archive as more {\em event} than object, as more an {\em encounter} than a knowledge base. And for that reason the biggest challenge (and excitement) of designing and developing the {\em ECDA} has been trying to use the affordances of a digital platform to shift how makers and users of archives might get their hands into the archive. How do we make it possible for users (and not just makers) to \quotation{stir the archive} and thus to perturb the knowledge structures that have made the enslaved and indigenous people of the early Caribbean so invisible and inaccessible to our existing accounts of history?\footnote{On \quotation{stirring the archive,} see Lauren Klein, \quotation{The Carework and Codework of the Digital Humanities,} \hyphenatedurl{http://lklein.com/2015/06/the-carework-and-codework-of-the-digital-humanities/.}}

We are pleased that the review of the {\em ECDA} notes two areas of success in this effort: our collection of embedded slave narratives and our classroom materials. By extracting embedded narratives of enslaved people from the texts of European colonial writers and placing them in the archive as discrete items with their own metadata, we aim to \quotation{remix} the archive and give pride of authorial place to new voices, as well as to challenge some of the metadata structures that have been used to shore up the coloniality of knowledge and power. The pages under the \quotation{Classroom} tab, which include guidelines about how one might use the embedded slave narratives in a class, work in tandem with the embedded slave narrative collection to invite people to explore what might be learned (or lost) from such processes of remix. In newer materials on the classroom pages, we also include sample assignments (and documentation) that invite students and users to identify additional embedded slave narratives for inclusion in the {\em ECDA} collection. We could not agree more with the review that our current collection is slim---indeed, it only begins to scratch the surface of what is possible. But in pioneering a method for building such a collection, we hope that we have opened the door for both our team and users of the archive to continue to expand the collection in exciting new directions.

As we write this, the {\em ECDA} team of graduate students and faculty is preparing a series of workshops in collaboration with the National Archive of Barbados and the University of West Indies, Cave Hill.\footnote{Warm thanks to our partners in Barbados, Amalia S. Levi, Tara Inniss, and Ingrid Thompson, for making this possible.} At these workshops, we will work with our partners in Barbados to invite university students, heritage professionals, and members of the public to transcribe runaway-slave advertisements from the recently digitized {\em Barbados Mercury} and to begin conducting further research into the narratives of the enslaved people whose lives are glimpsed---fugitively---in these ads. We are not entirely sure what will come of these workshops in terms of materials on our site. We do expect to add a section on runaway-slave ads to our collection of embedded slave narratives, but we also expect that collaborating with partners in Barbados may lead us in new directions that we have not yet considered. The collaboration may change the way we have been thinking about knowledge structures, histories, narratives, metadata, digital representation, digital access, and digital infrastructure; we are prepared and excited for that possibility.

The review of the site usefully points out places of incompletion (such as a \quotation{Representing Slavery} link---we are working on it, we promise) and places where greater clarity could be offered to users. We have already made some of the changes suggested. But we are also committed to a few choices that the review questions. Overall, we see the {\em ECDA} as a site of engagement rather than as an argument on the order of an academic essay. For that reason, we include less language than our reviewers might wish to explicate our methodological engagement with the coloniality of the archive. We foreground citations of a number of important scholars/writers on our splash page because these quotes are incisive and provocative---they speak to the heart of the issue of colonial archive and do so in a way that invites thought rather than imparts \quotation{knowledge.} (Marisa Fuentes's reference, in one such quotation, to the desirability of an \quotation{unbiased account} of history is one that she describes as a \quotation{disciplinary demand}---and thus one that she herself questions, as do we.)

An additional question raised in the review about whether embedded slave narratives are \quotation{mere literary devices} is an important one. Are the \quotation{voices} of enslaved people in these embedded narratives so heavily mediated that they cannot count as the voices of the enslaved? This is a question that we hope users of the site and the collection of embedded slave narratives will consider. We also hope that users of the site will be able to discern the deeply mediated nature of all the texts and images collected in the archive: for example, the voice of Bryan Edwards---redacted through his editor, publisher, and readers---is also mediated and shaped as a literary device, much as \useURL[url4][https://ecda.northeastern.edu/item/neu:m0415083s/][][Clara's voice]\from[url4] is within his text.\footnote{See \quotation{The Narrative of Clara} (from Bryan Edwards, {\em The History, Civil and Commercial, of the British Colonies in the West Indies}, 1793), under \quotation{Archive/Browse/Embedded Slave Narratives.}} Our scholarly introductions to items in the archive foreground the production, circulation, and reception histories of each item in order to make these multiple levels of mediation more evident and available to users of the archive.

The review also suggests that we organize the entirety of the site around the collection of embedded slave narratives. We like that idea---except, we have more modes of digital remix on deck. We are working, for instance, on exhibits or clusters of \quotation{text networks} (including one on the Haitian Maroon and \quotation{lord of poison} Makandal) that map constellations of related texts across different genres and languages---from, for instance, trial transcript, to newspaper, to short story, to poem, to novel. We also have users who are designing exhibits around archival materials that they have uncovered. In short, we are working to offer a platform for collaborative engagement, and we expect more experimental models of remixing the archive to appear as we continue to work with new collaborators and explore new possibilities of digital curation and knowledge creation.

\thinrule

\page
\subsection{Elizabeth Maddock Dillon}

Elizabeth Maddock Dillon is a professor in and the chair of the Department of English at Northeastern University and is a codirector of the NULab for Texts, Maps, and Networks there. She is the author of {\em New World Drama: The Performative Commons in the Atlantic World, 1649--1849} (2014) and {\em The Gender of Freedom: Fictions of Liberalism and the Literary Public Sphere} (2004). She is the coeditor, with Michael Drexler, of {\em The Haitian Revolution and the Early US: Histories, Geographies, Textualities} (forthcoming). She has published widely in journals on topics from aesthetics, to the novel and performance in the early Atlantic world, to Barbary pirates. She is the founder of the award-winning crowd-sourced digital archive {\em Our Marathon: The Boston Bombing Digital Archive} and the cofounder and codirector of the {\em Early Caribbean Digital Archive}.

\subsection{Nicole N. Aljoe}

Nicole N. Aljoe teaches English and Africana studies at Northeastern University. Her research focuses on eighteenth-~and nineteenth-century black Atlantic and Caribbean literatures, with a specialization on the slave narrative. She has published widely in these areas, and is the author of~{\em Creole Testimonies: Slave Narratives from the British West Indies, 1709--1836} (2011) and a coeditor of~{\em Journeys of the Slave Narrative in the Early Americas} (2014) and {\em Literary Histories of the Early Anglophone Caribbean: Islands in the Stream} (2018). She is currently at work on two new projects: one examines representations of Caribbean Women of color; the other explores the relationships between narratives of black lives and the rise of the novel in Europe and the Americas in the eighteenth~century. She is the cofounder and codirector of the {\em Early Caribbean Digital Archive}.

\stopchapter
\stoptext