\setvariables[article][shortauthor={Felix}, date={December 2020}, issue={5}, DOI={https://doi.org/10.7916/archipelagos-fx7k-bj27}]

\setupinteraction[title={Culture Jamming in the Caribbean: A Case of Alternative Media through Double Alternativity in Trinidad and Tobago},author={Jonathan J. Felix}, date={December 2020}, subtitle={Culture Jamming in the Caribbean}]
\environment env_journal


\starttext


\startchapter[title={Culture Jamming in the Caribbean: A Case of Alternative Media through Double Alternativity in Trinidad and Tobago}
, marking={Culture Jamming in the Caribbean}
, bookmark={Culture Jamming in the Caribbean: A Case of Alternative Media through Double Alternativity in Trinidad and Tobago}]


\startlines
{\bf
Jonathan J. Felix
}
\stoplines


{\startnarrower\it This essay serves as a written counterpart to a previously published visual case study that explored the cultural production of Caribbean creative Warren Le Platte and his creation of an internet meme series in 2016. In a continuation of that analysis here, Le Platte's work is again positioned as an articulation of {\em alternative media}, a concept defined by subversive, disruptive, or interrogative strategies in response to inequitable relations and power dynamics. Employed as a reflexive critique of his home island of Trinidad, Le Platte's work raises questions regarding the cultural logic of dysfunctionality underscored by his personal experience as a citizen. By employing a visual discourse analysis, this essay continues to explore alternative media in the Caribbean and how activities of \quotation{prosumption} (production + consumption) intersect with practices related to language, identity, and cultural memory in this context. \stopnarrower}

\blank[2*line]
\blackrule[width=\textwidth,height=.01pt]
\blank[2*line]

\subsection[title={Introduction},reference={introduction}]

Public discourse possesses its own distinct features in the \quotation{attention economy} of online social networks in various digital spaces bound by features such as language, identity, cultural memory, and nationality. What is interesting here is how locality bears on the content and creative forms of communicative expressions and how clusters of knowledge are prioritized. In this essay I explore the communication of public discourse through the concept of {\em alternative media}. My work here focuses on a case of an unusual articulation of the common internet-based practice of \quotation{culture jamming} in Trinidad and Tobago, which in this instance is manifest through the phenomena of internet memes. Given that culture jamming is a critical practice, I will give priority to the conceptual framework of alternative media as a means of situating this critical activity in keeping with the way it has been constructed by several scholars.

In general, internet memes can be better understood by first understanding the source material used to create the memetic derivative. This can often lead to a more defined understanding of performative dimension of the internet meme itself and the ways it may find circulation and currency in cyberspace. This is the case of the memes in the work of the creative practitioner Warren Le Platte, featured in this essay. I turn my attention in particular to the idiosyncrasies of Le Platte's meme constellation and how the collection has functioned online.

Even with the range of critical and creative practices within Caribbean culture, regional and international scholars have not conceptualized these practices according to the variety of alternative media frameworks that exist. Given the scarcity of references to Caribbean alternative media in existing scholarship, this essay will attempt to contribute to this critical dialogue by means of a specific case study in which an atypical articulation of internet memes was created and circulated online.\footnote{See Olga Guedes Bailey, Bart Cammaerts, and Nico Carpentier, {\em Understanding Alternative Media} (Maidenhead, UK: Open University Press, 2008); Ken Gelder, {\em Subcultures: Cultural Histories and Social Practice} (London: Routledge, 2007), \useURL[url1][https://doi.org/10.4324/9780203446850]\from[url1]; and Dorothy Kidd, \quotation{The Global Movement to Transform Communications,} in {\em The Alternative Media Handbook}, ed.~Kate Coyer, Tony Dowmunt, and Alan Fountain (Abingdon: Routledge, 2011), 239--48.} In addition to the alternative media gap in the literature, a history of cultural jamming in Trinidad and Tobago has also not been documented, although some authors have attempted to engage with histories of digital culture there, particularly through studies on social media and other online platforms.\footnote{See Daniel Miller and Jolynna Sinanan, {\em Visualising Facebook: A Comparative Perspective} (London: UCL Press, 2017), \useURL[url2][https://doi.org/10.2307/j.ctt1mtz51h]\from[url2]; and Jolynna Sinanan, {\em Social Media in Trinidad: Values and Visibility} (London: UCL Press, 2017).}

Notwithstanding these attempts, my aim here is not to provide an overview of internet memes or culture jamming in the context of Trinidad and Tobago but rather to situate an instance of this practice within the broader scope of alternative media scholarship relating to digital media. This essay also attempts to delineate an instance of \quotation{double alternativity,} in which an articulation of alternative media is further subverted or disrupted, resulting in a novel form that departs from standard practices of counter discursive action. I introduce and employ the term \quotation{meme constellation} to describe the atypical instance of double alternativity discussed here.

\subsection[title={Alternative Media, Culture Jamming, and Public Discourse},reference={alternative-media-culture-jamming-and-public-discourse}]

As a concept, {\em alternative media} positions practices and products of cultural production not simply as interpretations of information but as efforts to close the gap between production and consumption activities mediated through centralized institutions or traditional mass media structures.\footnote{See Chris Atton, {\em Alternative Media} (London: SAGE, 2001).} Practices, processes, and artifacts that fit within this designation can also be seen as expressions of resistance from the \quotation{counterpublics} against a particular locus of power.\footnote{See ibid.; and Chris Atton, \quotation{Introduction: Problems and Positions in Alternative and Community Media,} in {\em The Routledge Companion to Alternative and Community Media}, ed.~Chris Atton (Abingdon: Routledge, 2015), 1--18.} Alternative media concerns itself with structures of discourse and the relations of power that are at play in the polyvocal expressions of both dominant and marginal entities.\footnote{See Atton, {\em Alternative Media}; and Atton, \quotation{Introduction.}} Chris Atton summarizes his typology of alternative media by providing a broad definition that unites various articulations of this concept:

\startblockquote
Alternative and community media are media that bypass the usual channels of commercial production and distribution, and that are most often organised and produced by \quotation{ordinary} people, local communities and communities of interest. It is primarily interested in social and cultural practices that enable people to participate directly in the organisation, production and distribution of their own media, and how these media are used to construct and represent identity and community, as well as to present forms of information and knowledge that are under-represented, marginalised or ignored by other, more dominant media.\footnote{Atton, \quotation{Introduction,} 1.}
\stopblockquote

As a subdivision of alternative media, {\em culture jamming} usually entails the grassroots hijacking of cultural units, which in turn are repurposed through either recreation or reappropriation that results in a vastly different or opposing message to the original context of the cultural unit used.\footnote{In Coyer, Dowmunt, and Fountain, {\em Alternative Media Handbook}, see Tony Dowmunt, with Kate Coyer, introduction, 3--4; and the introduction to chap.~10, 163--64.} Existing cultural resources become the raw materials used for alternative or countercultural discourses expressed through a variety of creative forms, such as the various examples of image macro internet memes that will be discussed later in my analysis. While culture jamming has its roots in the activist efforts of the Situationists in the 1960s and their similar practice of {\em détournement}, culture jamming is not always associated with activism.\footnote{The Situationists was a small Paris-based collective of intellectuals who engaged in artistic interventions as a means of critiquing the idea of modernity within Western societies. See Louise Fabian and Camilla Møhring Reestorff, \quotation{Mediatization and the Transformations of Cultural Activism,} {\em Conjunctions} 2, no. 1 (2015): 1--20, \useURL[url3][https://doi.org/10.7146/tjcp.v2i1.22267]\from[url3]; Tema Milstein and Alexis Pulos, \quotation{Culture Jam Pedagogy and Practice: Relocating Culture by Staying on One's Toes,} {\em Communication, Culture, and Critique} 8, no. 3 (2015): 396--97, \useURL[url4][https://doi.org/10.1111/cccr.12090]\from[url4]; and Mike Mowbray, \quotation{Alternative Logics? Parsing the Literature on Alternative Media,} in Atton, {\em Routledge Companion to Alternative and Community Media}, 21, 28--29.} Détournement usually entailed the reappropriation of film and literature in an attempt to reconfigure these outputs either to convey informational and ideological content that often stood in contrast to their dominant interpretations or to serve the anticapitalistic sentiments of the Situationists.

The alternative media framework considers the relationship between form, content, distribution, and the social functioning of a text as it pertains to the notion of active audiences in their negotiated and oppositional decoding of \quotation{mainstream} practices in popular culture or dominant discourses.\footnote{Coyer, Dowmunt, and Fountain, {\em Alternative Media Handbook}; Bailey, Cammaerts, and Carpentier, {\em Understanding Alternative Media}; Nick Couldry and James Curran, \quotation{The Paradox of Media Power,} in {\em Contesting Media Power: Alternative Media in a Networked World}, ed.~Nick Couldry and James Curran (Lanham, MD: Rowman and Littlefield, 2003), 6--7.} In what follows, I take into account the subversion of traditional institutional constructs, creative processes, social transmission, and oppositional agency, in addition to the elements of language, identity, and cultural memory, in an example of online public discourse in Trinidad and Tobago.

The communication of public discourse through internet memes has already been identified by scholars as a subversive and popular form of alternative media practice.\footnote{See, for example, Carl Chen, \quotation{The Creation and Meaning of Internet Memes in 4chan: Popular Internet Culture in the Age of Online Digital Reproduction,} {\em Habitus}, no. 3 (Spring 2012): 6--19; Coyer, Dowmunt, and Fountain, {\em Alternative Media Handbook}; Graham Meikle, \quotation{Stop Signs: An Introduction to Culture Jamming,} in Coyer, Dowmunt, and Fountain, {\em Alternative Media Handbook}, 166; Vladimír P. Polách, \quotation{Memes, Trojan Horses, and the Discursive Power of Audience,} {\em Human Affairs} 25, no. 2 (2015): 194, 202, \useURL[url5][https://doi.org/10.1515/humaff-2015-0017]\from[url5]; Bradley E. Wiggins and G. Bret Bowers, \quotation{Memes as Genre: A Structurational Analysis of the Memescape,} {\em New Media and Society} 17, no. 11 (2015): 1898--1900, \useURL[url6][https://doi.org/10.1177/1461444814535194]\from[url6].} Semiotic references within internet memes can be seen as both a conceptual index and a point of departure relating to the creation of ongoing discourse. To better understand the case study presented in this essay, I first consider the phenomena of internet memes more broadly and examine how this cultural and communicative artifact or media functions in digital spaces.

\subsection[title={Defining Internet Memes},reference={defining-internet-memes}]

Media scholar Patrick Davison primarily confines internet memes within the realm of humor transmitted online.\footnote{Patrick Davison, \quotation{The Language of Internet Memes,} in {\em The Social Media Reader}, ed.~Michael Mandiberg (New York: New York University Press, 2012), 122--23.} In Davison's theorization, cultural information is \quotation{transmitted} through the meme form, whether this information refers to processes, practices, or ideas. Davison also observes that the formal characteristics and creative processes that surround internet memes are key components in the participatory culture of the internet, particularly through social media.\footnote{See Nick Douglas, \quotation{It's Supposed to Look like Shit: The Internet Ugly Aesthetic,} {\em Journal of Visual Culture} 13, no. 3 (2014): 314--39, \useURL[url7][https://doi.org/10.1177/1470412914544516]\from[url7].} By extension, these components can also be theorized as alternative media, given their usual \quotation{de-professionalized aesthetic} and disruptive form and content creation,\footnote{See Natalie Fenton, \quotation{Mediating Hope: New Media, Politics, and Resistance,} {\em International Journal of Cultural Studies} 11, no. 2 (2008): 230--48, \useURL[url8][https://doi.org/10.1177/1367877908089266]\from[url8].} in contrast to more \quotation{refined} forms of cultural production presented through traditional and commercial mass media structures.\footnote{See Atton, {\em Alternative Media}; Christian Fuchs, \quotation{Alternative Media as Critical Media,} {\em European Journal of Social Theory} 13, no. 2 (2010): 173--92, \useURL[url9][https://doi.org/10.1177/1368431010362294]\from[url9]; and Ryan M. Milner, \quotation{Hacking the Social: Internet Memes, Identity Antagonism, and the Logic of Lulz,} {\em Fibreculture Journal}, no. 22 (December 2013), FCJ-156, \useURL[url10][http://twentytwo.fibreculturejournal.org/fcj-156-hacking-the-social-internet-memes-identity-antagonism-and-the-logic-of-lulz/]\from[url10].}

Within the Trinbagonian digital landscape, the {\em image macro} is the most commonly used internet meme format circulated through local social media outlets and news websites.\footnote{An image macro is a digital image that has a word or caption superimposed on it to generate a typically humorous meaning. It is the most common form of internet meme. See Andre Bagoo, \quotation{Richards: Not Sure It's Legal,} {\em Trinidad and Tobago Newsday}, 24 October 2014, \useURL[url11][https://archives.newsday.co.tt/2014/10/24/richards-not-sure-its-legal/]\from[url11]; Janine Mendes-Franco, \quotation{Meme vs.~Movie Theatre---Dawn of Consumer Justice in Trinidad and Tobago?,} {\em Global Voices}, 6 April 2016, \useURL[url12][https://globalvoices.org/2016/04/06/meme-vs-movie-theatre-dawn-of-consumer-justice-in-trinidad-tobago/\%23]\from[url12]; and Chen, \quotation{Creation and Meaning of Internet Memes in 4chan.}} This may be because of how easily this meme form is transmitted by internet users, as well as how easy it is for this format to be copied or manipulated (in contrast to a video or an animated GIF image). Composed in most cases of a single image with a superimposed word, phrase, or caption, the image macro is often used to express pithiness, humor, or sarcasm.\footnote{See Sean Rintel, \quotation{Crisis Memes: The Importance of Templatability to Internet Culture and Freedom of Expression,} {\em Australasian Journal of Popular Culture} 2, no. 2 (2013): 253--71, \useURL[url13][https://doi.org/10.1386/ajpc.2.2.253_1]\from[url13]; and Kate Brideau and Charles Berret, \quotation{A Brief Introduction to Impact: \quote{The Meme Font,}} {\em Journal of Visual Culture} 13, no. 3 (2014): 307--13, \useURL[url14][https://doi.org/10.1177/1470412914544515]\from[url14].} Sometimes the text is white with a black outline; it may also be either completely black or white, depending on the nature of the image.\footnote{See Michele Knobel and Colin Lankshear, eds., {\em A New Literacies Sampler} (New York: Peter Lang, 2007).} Figure 1 is an image macro that bears the defining marks of a localized aesthetic and content, given the comparative visual and textual reference.

While the term {\em meme} has its roots in the mid-1970s in Richard Dawkins's ideas on sociocultural reproduction, in the early 1980s a more concrete expression of the concept emerged, leading to how {\em meme} is now commonly understood by Davison and other scholars over the last twenty years. During this period, internet users were able to categorize content for clarity using simple text, punctuation, and alphanumeric-based characters, commonly known today as {\em emoticons}. The need for this measure of clarity stemmed from the fact that in the early days of online forums, it became a challenge for some users to differentiate between humorous communication and other types of exchanges.\footnote{See Patrick Davison, \quotation{Because of the Pixels: On the History, Form, and Influence of MS Paint,} {\em Journal of Visual Culture} 13, no. 3 (2014): 275--97, \useURL[url15][https://doi.org/10.1177/1470412914544539]\from[url15]; and Davison, \quotation{Language of Internet Memes.}}

\placefigure{Natural Disasters}{\externalfigure[images/felix/fig1.png]}
Ryan Milner comments that internet memes are part of the lingua franca of the modern internet---a visual language representative of the cultural dynamics of social media. He notes that cultural logics often operate in tandem with contrasting social irrationalities to reinforce the binary oppositions of cultural dynamics.\footnote{See Ryan M. Milner, \quotation{Pop Polyvocality: Internet Memes, Public Participation, and the Occupy Wall Street Movement,} {\em International Journal of Communication} 7 (2013), \useURL[url16][https://ijoc.org/index.php/ijoc/article/view/1949]\from[url16]. See also Ryan M. Milner, \quotation{The World Made Meme: Discourse and Identity in Participatory Media} (PhD diss., University of Kansas, 2012), \useURL[url17][https://kuscholarworks.ku.edu/handle/1808/10256]\from[url17]; and Jean Burgess, \quotation{Hearing Ordinary Voices: Cultural Studies, Vernacular Creativity, and Digital Storytelling,} {\em Continuum} 20, no. 2 (2006): 201--14, \useURL[url18][https://doi.org/10.1080/10304310600641737]\from[url18].} Because cultures are composed of varying modes of thought, dominant ideas usually define social rationalities---in other words, normative patterns of sociocultural thought and practice.

Yet ideas that fall outside dominant rationalities are often considered irrational or as counter-rationalities that constitute the binary between dominant and marginal discourses. The interplay between the margins and center often influence paradigm shifts and cultural articulation(s). In this regard, internet memes may express a variety of rationalities and discourses. The internet memes featured in this essay provide examples of a dominant online social practice and the expression of a dominant discourse through a counter-rationality in an atypical instance of culture jamming.

\subsection[title={The Structure of Internet-Based Discourse},reference={the-structure-of-internet-based-discourse}]

The technological affordance of online social networking in digital visual culture also calls attention to the development of participatory practices in twenty-first-century societies, particularly concerning the ways the digital public sphere is constructed today.\footnote{See Atton, {\em Alternative Media}; and Davison, \quotation{Language of Internet Memes,} 130--32.} In this regard, \quotation{technological affordance} pertains to the way a device, media, system, or configuration possesses characteristics that influence its functionality.\footnote{William Lidwell, Kritina Holden, and Jill Butler, {\em Universal Principles of Design: 125 Ways to Enhance Usability, Influence Perception, Increase Appeal, Make Beter Design Decisions, and Teach through Design} (Beverly, MA: Rockport, 2010), 20.} For example, the way online social networks provide opportunities for the posting of multimedia content would in part determine how these opportunities might be exploited in a given context. With this understanding, social media functions as a nexus for mediated interaction between individuals and groups by means of the potential that is already part of its design through the connectivity of the internet. Of importance in this sphere, I argue, is not just {\em what} individuals and groups say but {\em how} they speak.

The internet operates as a networked index that functions as a site of communication and transaction. These core features of the internet allow for collaborative and participatory uses, as well as political ends, since users are often engaged in wide-ranging forms of expression and transmission.\footnote{See Chris Atton, \quotation{A Brief History: The Web and Interactive Media,} in Coyer, Dowmunt, and Fountain, {\em Alternative Media Handbook}, 59--65; and Edward Comor, \quotation{Digital Prosumption and Alienation,} {\em Ephemera} 10, nos. 3--4 (2010): 439--54, \useURL[url19][http://www.ephemerajournal.org/contribution/digital-prosumption-and-alienation]\from[url19].} As such, the development of the internet and corresponding technologies up to the present time has given rise to the concept of \quotation{prosumption}---wherein the {\em production} and {\em consumption} of information occur simultaneously. The once disparate acts of production and consumption have now compounded into a singular, indivisible act, becoming a dual expression of both participatory and autonomous agency.\footnote{See Brian O'Neill, J. Ignacio Gallego Pérez, and Frauke Zeller, \quotation{New Perspectives on Audience Activity: \quote{Prosumption} and Media Activism as Audience Practices,} in {\em Audience Transformations: Shifting Audience Positions in Late Modernity}, ed.~Nico Carpentier, Kim Christian Schrøder, and Lawrie Hallett (New York: Routledge, 2013), 157--71, \useURL[url20][https://doi.org/10.4324/9780203523162]\from[url20]; Roberta Paltrinieri and Piergiorgio Esposti, \quotation{Processes of Inclusion and Exclusion in the Sphere of Prosumerism,} {\em Future Internet} 5, no. 1 (2013): 21--33, \useURL[url21][https://doi.org/10.3390/fi5010021]\from[url21]; Melita Zajc, \quotation{Social Media, Prosumption, and Dispositives: New Mechanisms of the Construction of Subjectivity,} {\em Journal of Consumer Culture} 15, no. 1 (2015): 43--44, \useURL[url22][https://doi.org/10.1177/1469540513493201]\from[url22]; and Yochai Benkler, \quotation{Sharing Nicely: On Shareable Goods and the Emergence of Sharing as a Modality of Economic Production,} in Mandiberg, {\em Social Media Reader}, 17--23 (originally published in {\em Yale Law Journal} 114, no. 2 2004: 273--358, \useURL[url23][https://doi.org/10.2307/4135731]\from[url23]).} Media scholar Melita Zajc further expounds on this idea:

\startblockquote
Rather than asking what social media are, the focus should be on exploring how they can be conceptualized in relation to their use. The tradition of audience studies within communication and media studies testifies to aspirations for a desired ideal of audience participation, an ideal that seems to have reached its fulfillment within participatory practices of social media. The scholarly community appears to be divided with regard to this fulfillment. The conceptual approach, based on the notion of the dispositive, shows that diverging processes can indeed take place within the same media use. This approach offers a more complex insight into the use of social media. It does not provide a one-sided answer about the potentials that social media offer for more substantial participation in political processes, in the interpretation of media texts, and in media production.\footnote{Zajc, \quotation{Social Media, Prosumption, and Dispositives,} 44.}
\stopblockquote

Moreover, ideas of the \quotation{the expert} or \quotation{professional} are undermined in their linkage to centralized production, since a key aspect of alternative media involves collective or participatory engagement made possible through more inclusive access to the means of production and channels of distribution.\footnote{See Atton, {\em Alternative Media}; and Mowbray, \quotation{Alternative Logics?,} 23--25.} Davison points out that certain technical and communicative limitations are among the many other contingencies that influence the transgressive nature of internet memes.\footnote{See Davison, \quotation{Language of Internet Memes.} See also Douglas, \quotation{It's Supposed to Look like Shit.}} These contingencies have greatly contributed to the decentered production of internet memes, in contrast to the institutionalized production of the traditional mass media industries that are often commercial in nature.

With social media use currently epitomizing prosumption in its fullness, scholars have observed how the internet has blurred previously established social and cultural demarcations, resulting in a variety of ways relations of power are reconfigured.\footnote{See, for example, Christian Fuchs, \quotation{Digital Prosumption Labour on Social Media in the Context of the Capitalist Regime of Time,} {\em Time and Society} 23, no. 1 (2014): 97--123, \useURL[url24][https://doi.org/10.1177/0961463X13502117]\from[url24]; O'Neill, Gallego Pérez, and Zeller, \quotation{New Perspectives on Audience Activity}; Jay Rosen, \quotation{The People Formerly Known as the Audience,} in Mandiberg, {\em Social Media Reader}, 13--16; Daren C. Brabham, {\em Crowdsourcing} (Cambridge, MA: MIT Press, 2013); and Benkler, \quotation{Sharing Nicely.}} The {\em crowdsourcing} labor model exemplifies this reconfiguration, since the internet serves as a nexus for shared communication and creativity (between organizations and online communities), resulting in mutual benefits that are the sum of equitable relations of power.\footnote{Crowdsourcing is an organizational labor model that uses internet-based platforms to engage in collective problem-solving through individual and collaborative work for achieving clearly defined ends. See Brabham, {\em Crowdsourcing}; and Linda K. Börzsei, \quotation{Makes a Meme Instead: A Concise History of Internet Memes,} {\em New Media Studies Magazine}, no. 7 (2013), \useURL[url25][http://works.bepress.com/linda_borzsei/2/]\from[url25].} Similar to how participatory communication and creativity are built on the production and consumption of information, communication and creativity bear on how public discourse is organized and understood. The construction and communication of public discourse through alternative media in Trinidad and Tobago underscores the significance of language, identity, and cultural memory. With the internet as a space for social interaction, internet memes, as a specific example in the following case study, function as {\em signifiers} of matters of public interest. In this local setting, the circulation and currency of internet memes manifest particular patterns.

Given that the replication of structure and content is a defining aspect of memes, these features suggest a form of \quotation{digital storytelling} in which form eclipses authorship in the construction of \quotation{public culture.}\footnote{See Jean Burgess and Joshua Green, {\em Youtube: Online Video and Participatory Culture} (Oxford: Polity, 2009).} As such, the ongoing significance of internet memes lies in their growing evolution and preponderance, including the relationship between these digital artifacts and the physical lifeworld. The \quotation{open} nature of the internet reinforces the structure of social media, which provides a participatory agency to users indicative of the influx of content developed from the grassroots communities that eclipses the centralized, formal institutionality of the mass media industries.\footnote{See Francis P. Barclay, C. Pichandy, and Anusha Venkat, \quotation{India Elections 2014: Time-Lagged Correlation between Media Bias and Facebook Trend,} {\em Global Journal of Human-Social Science} 15, no. 2 (2015): 29--41; and Atton, {\em Alternative Media}.}

In this regard, media theorists Jean Burgess and Joshua Green suggest that any attempt at understanding the cultural flow of the digital landscape requires a shift in the way one views cultural production. This requires a conceptual reorientation that positions online practices via social media practices along a continuum of cultural production, which favors prosumption, and a negation of the \quotation{professional versus amateur} divide evident in more industrialized paradigms.\footnote{See Burgess and Green, {\em Youtube}; see also Barclay, Pichandy, and Venkat, \quotation{India Elections 2014.}} In this sense, according to Burgess and Green, social media usage presents a range of modalities for the production and consumption of media, and as such, social media platforms provide tools that enable individuals and groups to produce and consume texts without positioning users as either professionals or amateurs. In addition, social media bears the technical infrastructure that allows for online communication insofar as it provides an environment for various forms of participatory practice.\footnote{See Burgess and Green, {\em Youtube}; and Burgess, \quotation{Hearing Ordinary Voices.}}

\subsection[title={Social Media and Internet Memes in Trinidad and Tobago},reference={social-media-and-internet-memes-in-trinidad-and-tobago}]

Facebook has been regarded as one of the world's most popular social networks, and the nature of its design makes it the most participatory social media platform to date.\footnote{See Richard Rutter, Stuart Roper, and Fiona Lettice, \quotation{Social Media Interaction, the University Brand and Recruitment Performance,} {\em Journal of Business Research} 69, no. 8 (2016): 3096--104, \useURL[url26][https://doi.org/10.1016/j.jbusres.2016.01.025]\from[url26]; and Richard Campbell, Christopher R. Martin, and Bettina Fabos, {\em Media and Culture: An Introduction to Mass Communication}, 8th ed. (Boston: Bedford/St.~Marten's, 2012).} This is partly the result of Facebook's construction as a convergent media platform, whereby a variety of creative forms, content, and technologies intersect.\footnote{On media convergence, see Henry Jenkins, \quotation{The Cultural Logic of Media Convergence,} {\em International Journal of Cultural Studies} 7, no. 1 (2004): 33--43, \useURL[url27][https://doi.org/10.1177/1367877904040603]\from[url27]; Klaus Bruhn Jensen, \quotation{The Double Hermeneutics of Audience Research,} {\em Television and New Media} 20, no. 2 (2019): 142--54, \useURL[url28][https://doi.org/10.1177/1527476418811103]\from[url28]; and Sinanan, {\em Social Media in Trinidad}.} By design, Facebook is a relatively accessible platform for internet users across a variety of digital devices and access points, unlike Instagram, for example, which is by design primarily restricted to smartphone usage.

Anthropologist Jolynna Sinanan identifies social media in Trinidad and Tobago as a space often devoted to the idea of visibility or, more specifically, to a sense of personal unpretentiousness. Facebook in particular has been noted by researchers as a platform for locals to negotiate both individual freedom and national community. As such, Sinanan identifies {\em normativity} as the underlying value of online discourse across Facebook in Trinidad and Tobago, as a confluence of language, identity, and cultural memory.\footnote{See Miller and Sinanan, {\em Visualising Facebook}.} These features, characteristic of Facebook use in Trinidad and Tobago, are neither prominent nor evident in other foreign contexts.

Similarly, both Daniel Miller and Sinanan note that the visual content found on online social networks in Trinidad and Tobago function as a kind of visual language that often relies on memes. In light of this, Miller and Sinanan argue that online discourse in Trinidad and Tobago can be observed in the posting of content on social networks, particularly Facebook, because these are instances wherein individuals and groups function as \quotation{objects} that transmit and embody a range of values they identify with through the use of language and the invocation of cultural memory.\footnote{Shahad Ali, \quotation{\quote{Meanwhile in T&T} Discusses Internet Memes,} {\em Trinidad Guardian}, 17 September 2012, \useURL[url29][https://guardian.co.tt/article-6.2.430946.7a94c69891]\from[url29].} Both Miller and Sinanan posit the idea that sociocultural normativity is an overarching feature of social media use in Trinidad and Tobago.\footnote{See Miller and Sinanan, {\em Visualising Facebook}; and Sinanan, {\em Social Media in Trinidad}.}

Let me point out that the content and circulation of local image macros is the way cultural critique and social commentary are presented. In addition, users construct a sense of local identity through the humor and reflexivity of the images they post. Michele Knobel and Lankshear also note in their work that these kinds of practices are connected to meaning-making, identity formation, and cultural hierarchy:

\startblockquote
Analysis and dissection of online memes can be used to explore why some ideas are more easily replicated, are more fecund and have more longevity than others, and what the consequences of this are or might be. Studying online memes that aim at promoting social critique can help educators to rethink conventional approaches to critical literacy that all too often operate at the level of text analysis without taking sufficient account of the social practices, ideas, affinities and new forms of social participation and cultural production that generated the phenomenon under examination.\footnote{Michele Knobel and Colin Lankshear, \quotation{Online Memes, Affinities, and Cultural Production,} in Knobel and Lankshear, {\em New Literacies Sampler}, 225.}
\stopblockquote

In the case of the usual articulation of culture jamming, local creative professional Warren Le Platte produced a popular internet meme series documenting a personal experience with a local franchise. This is an unusual occurrence because most internet users do not produce memes as sequential narratives. Furthermore, most meme creators remain anonymous---that is, an internet meme usually functions to direct attention to itself rather than to the person(s) who created it. I use \quotation{meme constellation}---a single collection of memes having a unifying theme, structure, or object---to describe the series produced by Le Platte. This analysis takes into account the visual characteristics and semiotic references that mark the series and considers how these bear on the communication of public discourse.

Two key questions undergird my analysis here, namely, How do image-and-text-based internet memes gain their currency across Trinidad and Tobago's digital landscape? And what local discourses are communicated through the internet meme form? Both issues have significant implications regarding the agency of the (online) active audience in Trinidad and Tobago and yield insight into the ways locals negotiate their use of language, national identity, and cultural memory through lived experience. Using textual analysis ({\em visual} and {\em semiotic} analysis, respectively), a corresponding interpretive discussion will follow.

\subsection[title={Enter Warren Le Platte},reference={enter-warren-le-platte}]

Before my detailed discussion of Warren Le Platte's experience and subsequent work, I will first provide some context to his designation as a creative professional. With a creative career spanning more than two decades, Le Platte has grown to become a well-respected multidisciplinary professional with a practice that includes photography, graphic design, illustration, writing, and cartooning.\footnote{See \quotation{Shades of a Crayon---Warren Le Platte: Graphic Designer, Photographer, Illustrator, Cartoonist, Writer,} A BigBox of Crayons, 2015, \useURL[url30][https://www.abigboxofcrayons.com/blog/20-shades-of-a-crayon-warren-le-platte]\from[url30]; \quotation{Warren Le Platte: Graphic Designer, Illustrator, Photographer,} Bocas Lit Fest, 2019, \useURL[url31][https://www.bocaslitfest.com/author/warren-le-platte/]\from[url31]; and Le Platte Studios, \useURL[url32][https://leplattestudios.com]\from[url32].} The range of Le Platte's practice has covered the realms of academia and industry, and he has also been involved in local media and with carnival festivities. In his ongoing work to develop and convey a visual language that exemplifies a distinct Caribbean aesthetic, Le Platte's latest venture has been the development of the board game \quotation{Santimanitay,} based on Trinidad and Tobago's cultural history and on a contemporary articulation of carnival festivities.\footnote{See \quotation{Board Game: Play Mas with \quote{Santimanitay,}} {\em Sweet TnT Magazine}, February 2018, \useURL[url33][https://sweettntmagazine.com/board-game-santimanitay/]\from[url33]; Maria Tumolo, \quotation{An Interview with Warren Le Platte: Creator of the Board Game \quote{Santimanitay,}} {\em The Tiger Tales} (blog), 14 January 2018, \useURL[url34][https://thetigertales.co.uk/interview-warren-le-platte-creator-board-game-santimanitay/]\from[url34]; and Paula Lindo, \quotation{Board Games Trending Offline,} {\em Trinidad and Tobago Newsday}, 3 June 2020, \useURL[url35][https://newsday.co.tt/2020/06/03/board-games-trending-offline/]\from[url35].} Le Platte's career and visibility as a creative practitioner presents an interesting frame of reference with which to better understand his internet meme constellation.

In response to the worldwide March 2016 release of the popular film {\em Batman v Superman: Dawn of Justice}, Le Platte posted on his personal Facebook page a series of thirty-one internet memes utilizing local vernacular. In doing this, Le Platte narrated his unsuccessful attempt to view the film at one of the local multiplex franchises and an unpleasant customer-service experience that followed. To summarize: Le Platte had gone to considerable lengths to attend the cinema at the end at an exhausting day at work, even missing an opportunity to purchase some local cuisine. During the film's screening, technical difficulties consistently interrupted the film, making it more-or-less impossible to enjoy, yet he was unable to get a refund for his ticket.

In the end, Le Platte was unable to see the film he had anticipated and was not offered a cash refund by the cinema staff, who he claims treated him in a disagreeable fashion. Le Platte made a point to note that his experience at this particular multiplex was a first for him and also a foreign one, given the previous pleasant experiences he had had with traditional and longstanding single-theater cinema outlets. Mere hours after posting the series online, Le Platte's culture jamming activity garnered several hundred interactions by local Facebook users, including comments and shares, and went on to draw the attention of the international media collective {\em Global Voices} (see fig.~2).\footnote{See Mendes-Franco, \quotation{Meme vs.~Movie Theatre.}} In the context of this example, at least five key elements are of importance.

\placefigure{Warren Le Platte Facebook}{\externalfigure[images/felix/fig2.png]}
Firstly, the film Le Platte attempted to view at the local cinema was of considerable popular interest both locally and worldwide, and both the period of the film's international release and the creation of Le Platte's series of memes coincided. More importantly, the film at the center of Le Platte's narrative was a superhero film, featuring popular characters such as Batman and Superman, along with others of DC Comics lore, all of whom are part of the contemporary popular culture mythos.

Given this context, I argue that the {\em currency} of Le Platte's series was connected to its {\em circulation}; at the same time, its {\em circulation} was a result of its {\em currency}. While comic books are key artifacts of popular culture, few comic book characters have as much intergenerational popularity and widespread appeal as Batman, Superman, and the other related franchised characters. Scholars have noted that the popularity of comic book characters is linked to their ideological signification, as well as to the social function of comic books and graphic novels as narrative forms of entertainment and identification.\footnote{See Matthew P. McAllister, Edward H. Sewell Jr., and Ian Gordon, \quotation{Introducing Comics and Ideology,} in {\em Comics and Ideology}, ed. Matthew P. McAllister, Edward H. Sewell Jr., and Ian Gordon (New York: Peter Lang, 2001), 1--13, \useURL[url36][http://www.peterlang.de/download/datasheet/46603/datasheet_69249.pdf]\from[url36]; and John Fiske, \quotation{The Cultural Economy of Fandom,} in {\em The Adoring Audience: Fan Culture and Popular Media}, ed.~Lisa A. Lewis (London: Routledge, 1992), 30--49.}

John Fiske's theory of popular culture is useful here. Fiske posits that meaning-making and the construction of social identity are but a sample of various factors that influence the selection of pop-culture artifacts that find currency among individuals and groups.\footnote{See John Fiske, {\em Understanding Popular Culture} (London: Routledge, 1994).} However, Le Platte's culture jamming practice involved the creation of derivative work that had been twice removed from its initial context: the initial reference in in popular media and its subsidiary use in existing internet memes.

Secondly, the narrative constructed by Le Platte details a highly unpleasant customer-service experience at a popular local establishment, referencing the common narrative of poor customer service by organizations in Trinidad and Tobago. Within this culture jamming narrative, Le Platte thus positioned himself as a proxy for countless other nationals who happen to share similar experiences and sentiments, as evidenced by the online responses posted by local social media users. This move by Le Platte reinforces public discourse on local {\em commess} culture---local parlance for a cultural rationality of disorder.\footnote{{\em Konmès} or {\em commess} is a Caribbean-based term referring to the creation of extreme confusion, scandal, and disorder. See {\em Dictionary of Caribbean English Usage}, ed.~Richard Allsopp (1996; repr., Kingston: University of the West Indies Press, 2003), s.v. \quotation{konmès.}}

Le Platte expresses the challenges he experienced in his moviegoing efforts by referencing the commercially and critically acclaimed {\em Lord of the Rings} film franchise, which is known for its unusually lengthy screening times. In figure 3 Le Platte presents his experience at the local cinema franchise as being tediously lengthy, given the technical difficulties he encountered, by using a promotional image from the {\em Lord of the Rings} film franchise; he also superimposes a promotional image from the film he attempted to watch at the cinema. This juxtaposition creates a hyperbolic comedic effect. As Le Platte likens his failure to view {\em Batman v Superman} to watching another film franchise altogether, he links his failed moviegoing experience to spending more time than needed at the cinema.

Le Platte's placement of himself as a comic book character in his personal narrative (see fig.~7) served as an ideological signifier for prevailing ideas regarding the Trinbagonian experience of commess or confusion, while also performing as an institutional critique. Moreover, Le Platte's identity as author of this narrative plays an important role in his identification of institutional dysfunctionality being a prominent part of the local cultural landscape. He is identified as an everyman, placed in an unusual circumstance of a failed movie-watching attempt, yet still a very common occurrence with respect to bad customer service and the impaired operation of local establishments. The local rapport of Le Platte's meme constellation also underscores the observation by Elad Segev, Asaf Nissenbaum, Nathan Stolero, and Limor Shifman regarding the use of internet memes as prime vehicles for mainstream discourses on popular culture.\footnote{See Elad Segev, Asaf Nissenbaum, Nathan Stolero, and Limor Shifman, \quotation{Families and Networks of Internet Memes: The Relationship Between Cohesiveness, Uniqueness, and Quiddity Concreteness,} {\em Journal of Computer-Mediated Communication} 20, no. 4 (2015): 417--33, \useURL[url37][https://doi.org/10.1111/jcc4.12120]\from[url37].} Their work on internet memes highlights politics, an unpopular topic in alternative national contexts, yet politics is far from being a marginal discourse within Trinidad and Tobago's cultural landscape, as evidenced by the institutional critique presented in Le Platte's culture jamming that evokes a sense of national identity and cultural memory through the mediated presentation of his lived experience.\footnote{See Curwen Best, {\em The Politics of Caribbean Cyberculture} (New York: Palgrave Macmillan, 2008), \useURL[url38][https://doi.org/10.1057/9780230610132]\from[url38].}

\placefigure[here]{So I spend 4 hours in de cinema\ldots{}}{\externalfigure[issue05/felix/fig3.png]}


Thirdly, Le Platte employed the internet meme---one of the common communication tools of the so-called digital age---as provocation to dialogue with Facebook users. This is particularly important, since internet memes, as noted by Milner, form part of the lingua franca of twenty-first century. In Le Platte's case, his use of the visual language of internet memes to evoke a shared and personal experience among nationals is a clear example of this phenomenon. This further supports Limor Shifman, Hadar Levy, and Mike Thelwall's argument regarding the use of internet memes in the construction of nationalistic discourses through visually based humor.\footnote{See Limor Shifman, Hadar Levy, and Mike Thelwall, \quotation{Internet Jokes: The Secret Agents of Globalization?,} {\em Journal of Computer-Mediated Communication} 19, no. 4 (2014): 727--43, \useURL[url39][https://doi.org/10.1111/jcc4.12082]\from[url39]. See also Limor Shifman, \quotation{Memes in a Digital World: Reconciling with a Conceptual Troublemaker,} {\em Journal of Computer-Mediated Communication} 18, no. 3 (2013): 362--77, \useURL[url40][https://doi.org/10.1111/jcc4.12013]\from[url40].} Le Platte's work demonstrates a quality of cohesiveness and quiddity evidenced through appeals to language, identity, and cultural memory by references to local sites and institutions, as is the case with figure 6, and to vernacular, idioms, and ideas, as exemplified in figure 4. Nationals confirm this with their favorable comments in response to his series of online posts.

\placefigure{What is my life?}{\externalfigure[images/felix/fig4.png]}
While disjointed in terms of the continuity of visual references, when taken as a whole, the images in Le Platte's meme constellation cohesively and emotively narrate his personal experience. However, the disorderly nature of the visual references causes his work to possess a character of quiddity that is accentuated by use of colloquial language. Figure 4 features a popular character from the US television series {\em The Fresh Prince of Bel-Air}. The irony of this particular image lies in the fact that this sitcom character is generally always in positive spirits. The tone of despondency presented in this component of the series is used to express not only Le Platte's moviegoing failure but also his lack of success in procuring local cuisine, doubling his frustration and disappointment. The colloquial caption also directly refers to Le Platte's reflection on his experience as he left the cinema.

Le Platte employed visual and textual means through his meme constellation to communicate both meaning and information beyond basic denotative dimensions of the images themselves. Additionally, Le Platte reappropriated the images used in the creation of his internet meme series: the connotative dimensions of these familiar images defied their more common uses among internet users and across social networks. Many of the images Le Platte used in the series were already in circulation, and he pulled them from existing and already popular internet memes. In essence, Le Platte made a series of memes from the materials of existing-image macro internet memes---primarily using the {\em imagery} featured in them as opposed to their {\em text-based content}. Also important is that in the creation of his meme constellation Le Platte did not employ the usual typeface, Impact, but instead used Myriad, a typeface not typical to the image macro format. While the visual and textual content in Le Platte's work references popular films, celebrities, television series, local sites, and more, this typeface detail causes Le Platte's work to depart from standard visual internet meme image macro forms.

In addition, a {\em double use} of pastiche and bricolage can be observed here: first, in the case of the construction of the original memes, since images are removed from their original contexts; and second, in how Le Platte reappropriates these images into his series as a {\em templatable format} and {\em reproductive articulation}, based on Bradley E. Wiggins and G. Bret Bowers's conceptualization of the internet meme genres. Wiggins and Bowers note a distinction between what can be termed as spreadable media, emergent memes, and memes (proper). In this taxonomy, {\em spreadable media} refers to \quotation{untouched} digital artifacts (in their original or base contexts) and the internet-based systems that enable their networked diffusion and circulation through personal sharing by members of an online community rather than through top-down broadcast media; {\em emergent memes} refer to the initial instance of the articulation of a digital artifact that may be reappropriated or altered through pastiche or bricolage; finally, {\em memes}, in the proper sense of the term/ concept in popular usage, refers to the ongoing templatable format and reproductive articulation(s) spawned from the initial first-instance emergent meme.\footnote{See Wiggins and Bowers, \quotation{Memes as Genre.}} Le Platte created a consistent structure with each component of his meme constellation by juxtaposing text and image to narrate a personal experience. The confluence of cultural referencing and reappropriation in Le Platte's work presents a rich tapestry of references, which then imparts an added layer of depth to the series. Observing the numerous popular culture references in Le Platte's work accords with Knobel and Lankshear position that intertextuality functions as a cue for creative engagement particularly in the case of internet memes because it fuels public discourse and the collective narrative of digital spaces.\footnote{See Knobel and Lankshear, {\em New Literacies Sampler}. See also Paul Lopes, \quotation{Culture and Stigma: Popular Culture and the Case of Comic Books,} {\em Sociological Forum} 21, no. 3 (2006): 387--414, \useURL[url41][https://link.springer.com/article/10.1007/s11206-006-9022-6]\from[url41].}

A fourth key element is that Le Platte's use of local parlance necessarily intensified the series' resonance with Trinbagonian online audiences. This point regarding language also corresponds with Shifman, Levy, and Thelwall's comment on linguistic proximity bearing on the circulation of internet memes and the contribution of Segev, Nissenbaum, Stolero, and Shifman on how language categorizes internet memes based on their textual characteristics, coherency, and singularity and their relationship to other internet meme forms.\footnote{See Shifman, Levy, and Thelwall, \quotation{Internet Jokes}; and Segev, Nissenbaum, Stolero, and Shifman, \quotation{Families and Networks of Internet Memes.}}

In an unusual fashion, Le Platte's meme constellation featured a considerable degree of text-based communication in the construction of his personal narrative. This is not the norm for internet memes, which usually contain pithy written and visual content. It is also uncommon for internet users to produce {\em individual} memes in a fashion that allows for a coherent sequential narrative as an outcome. The measure of visual and textual language employed by Le Platte, in addition to the response by locals online, subverts Piia Varis and Jan Bloomaert's notion of memes being a significant form of phatic communication, since this line of reasoning negates the potential of internet memes to deliver content beyond banal humor.\footnote{See Piia Varis and Jan Blommaert, \quotation{Conviviality and Collectives on Social Media: Virality, Memes, and New Social Structures,} {\em Multilingual Margins} 2, no. 1 (2015): 31--45, \useURL[url42][https://doi.org/10.14426/mm.v2i1.55]\from[url42].}

What can also be observed in this unusual instance is how Le Platte's meme constellation supports Burgess's view of the memetic employed in \quotation{digital storytelling} as a narrative form.\footnote{See Burgess, \quotation{Hearing Ordinary Voices}; and Burgess and Green, {\em Youtube}.} Yet Le Platte's culture jamming practice stands in stark contrast to this idea, since authorship, an underexplored dimension of internet meme scholarship, is a defining feature in this specific case. Le Platte's work is illustrative of an atypical employment of alternative media in general and internet memes in particular, since memes are content-driven artifacts rather than author-driven or {\em auteur-like} devices, as in the case of music, film, visual art, or other creative artifacts. Le Platte's work defies this common characterization.

Le Platte's self-contained meme constellation combines clusters of images from the broader scope of Western popular culture with very specific references to Trinidad and Tobago, in terms of locality and dialect. Le Platte used these elements to lament his negative experience as a consumer of public entertainment and, by extension, to reflect public discourse of institutional dysfunctionality or commess.

In a Michael Jackson meme (fig.~5), Le Platte uses these elements to underscore his disappointment and disenfranchisement that were the result of substandard service by what may have seemingly been considered a respectable local franchise. He expresses initial optimism, despite the ongoing challenges he has experienced, before ultimately becoming resigned. The text, written in local dialect, roughly translates to, \quotation{The movie started to play again; lots of action; I'm paying close attention to the screen,} conveying the sense of excitement and enthusiasm Le Platte initially felt.

\placefigure[here]{So I dey like\ldots{}}{\externalfigure[issue05/felix/fig5.png]}


In another example, featuring an image of Michelle Obama, Le Platte evokes a sense of cultural memory with a nostalgic reference to the now defunct Globe Cinema outlet, situated in the heart of the nation's capital (fig.~6). Having served local patrons for close to eighty years since it first opened in 1933, the Globe Cinema is considered a historical site and an important aspect of local heritage.\footnote{See Andre Bagoo, \quotation{Owners Hope to Get \$25M for Property,} {\em Trinidad and Tobago Newsday}, 21 March 2013, \useURL[url43][https://archives.newsday.co.tt/2013/03/21/owners-hope-to-get-25m-for-property/]\from[url43]; and Mark Wilson, \quotation{The Cinema Glory Days Are Coming Back,} {\em Caribbean Beat}, March--April 2002, \useURL[url44][https://www.caribbean-beat.com/issue-54/glory-days-coming-back\%23axzz6KH65Kv4z]\from[url44].} However, the popularity of the Globe Cinema was eclipsed by the rise of VCR technologies in the 1980s, the introduction of cable television in the 1990s, and multiplex cinema chains in the early 2000s.\footnote{See Jocelyne Guilbault, \quotation{The Question of Multiculturalism in the Arts in the Postcolonial Nation-State of Trinidad and Tobago,} {\em Music and Politics} 5, no. 1 (2011), \useURL[url45][https://doi.org/10.3998/mp.9460447.0005.101]\from[url45]; and Wilson, \quotation{Cinema Glory Days.}} At the time of this writing, the Globe Cinema has been marginalized, used only intermittently for local film screenings and other related creative events, while the multiplex cinema chains---heralded by locals in Trinidad and Tobago as \quotation{cultural} and technological advancement---have been the dominant means of accessing Hollywood films.\footnote{See Bobie-Lee Dixon, \quotation{National Literary Festival Has Longer Run, More Features,} {\em Trinidad and Tobago Guardian}, 21 March 2016, \useURL[url46][http://www.guardian.co.tt/article-6.2.352140.f3d3155555]\from[url46]; and Mark Fraser, \quotation{Facelift for Globe as New Owner Takes Charge,} {\em Trinidad Express}, 1 September 2013, \useURL[url47][https://trinidadexpress.com/news/local/facelift-for-globe-as-new-owner-takes-charge/article_f4b0d5b1-1445-5f47-8a55-b2782c7f3194.html]\from[url47]. See also Wilson, \quotation{Cinema Glory Days.}}

\placefigure{\#BringBackOurGlobeCinema}{\externalfigure[images/felix/fig6.png]}
Le Platte repurposed the popular image of the former US first lady and her appeal \#BringBackOurGirls, referencing the highly publicized kidnapping of two hundred Nigerian schoolgirls at the hands of the West African Boko Haram terrorist group in 2014. This was another attempt by Le Platte to demonstrate his sense of despondency regarding his current plight at the local multiplex. Given his unpleasant customer service experience at what might have been considered a professional and somewhat reputable establishment, Le Platte's meme constellation presents a critique of the local multiplex similar to an expression of dispiritedness against Boko Haram. Moreover, there are two additional points that underpin dimensions of Le Platte's critique: the Americanized aesthetic of the Movie Towne multiplex franchise and the theater's lack of general accessibility because of its location on the outskirts of the capital.\footnote{See Susan Lillian McFarlane-Alvarez, \quotation{Imaging and the National Imagining: Theorizing Visual Sovereignty in Trinidad and Tobago Moving Image Media through Analysis of Television Advertising} (PhD diss., Georgia State University, 2006), \useURL[url48][https://scholarworks.gsu.edu/communication_diss/3]\from[url48].} These points question perceived ideas regarding the sense of inclusiveness and \quotation{cultural} advancement multiplex cinemas were supposed to represent. The service offered by the multiplex franchise had not proved to be, in Le Platte's opinion, more favorable to that of the longstanding Globe Cinema in the heart of Port-of-Spain. His use of the Obama image in figure 6 was a nostalgic harkening to both history and heritage, especially given the place of Globe Cinema in local cultural memory.\footnote{See also Wilson, \quotation{Cinema Glory Days.}}

The \quotation{de-professionalized aesthetic} of internet memes is one of their defining features.\footnote{See Börzsei, \quotation{Makes a Meme Instead.}} A fifth key element is that Le Platte's background as a creative professional does not appear to have had much bearing on the creation of his meme series. It also does not appear that he employed any formal rules of art, graphic design, or \quotation{standard} English in the creation of the meme constellation posted to Facebook. In addition to the de-professionalization that is characteristic of alternative media, the use of language in Le Platte's work is also the byproduct of the local cultural context of the memes he generated.

Le Platte's cultural production of internet memes creates a unique tension between his professional designation as a known local creative industry practitioner and the \quotation{de-professionalized} output of his creative efforts online. The circulation and currency of Le Platte's creation of internet memes operates in keeping with Burgess and Green's work on the flow of cultural production within the digital landscape.\footnote{See Burgess and Green, {\em Youtube}.} In this respect, Le Platte's work is a prosumption-based activity that invalidates the professional-amateur binary, outlined earlier, since he produces and consumes knowledge in his contribution to online public discourse through his culture jamming practice.

\placefigure{Comments}{\externalfigure[images/felix/fig7.png]}
With this practice, Le Platte identifies with individuals across the \quotation{professional versus amateur} continuum in authorship of his meme series. Yet Le Platte's \quotation{power} as a creative professional does not (centrally) characterize his creative effort in making internet memes. I contend that his seemingly amateurish \quotation{practice of meming} invites public engagement with this work through its relative accessibility, given the popularity of the internet meme form. This is also in tandem with the communication of his personal experience, since Le Platte draws on the elements of language, identity, and cultural memory in an intelligible fashion, stirring online public discourse as a result of his efforts. As such, Le Platte's practice correlates with an alternative media participatory model outlined by Jan Servaes and Patchanee Malikhao, which emphasizes the necessity of public discourse and an appropriate framework for this interaction; this model is oppositional in that it provides institutional critique, with creative and varied reception from participants who engage in discourse of this nature. Le Platte's \quotation{practice(s) of meming} yield both viral and memetic output, according to Limor Shifman's definition(s), yet Servaes and Malikhao also present another important paradigm for understanding Le Platte's creative practice as alternative media.\footnote{See Jan Servaes and Patchanee Malikhao, \quotation{Participatory Communication: The New Paradigm,} in {\em Media and Glocal Change: Rethinking Communication for Development}, ed.~Oscar Hemer and Thomas Tufte (Buenos Aires: CLASCO, 2005), 91--103, \useURL[url49][http://bibliotecavirtual.clacso.org.ar/clacso/coediciones/20100824064944/09Chapter5.pdf]\from[url49]. On practices of meming, see Lopes, \quotation{Culture and Stigma.} See also Limor Shifman, \quotation{Humor in the Age of Digital Reproduction: Continuity and Change in Internet-Based Comic Texts,} {\em International Journal of Communication} 1 (2007): 187­--209, \useURL[url50][https://ijoc.org/index.php/ijoc/article/view/11]\from[url50]; Shifman, \quotation{Memes in a Digital World}; Limor Shifman, \quotation{The Cultural Logic of Photo-Based Meme Genres,} {\em Journal of Visual Culture} 13, no. 3 (2014): 340--58, \useURL[url51][https://doi.org/10.1177/1470412914546577]\from[url51]; Limor Shifman, \quotation{An Anatomy of a Youtube Meme,} {\em New Media and Society} 14, no. 2 (2012): 187--203, \useURL[url52][https://doi.org/10.1177/1461444811412160]\from[url52]; Segev, Nissenbaum, Stolero, and Shifman, \quotation{Families and Networks of Internet Memes}; and Shifman, Levy, and Thelwall, \quotation{Internet Jokes.}}

Le Platte's engagement via social media is a creative participatory process that involves a particular structural arrangement in the redistribution of power.\footnote{In addition to Servaes and Malikhao, \quotation{Participatory Communication,} see Fabian and Reestorff, \quotation{Mediatization and the Transformations of Cultural Activism}; and Sunny Sui-kwong Lam, \quotation{Multi-layered Identities by Social Media and Prosumption Practices in Digital and Participatory Communication,} {\em Journal of Digital Media and Interaction} 2, no. 3 (2019): 7--22, \useURL[url53][https://doi.org/10.34624/jdmi.v2i3.3747]\from[url53].} Drawing from Atton's topology of alternative media, I posit that the character of Le Platte's meme constellation features a marked degree of hybridity in an artistic, literary, and contextual sense.\footnote{See Atton, {\em Alternative Media}.} Le Platte evidences this hybridized display by his creative expression vis-à-vis language, identity, and cultural memory: using the image macro format, drawing from existing meme and popular culture references, the use of local dialect to inform the lexical dimension of his work and the dissemination of his internet meme series across a public and participatory forum such as Facebook. This last point is particularly noteworthy. Internet memes are found on a range of online social networks such as Twitter, Instagram, and YouTube; nevertheless, Facebook was selected by Le Platte as his prime platform for dissemination. This observation is in keeping with researchers who note Facebook as the preferred social media platform for Trinidad and Tobago netizens.\footnote{See Miller and Sinanan, {\em Visualising Facebook}; and Sinanan, {\em Social Media in Trinidad}.} This preference relates to not only the technical affordance of the platform itself but also the way it has been used reflexively by locals, which is one of many ways the use of Facebook can be represented.

Atton notes that alternative media provides discursive spaces for counternarratives, providing unique access into the lived experiences of others.\footnote{See Atton, {\em Alternative Media}; and Atton, \quotation{Introduction.}} Le Platte's internet meme constellation provides such access into his own experience as a Trinidad and Tobago national and member of the moviegoing public. Furthermore, Le Platte's role as a {\em prosumer} situates him, via social media, as a proxy for other prosumers---a result of the shared discourse that underpins his personal narrative. As such, Le Platte's culture jamming further mobilizes the participatory (and subversive) orientation of meme-making and online public discourse. Conceptually, the alternative media framework acknowledges the political act of self-publishing as a critical practice that resists commodification, democratizes public involvement, and allows for the creation of independent sites or \quotation{free spaces} for counter-discourse and experimentation.\footnote{See Atton, {\em Alternative Media}; and Robert W. Gehl, \quotation{The Case for Alternative Social Media,} {\em Social Media and Society} 1, no. 2 (2015), \useURL[url54][https://doi.org/10.1177/2056305115604338]\from[url54].} Le Platte's internet meme series manifests compatibility with this idea because his noncommercial work invites public engagement and dissemination among Facebook users, providing an institutional critique that reinscribes popular narratives of dysfunctionality and commess.

The technological affordance of Facebook's \quotation{free space} facilitated Le Platte's experimental image macro narrative and its counter-discursive content. Like the double use of pastiche and bricolage mentioned earlier, Le Platte's work presents an instance and example of \quotation{double alternativity}: the alternative practice of culture jamming or practice(s) of meming is reappropriated into a new disruptive form that functions as a participatory and subversive device. {\em Le Platte employs alternative media in an alternative way}. However, in this case, Le Platte's authorship appears to influence the cultural resonance of his meme constellation in the communication of public discourse. Le Platte's series is an atypical instance of double alternativity, in which, through the practice of culture jamming, he produces an articulation of alternative media that is further subverted, disrupted, and twice removed from its initial context, resulting in a novel form that departs from standard practices of counter-discursive action often seen in internet memes.

\subsection[title={Conclusion},reference={conclusion}]

Warren Le Platte's singular meme constellation---a single collection of memes having a unifying theme, structure, or object---presents a noteworthy example of alternative media in Trinidad and Tobago; it demonstrates how internet memes are vehicles for public discourse, given this local context, while also bearing a unique character in relation to common memetic forms. With a brand of humor based on Western popular culture and local parlance, Le Platte's series draws on aspects of local experience and practice, creating ongoing narratives of commess and dysfunctionality that provide an ongoing critique of institutions and governance. Stuart Hall notes two key ways of thinking about cultural identity and, by extension, cultural memory: as shared history and experience and as discontinuity and departure.\footnote{See Stuart Hall, \quotation{Cultural Identity and Diaspora,} in {\em Diaspora and Visual Culture: Representing Africans and Jews}, ed.~Nicholas Mirzoeff, 2nd. ed.~(Abingdon: Routledge, 2014), 21--33.} These two aspects of cultural identity and memory are but a few means of mobilizing public discourse through internet memes in Trinidad and Tobago.\footnote{See Ali, \quotation{\quote{Meanwhile in T&T} Discusses Internet Memes}; Miller and Sinanan, {\em Visualising Facebook}; and Sinanan, {\em Social Media in Trinidad}.}

Le Platte's work represents ordinary citizens looking toward local establishments to stay true to their commitments to the public. As such, it connotes and suggests the experience of a large sector of Trinidad and Tobago's population---a sector disillusioned by institutional rhetoric. This idea of discontinuity also translates into the formal aspects of Le Platte's meme constellation: reappropriated images establishing new referential contexts and the composition of proposed alternative narratives. Audiences, or prosumers, interpret these contexts and narratives in a manner that they are able to identify with and respond in ways that affirm the discourse(s) communicated through his work. As such, while these image macros follow the formal conventions of memes, the element of discontinuity is also present in the way their form and content are articulated in the local context.

Trinbagonian digital culture is not only the result of mobile and internet-based technology use. It is also a means of maintaining the contours of local identity, considering increasing globalization and especially the pervasive process of Americanization. Atton posits the nature of alternative media through the participatory practice of self-publishing, which offers opportunity for the construction of self-identity.\footnote{See Atton, \quotation{Introduction.}} Le Platte engaged in a self-published narrative of personal experience, which is linked to a broader sense of national identity. The participatory nature of social media, coupled with the discursive apparatus of the image macro enables locals such as Le Platte to exercise their own agency through self-publishing and, following Atton, provides avenues for \quotation{solidarity and the development of reflexivity in the creative processes of democratic media production.}\footnote{{[}{]}\{\#_Hlk51252827 .anchor\}Chris Atton, \quotation{Infoshops in the Shadow of the State,} in Couldry and Curran, {\em Contesting Media Power}, 59.} As such, the articulation of internet memes in the case of Warren Le Platte confirms the construction of narratives of dysfunctionality by presenting a reflexive critique through a self-published narrative.

Le Platte's work not only uses humor but also comments on and ridicules a particular establishment while also providing a light-hearted public reflection on national life through language, identity, and cultural memory. This reflection sees his work embrace of \quotation{commess culture} and dysfunction as an integral part of local identity. It is with this understanding of local identity that the proliferation and circulation of internet memes makes sense in this context. While local internet memes can and often do provide institutional critique, their social function makes light of serious and incensing matters, providing opportunity for cultural reflexivity or, in other words, a reflection on public life and local identity. Forms of alternative media, such as the culture jamming practiced by Le Platte, provide Trinidadians and Tobagonians with a sociocultural mirror, offering them the ability to see themselves, to participate in public discourse, and to exercise their agency in contributing to local online discourse across the digital public sphere.

\thinrule

\subsection[title={Bibliography},reference={bibliography}]

Ali, Shahad. \quotation{\quote{Meanwhile in T&T} Discusses Internet Memes.} {\em Trinidad Guardian}, 17 September 2012. \useURL[url55][https://guardian.co.tt/article-6.2.430946.7a94c69891]\from[url55].

Atton, Chris. {\em Alternative Media}. London: SAGE, 2001.

Atton, Chris. \quotation{A Brief History: The Web and Interactive Media.} In Coyer, Dowmunt, and Fountain, {\em Alternative Media Handbook}, 59--65.

Atton, Chris. \quotation{Infoshops in the Shadow of the State.} In Couldry and Curran, {\em Contesting Media Power: Alternative Media in a Networked World}, 57--69.

Atton, Chris. \quotation{Introduction: Problems and Positions in Alternative and Community Media.} In Atton, {\em Routledge Companion to Alternative and Community Media}, 1--18.

Atton, Chris, ed.~{\em The Routledge Companion to Alternative and Community Media}. Abingdon: Routledge, 2015.

Bagoo, Andre. \quotation{Owners Hope to Get \$25M for Property.} {\em Trinidad and Tobago Newsday}, 21 March 2013. \useURL[url56][https://archives.newsday.co.tt/2013/03/21/owners-hope-to-get-25m-for-property/]\from[url56].

Bagoo, Andre. \quotation{Richards: Not Sure It's Legal.} {\em Trinidad and Tobago Newsday}, 24 October 2014. \useURL[url57][https://archives.newsday.co.tt/2014/10/24/richards-not-sure-its-legal/]\from[url57].

Bailey, Olga Guedes, Bart Cammaerts, and Nico Carpentier. {\em Understanding Alternative Media}. Maidenhead, UK: Open University Press, 2008.

Barclay, Francis P., C. Pichandy, and Anusha Venkat. \quotation{India Elections 2014: Time-Lagged Correlation between Media Bias and Facebook Trend.} {\em Global Journal of Human-Social Science} 15, no. 2 (2015): 29--41.

Benkler, Yochai. \quotation{Sharing Nicely: On Shareable Goods and the Emergence of Sharing as a Modality of Economic Production.} In Mandiberg, {\em Social Media Reader}, 17--23. Originally published in {\em Yale Law Journal} 114, no. 2 (2004): 273--358. \useURL[url58][https://doi.org/10.2307/4135731]\from[url58].

Best, Curwen. {\em The Politics of Caribbean Cyberculture}. New York: Palgrave Macmillan, 2008. \useURL[url59][https://doi.org/10.1057/9780230610132]\from[url59].

\quotation{Board Game: Play Mas with \quote{Santimanitay,}} {\em Sweet TnT Magazine}, February 2018. \useURL[url60][https://sweettntmagazine.com/board-game-santimanitay/]\from[url60].

Börzsei, Linda K. \quotation{Makes a Meme Instead: A Concise History of Internet Memes.} {\em New Media Studies Magazine} 7 (2013). \useURL[url61][http://works.bepress.com/linda_borzsei/2/]\from[url61].

Brabham, Daren C. {\em Crowdsourcing}. Cambridge, MA: MIT Press, 2013.

Brideau, Kate, and Charles Berret. \quotation{A Brief Introduction to Impact: \quote{The Meme Font.}} {\em Journal of Visual Culture} 13, no. 3 (2014): 307--13. \useURL[url62][https://doi.org/10.1177/1470412914544515]\from[url62].

Burgess, Jean. \quotation{Hearing Ordinary Voices: Cultural Studies, Vernacular Creativity, and Digital Storytelling.} {\em Continuum} 20, no. 2 (2006): 201--14. \useURL[url63][https://doi.org/10.1080/10304310600641737]\from[url63].

Burgess, Jean, and Joshua Green. {\em Youtube: Online Video and Participatory Culture}. Oxford: Polity, 2009.

Campbell, Richard, Christopher R. Martin, and Bettina Fabos. {\em Media and Culture: An Introduction to Mass Communication}. 8th ed.~Boston: Bedford/St.~Marten's, 2012.

Chen, Carl. \quotation{The Creation and Meaning of Internet Memes in 4chan: Popular Internet Culture in the Age of Online Digital Reproduction.} {\em Habitus}, no. 3 (Spring 2012): 6--19.

Comor, Edward. \quotation{Digital Prosumption and Alienation.} {\em Ephemera} 10, nos. 3--4 (2010): 439--54. \useURL[url64][http://www.ephemerajournal.org/contribution/digital-prosumption-and-alienation]\from[url64].

Couldry, Nick, and James Curran, eds.~{\em Contesting Media Power: Alternative Media in a Networked World}. Lanham, MD: Rowman and Littlefield, 2003. \useURL[url65][https://lccn.loc.gov/2003004221]\from[url65].

Couldry, Nick, and James Curran. \quotation{The Paradox of Media Power.} In Couldry and Curran, {\em Contesting Media Power}, 3--16.

Coyer, Kate, Tony Dowmunt, and Alan Fountain, eds.~{\em The Alternative Media Handbook}. Abingdon: Routledge, 2011.

Davison, Patrick. \quotation{Because of the Pixels: On the History, Form, and Influence of MS Paint.} {\em Journal of Visual Culture} 13, no. 3 (2014): 275--97. \useURL[url66][https://doi.org/10.1177/1470412914544539]\from[url66].

Davison, Patrick. \quotation{The Language of Internet Memes.} In Mandiberg, {\em Social Media Reader}, 120--34.

Dixon, Bobie-Lee. \quotation{National Literary Festival Has Longer Run, More Features.} {\em Trinidad and Tobago Guardian}, 21 March 2016. \useURL[url67][http://www.guardian.co.tt/article-6.2.352140.f3d3155555]\from[url67].

Douglas, Nick. \quotation{It's Supposed to Look like Shit: The Internet Ugly Aesthetic.} {\em Journal of Visual Culture} 13, no. 3 (2014): 314--39. \useURL[url68][https://doi.org/10.1177/1470412914544516]\from[url68].

Fabian, Louise, and Camilla Møhring Reestorff. \quotation{Mediatization and the Transformations of Cultural Activism.} {\em Conjunctions} 2, no. 1 (2015): 1--20. \useURL[url69][https://doi.org/10.7146/tjcp.v2i1.22267]\from[url69].

Fenton, Natalie. \quotation{Mediating Hope: New Media, Politics, and Resistance.} {\em International Journal of Cultural Studies} 11, no. 2 (2008): 230--48. \useURL[url70][https://doi.org/10.1177/1367877908089266]\from[url70].

Fiske, John. \quotation{The Cultural Economy of Fandom.} In {\em The Adoring Audience: Fan Culture and Popular Media}, edited by Lisa A. Lewis, 30--49. London: Routledge, 1992.

Fiske, John. {\em Understanding Popular Culture}. London: Routledge, 1994.

Fraser, Mark. \quotation{Facelift for Globe as New Owner Takes Charge.} {\em Trinidad Express}, 1 September 2013. \useURL[url71][https://trinidadexpress.com/news/local/facelift-for-globe-as-new-owner-takes-charge/article_f4b0d5b1-1445-5f47-8a55-b2782c7f3194.html]\from[url71].

Fuchs, Christian. \quotation{Alternative Media as Critical Media.} {\em European Journal of Social Theory} 13, no. 2 (2010): 173--92. \useURL[url72][https://doi.org/10.1177/1368431010362294]\from[url72].

Fuchs, Christian. \quotation{Digital Prosumption Labour on Social Media in the Context of the Capitalist Regime of Time.} {\em Time and Society} 23, no. 1 (2014): 97--123. \useURL[url73][https://doi.org/10.1177/0961463X13502117]\from[url73].

Gehl, Robert W. \quotation{The Case for Alternative Social Media.} {\em Social Media and Society} 1, no. 2 (2015). \useURL[url74][https://doi.org/10.1177/2056305115604338]\from[url74].

Gelder, Ken. {\em Subcultures: Cultural Histories and Social Practice}. London: Routledge, 2007. \useURL[url75][https://doi.org/10.4324/9780203446850]\from[url75].

Guilbault, Jocelyne. \quotation{The Question of Multiculturalism in the Arts in the Postcolonial Nation-State of Trinidad and Tobago.} {\em Music and Politics} 5, no. 1 (2011). \useURL[url76][https://doi.org/10.3998/mp.9460447.0005.101]\from[url76].

Hall, Stuart. \quotation{Cultural Identity and Diaspora.} In {\em Diaspora and Visual Culture: Representing Africans and Jews}, edited by Nicholas Mirzoeff, 21--33. 2nd. ed.~Abingdon: Routledge, 2014.

Jenkins, Henry. \quotation{The Cultural Logic of Media Convergence.} {\em International Journal of Cultural Studies} 7, no. 1 (2004): 33--43. \useURL[url77][https://doi.org/10.1177/1367877904040603]\from[url77].

Jensen, Klaus Bruhn. \quotation{The Double Hermeneutics of Audience Research.} {\em Television and New Media} 20, no. 2 (2019): 142--54. \useURL[url78][https://doi.org/10.1177/1527476418811103]\from[url78].

Kidd, Dorothy. \quotation{The Global Movement to Transform Communications.} In Coyer, Dowmunt, and Fountain, {\em Alternative Media Handbook}, 239--48.

Knobel, Michele, and Colin Lankshear. \quotation{Online Memes, Affinities, and Cultural Production.} In Knobel and Lankshear, {\em New Literacies Sampler}, 199--227.

Knobel, Michele, and Colin Lankshear, eds.~{\em A New Literacies Sampler}. New York: Peter Lang, 2007.

Lam, Sunny Sui-kwong. \quotation{Multi-layered Identities by Social Media and Prosumption Practices in Digital and Participatory Communication.} {\em Journal of Digital Media and Interaction} 2, no. 3 (2019): 7--22. \useURL[url79][https://doi.org/10.34624/jdmi.v2i3.3747]\from[url79].

Lidwell, William, Kritina Holden, and Jill Butler. {\em Universal Principles of Design: 125 Ways to Enhance Usability, Influence Perception, Increase Appeal, Make Beter Design Decisions, and Teach through Design}. Beverly, MA: Rockport, 2010.

Lindo, Paula. \quotation{Board Games Trending Offline.} {\em Trinidad and Tobago Newsday}, 3 June 2020. \useURL[url80][https://newsday.co.tt/2020/06/03/board-games-trending-offline/]\from[url80].

Lopes, Paul. \quotation{Culture and Stigma: Popular Culture and the Case of Comic Books.} {\em Sociological Forum} 21, no. 3 (2006): 387--414. \useURL[url81][https://link.springer.com/article/10.1007/s11206-006-9022-6]\from[url81].

Mandiberg, Michael, ed.~{\em The Social Media Reader}. New York: New York University Press, 2012.

McAllister, Matthew P., Edward H. Sewell Jr., and Ian Gordon. \quotation{Introducing Comics and Ideology.} In {\em Comics and Ideology}, edited by Matthew P. McAllister, Edward H. Sewell Jr., and Ian Gordon, 1--13. New York: Peter Lang, 2001. \useURL[url82][http://www.peterlang.de/download/datasheet/46603/datasheet_69249.pdf]\from[url82].

McFarlane-Alvarez, Susan Lillian. \quotation{Imaging and the National Imagining: Theorizing Visual Sovereignty in Trinidad and Tobago Moving Image Media through Analysis of Television Advertising.} PhD dissertation, Georgia State University, 2006. \useURL[url83][https://scholarworks.gsu.edu/communication_diss/3]\from[url83].

Meikle, Graham. \quotation{Stop Signs: An Introduction to Culture Jamming.} In Coyer, Dowmunt, and Fountain, {\em Alternative Media Handbook}, 166--79.

Mendes-Franco, Janine. \quotation{Meme vs.~Movie Theatre---Dawn of Consumer Justice in Trinidad and Tobago?} {\em Global Voices}, 6 April 2016. \useURL[url84][https://globalvoices.org/2016/04/06/meme-vs-movie-theatre-dawn-of-consumer-justice-in-trinidad-tobago/\%23]\from[url84].

Miller, Daniel, and Jolynna Sinanan. {\em Visualising Facebook: A Comparative Perspective}. London: UCL Press, 2017. \useURL[url85][https://doi.org/10.2307/j.ctt1mtz51h]\from[url85].

Milner, Ryan M. \quotation{Hacking the Social: Internet Memes, Identity Antagonism, and the Logic of Lulz.} {\em Fibreculture Journal}, no. 22 (December 2013). FCJ-156. \useURL[url86][http://twentytwo.fibreculturejournal.org/fcj-156-hacking-the-social-internet-memes-identity-antagonism-and-the-logic-of-lulz/]\from[url86].

Milner, Ryan M. \quotation{Pop Polyvocality: Internet Memes, Public Participation, and the Occupy Wall Street Movement.} {\em International Journal of Communication} 7 (2013). \useURL[url87][https://ijoc.org/index.php/ijoc/article/view/1949]\from[url87].

Milner, Ryan M. \quotation{The World Made Meme: Discourse and Identity in Participatory Media.} PhD dissertation, University of Kansas, 2012. \useURL[url88][https://kuscholarworks.ku.edu/handle/1808/10256]\from[url88].

Milstein, Tema, and Alexis Pulos. \quotation{Culture Jam Pedagogy and Practice: Relocating Culture by Staying on One's Toes.} {\em Communication, Culture, and Critique} 8, no. 3 (2015): 395--413. \useURL[url89][https://doi.org/10.1111/cccr.12090]\from[url89].

Mowbray, Mike. \quotation{Alternative Logics? Parsing the Literature on Alternative Media.} In Atton, {\em Routledge Companion to Alternative and Community Media}, 21--31.

O'Neill, Brian, J. Ignacio Gallego Pérez, and Frauke Zeller. \quotation{New Perspectives on Audience Activity: \quote{Prosumption} and Media Activism as Audience Practices.} In {\em Audience Transformations: Shifting Audience Positions in Late Modernity}, edited by Nico Carpentier, Kim Christian Schrøder, and Lawrie Hallett, 157--71. New York: Routledge, 2013. \useURL[url90][https://doi.org/10.4324/9780203523162]\from[url90].

Paltrinieri, Roberta, and Piergiorgio Esposti. \quotation{Processes of Inclusion and Exclusion in the Sphere of Prosumerism.} {\em Future Internet} 5, no. 1 (2013): 21--33. \useURL[url91][https://doi.org/10.3390/fi5010021]\from[url91].

Polách, Vladimír P. \quotation{Memes, Trojan Horses, and the Discursive Power of Audience.} {\em Human Affairs} 25, no. 2 (2015): 189--203. \useURL[url92][https://doi.org/10.1515/humaff-2015-0017]\from[url92].

Rintel, Sean. \quotation{Crisis Memes: The Importance of Templatability to Internet Culture and Freedom of Expression.} {\em Australasian Journal of Popular Culture} 2, no. 2 (2013): 253--71. \useURL[url93][https://doi.org/10.1386/ajpc.2.2.253_1]\from[url93].

Rosen, Jay. \quotation{The People Formerly Known as the Audience.} In Mandiberg, {\em Social Media Reader}, 13--16.

Rutter, Richard, Stuart Roper, and Fiona Lettice. \quotation{Social Media Interaction, the University Brand and Recruitment Performance.} {\em Journal of Business Research} 69, no. 8 (2016): 3096--104. \useURL[url94][https://doi.org/10.1016/j.jbusres.2016.01.025]\from[url94].

Segev, Elad, Asaf Nissenbaum, Nathan Stolero, and Limor Shifman. \quotation{Families and Networks of Internet Memes: The Relationship Between Cohesiveness, Uniqueness, and Quiddity Concreteness.} {\em Journal of Computer-Mediated Communication} 20, no. 4 (2015): 417--33. \useURL[url95][https://doi.org/10.1111/jcc4.12120]\from[url95].

Servaes, Jan, and Patchanee Malikhao. \quotation{Participatory Communication: The New Paradigm.} In {\em Media and Glocal Change: Rethinking Communication for Development}, edited by Oscar Hemer and Thomas Tufte, 91--103. Buenos Aires: CLASCO, 2005. \useURL[url96][http://bibliotecavirtual.clacso.org.ar/clacso/coediciones/20100824064944/09Chapter5.pdf]\from[url96].

Shifman, Limor. \quotation{An Anatomy of a Youtube Meme.} {\em New Media and Society} 14, no. 2 (2012): 187--203. \useURL[url97][https://doi.org/10.1177/1461444811412160]\from[url97].

Shifman, Limor. \quotation{The Cultural Logic of Photo-Based Meme Genres.} {\em Journal of Visual Culture} 13, no. 3 (2014): 340--58. \useURL[url98][https://doi.org/10.1177/1470412914546577]\from[url98].

Shifman, Limor. \quotation{Humor in the Age of Digital Reproduction: Continuity and Change in Internet-Based Comic Texts.} {\em International Journal of Communication} 1 (2007): 187--209. \useURL[url99][https://ijoc.org/index.php/ijoc/article/view/11]\from[url99].

Shifman, Limor. \quotation{Memes in a Digital World: Reconciling with a Conceptual Troublemaker.} {\em Journal of Computer-Mediated Communication} 18, no. 3 (2013): 362--77. \useURL[url100][https://doi.org/10.1111/jcc4.12013]\from[url100].

Shifman, Limor, Hadar Levy, and Mike Thelwall. \quotation{Internet Jokes: The Secret Agents of Globalization?} {\em Journal of Computer-Mediated Communication} 19, no. 4 (2014): 727--43. \useURL[url101][https://doi.org/10.1111/jcc4.12082]\from[url101].

Sinanan, Jolynna. {\em Social Media in Trinidad: Values and Visibility}. London: UCL Press, 2017.

Tumolo, Maria. \quotation{An Interview with Warren Le Platte: Creator of the Board Game \quote{Santimanitay.}} {\em The Tiger Tales} (blog), 14 January 2018. \useURL[url102][https://thetigertales.co.uk/interview-warren-le-platte-creator-board-game-santimanitay/]\from[url102].

Varis, Piia, and Jan Blommaert. \quotation{Conviviality and Collectives on Social Media: Virality, Memes, and New Social Structures.} {\em Multilingual Margins} 2, no. 1 (2015): 31--45. \useURL[url103][https://doi.org/10.14426/mm.v2i1.55]\from[url103].

Wiggins, Bradley E., and G. Bret Bowers. \quotation{Memes as Genre: A Structurational Analysis of the Memescape.} {\em New Media and Society} 17, no. 11 (2015): 1886--1906. \useURL[url104][https://doi.org/10.1177/1461444814535194]\from[url104].

Wilson, Mark. \quotation{The Cinema Glory Days Are Coming Back.} {\em Caribbean Beat}, March--April 2002. \useURL[url105][https://www.caribbean-beat.com/issue-54/glory-days-coming-back\%23axzz6KH65Kv4z]\from[url105].

Zajc, Melita. \quotation{Social Media, Prosumption, and Dispositives: New Mechanisms of the Construction of Subjectivity.} {\em Journal of Consumer Culture} 15, no. 1 (2015): 28--47. \useURL[url106][https://doi.org/10.1177/1469540513493201]\from[url106].

\thinrule

Notes

\page
\subsection{Jonathan J. Felix}

Jonathan J. Felix is a transdisciplinary academic at the intersection of sociology, design, media, and cultural studies. His research interests include higher education, digital cultures, and alternative media. Jonathan is a Visiting Fellow with the International and Comparative Education Research Group with the Universiti Brunei Darussalam (ICE-UBD) and an Associate Fellow with the Higher Education Academy (AFHEA). He is also a member of Design Objective (DO), the International Communication Association (ICA), and the Media, Communications, and Cultural Studies Association (MeCCSA). With more than seventeen years of collective experience in the creative industries and higher education, he has worked independently and collaboratively in a range of roles related to teaching, leadership, and research. He currently develops and coordinates the delivery of communication literacies and research skills for courses in cultural studies, communication, and design theory, in addition to practical digital media skills. Favoring textual research methods and publishing in traditional and alternative modes of scholarship, his work is influenced by poststructuralist and continental philosophical schools of thought.

\stopchapter
\stoptext